% LaTeX source for ``Algorithms for Computer Simulation of Molecular Systems''
% Copyright (c) 2023 รังสิมันต์ เกษแก้ว (Rangsiman Ketkaew).

% License: Creative Commons Attribution-NonCommercial-NoDerivatives 4.0 International (CC BY-NC-ND 4.0)
% https://creativecommons.org/licenses/by-nc-nd/4.0/

{
% \pagenumbering{gobble}

\chapter*{\centering คำนำ}
\addcontentsline{toc}{chapter}{คำนำ}

การจำลองและคำนวณระบบโมเลกุลด้วยวิธีทางคอมพิวเตอร์นั้นช่วยให้นักวิทยาศาสตร์สามารถเข้าใจผลการทดลองในห้องปฏิบัติการ วิธีคำนวณทางเคมี%
ในระดับอะตอมนั้นสามารถแบ่งออกเป็นสองประเภทคือการคำนวณด้วยวิธีควอนตัม (Quantum Chemistry) และการคำนวณด้วยพลวัตเชิงโมเลกุล 
(Molecular Dynamics) ซึ่งทั้งสองวิธีนี้มีความแตกต่างกันอย่างสิ้นเชิงทั้งในแง่ทฤษฎี การใช้งานและความถูกต้องของการคำนวณ โดยวิธี Quantum 
Chemistry นั้นถูกนำมาใช้ในการศึกษาคุณสมบัติของโมเลกุลที่อธิบายด้วยโครงสร้างเชิงอิเล็กทรอนิกส์ซึ่งจะเกี่ยวข้องกับการคำนวณอันตรกิริยาระหว่าง%
อิเล็กตรอนโดยวิธีการประมาณ ในขณะที่วิธี Molecular Dynamics นั้นถูกนำมาใช้ศึกษาระบบโมเลกุลที่มีขนาดใหญ่และมีความซับซ้อนมากกว่าระบบ%
ที่โมเลกุลมีขนาดเล็กด้วยการจำลองการเคลื่อนที่ของอะตอมโดยอ้างอิงหลักกลศาสตร์แบบดั้งเดิม ซึ่งวิธีการนี้ช่วยให้เราศึกษาคุณสมบัติของโมเลกุลได้%
ในระดับของขนาดที่ใหญ่ขึ้นและระดับของระยะเวลาที่กว้างขึ้นด้วย

การจำลองทางคอมพิวเตอร์เพื่อศึกษาระบบโมเลกุลด้วยวิธี Molecular Dynamics แบบดั้งเดิมนั้นจะใช้แนวคิดของสนามแรง (Force Field) ซึ่งเป็น%
รูปแบบของฟังก์ชันทางคณิตศาสตร์ที่มีพารามิเตอร์ที่สามารถอธิบายพลังงานศักย์ของโมเลกุล (กลุ่มอะตอม) ซึ่งพลังงานศักย์ที่ได้มาจากการคำนวณ Force 
Field นี้คือสิ่งสำคัญที่เรามานำใช้ในการคำนวณแรงของอะตอมแต่ละอะตอมเพื่อใช้ในการหาโครงสร้างของโมเลกุล ณ จุดเวลาต่อ ๆ ไปได้ ซึ่งวิธี 
Molecular Dynamics แบบที่อ้างอิงกับ Force Field นั้นถูกนำมาใช้อย่างแพร่หลายในการศึกษาระบบต่าง ๆ เช่น การเปลี่ยนสถานะของสสาร 
ปฏิกิริยาเคมีหรือการเปลี่ยนแปลงเชิงโครงสร้างของชีวโมเลกุล นอกจากนี้การพัฒนา Force Field ก็เป็นหนึ่งในหัวข้องานวิจัยที่ได้รับความสนใจมาจน%
ถึงปัจจุบัน

แนวคิดของการรวมวิธีการคำนวณ Quantum Chemistry เข้ากับ Molecular Dynamics เพื่อเพิ่มความถูกต้องให้กับการคำนวณโดยการรวมนำข้อดี%
ของทั้งสองวิธีมารวมกันนั้นมีมานานมากกว่า 40 ปีแล้ว โดยวิธีการคำนวณที่ถูกพัฒนาขึ้นมาใหม่นี้มีชื่อเรียกว่า \textit{ab initio} Molecular 
Dynamics (AIMD) หรือพลวัตเชิงโมเลกุลแบบแอบ อินิชิโอ โดยในปัจจุบันวิธี AIMD นั้นมีความสำคัญและได้ถูกนำมาใช้อย่างแพร่หลายอย่างมากใน%
การจำลองระบบโมเลกุล โดยเฉพาะทางด้านวัสดุศาสตร์และฟิสิกส์ เช่น โครงสร้างผลึกและสภาวะของแข็งของโมเลกุล เพื่อศึกษาคุณสมบัติต่าง ๆ 
ของระบบเหล่านั้นก่อนที่จะมีการนำไปศึกษาจริงในห้องปฏิบัติการ ช่วยให้ประหยัดงบประมาณในการทำงานวิจัยได้อย่างมหาศาลและช่วยให้นักวิทยาศาสตร์%
เข้าใจถึงพฤติกรรมและอธิบายถึงปัจจัยที่ส่งผลต่อคุณสมบัติของโมเลกุลได้ด้วย 

เนื่องจากว่าในปัจจุบันนั้นแหล่งความรู้สำหรับการศึกษาอัลกอริทึมการจำลองทางคอมพิวเตอร์ของระบบโมเลกุลนั้นมักจะอยู่ในรูปของบทความวิชาการเป็น%
ส่วนใหญ่ ซึ่งบทความวิชาการเหล่านั้นส่วนใหญ่แล้วจะมีเนื้อหาที่ซับซ้อนและอาจจะทำให้ยากต่อการทำความเข้าใจของผู้ที่เพิ่มเริ่มศึกษา ยิ่งไปกว่านั้นผู้เขียน%
พบว่ายังไม่มีหนังสือหรือตำราทางวิชาการภาษาไทยที่เกี่ยวกับการจำลองทางคอมพิวเตอร์ของระบบโมเลกุล ดังนั้นหนังสือเล่มนี้จึงเป็นอีกช่องทางหนึ่งสำหรับ%
ผู้ที่สนใจและต้องการศึกษาอัลกอริทึมของ Molecular Dynamics เพื่อนำไปใช้ต่อยอดในงานวิจัยทางด้านเคมีทฤษฎีและเคมีเชิงคำนวณ รวมไปถึงสาขา%
ที่เกี่ยวข้องด้วย

สุดท้ายนี้ผู้เขียนหวังเป็นอย่างยิ่งว่าหนังสือเล่มนี้จะช่วยให้ผู้อ่านได้รับความรู้ที่ถูกต้องและครบถ้วนเกี่ยวกับวิธี AIMD ถ้าหากผู้อ่านมีข้อเสนอแนะ ข้อสงสัย%
หรือพบข้อผิดพลาดเกี่ยวกับเนื้อหาในหนังสือเล่มนี้สามารถติดต่อผู้เขียนได้โดยตรงผ่านทางอีเมล rangsiman1993[at]gmail[dot]com

\medskip

\begin{flushright}
รังสิมันต์ เกษแก้ว \\
12 มกราคม พ.ศ. 2566
\end{flushright}
}
