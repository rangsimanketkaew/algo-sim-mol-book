% LaTeX source for ``Ab initio Moleculardynamics''
% Copyright (c) 2023 รังสิมันต์ เกษแก้ว (Rangsiman Ketkaew).

% License: Creative Commons Attribution-NonCommercial-NoDerivatives 4.0 International (CC BY-NC-ND 4.0)
% https://creativecommons.org/licenses/by-nc-nd/4.0/

{
% \pagenumbering{gobble}

\chapter*{\centering คำนำ}
\addcontentsline{toc}{chapter}{คำนำ}

วิธีคำนวณทางเคมีในระดับอะตอมนั้นสามารถแบ่งออกเป็นสองประเภทคือการคำนวณด้วยวิธีควอนตัม (Quantum Chemistry) และการคำนวณด้วย%
พลวัตเชิงโมเลกุล (Molecular Dynamics) ซึ่งทั้งสองวิธีนี้มีความแตกต่างกันอย่างสิ้นเชิงทั้งในแง่ทฤษฎี การใช้งานและความถูกต้องของการคำนวณ 
โดยวิธี Quantum Chemistry นั้นถูกนำมาใช้ในการศึกษาคุณสมบัติของโมเลกุลที่อธิบายด้วยโครงสร้างเชิงอิเล็กทรอนิกส์ซึ่งจะเกี่ยวข้องกับการคำนวณ%
อันตรกิริยาระหว่างอิเล็กตรอนโดยวิธีการประมาณ ในขณะที่วิธี Molecular Dynamics นั้นถูกนำมาใช้ศึกษาระบบโมเลกุลที่มีขนาดใหญ่และมีความ%
ซับซ้อนมากกว่าระบบที่โมเลกุลมีขนาดเล็กด้วยการจำลองการเคลื่อนที่ของอะตอมโดยอ้างอิงหลักกลศาสตร์แบบดั้งเดิม ซึ่งวิธีการนี้ช่วยให้เราศึกษาคุณสมบัติ%
ของโมเลกุลได้ในระดับของขนาดที่ใหญ่ขึ้นและระดับของระยะเวลาที่กว้างขึ้นด้วย

แนวคิดของการรวมวิธีการคำนวณ Quantum Chemistry เข้ากับ Molecular Dynamics เพื่อเพิ่มความถูกต้องให้กับการคำนวณโดยการรวมนำข้อดี%
ของทั้งสองวิธีมารวมกันนั้นมีมานานมากกว่า 40 ปีแล้ว โดยวิธีการคำนวณที่เกิดขึ้นใหม่นี้มีชื่อเรียกว่า Ab initio Molecular Dynamics (AIMD) 
หรือพลวัตเชิงโมเลกุลแบบแอบ อินิชิโอ โดยในปัจจุบันวิธี AIMD นั้นมีความสำคัญและได้ถูกนำมาใช้อย่างแพร่หลายอย่างมากในการจำลองระบบโมเลกุล%
ทางเคมี ชีวะ และฟิสิกส์เพื่อศึกษาคุณสมบัติต่าง ๆ ของระบบเหล่านั้นก่อนที่จะมีการนำไปศึกษาจริงในห้องปฏิบัติการ ช่วยให้ประหยัดงบประมาณในการ%
ทำงานวิจัยได้อย่างมหาศาลและช่วยให้นักวิทยาศาสตร์เข้าใจถึงพฤติกรรมและอธิบายถึงปัจจัยที่ส่งผลต่อคุณสมบัติของโมเลกุลได้ด้วย 

หนังสือเล่มนี้จึงเป็นอีกช่องทางหนึ่งสำหรับผู้ที่สนใจและต้องการศึกษาวิธี AIMD เพื่อนำไปใช้ต่อยอดในงานวิจัยทางด้านเคมีทฤษฎีและเคมีเชิงคำนวณต่อไป

\medskip

\begin{flushright}
รังสิมันต์ เกษแก้ว \\
12 มกราคม พ.ศ. 2566
\end{flushright}
}
