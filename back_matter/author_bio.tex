% LaTeX source for ``Algorithms for Computer Simulation of Molecular Systems''
% Copyright (c) 2023 รังสิมันต์ เกษแก้ว (Rangsiman Ketkaew).

% License: Creative Commons Attribution-NonCommercial-NoDerivatives 4.0 International (CC BY-NC-ND 4.0)
% https://creativecommons.org/licenses/by-nc-nd/4.0/

{
\thispagestyle{empty}

\begin{center}
    \LARGE\textbf{ประวัติผู้เขียน}
\end{center}

รังสิมันต์ เกษแก้ว สำเร็จการศึกษาปริญญาตรี (พ.ศ. 2559) และปริญญาโท (พ.ศ. 2562) สาขาเคมี จากภาควิชาเคมี 
คณะวิทยาศาสตร์และเทคโนโลยี มหาวิทยาลัยธรรมศาสตร์ ปัจจุบันกำลังศึกษาปริญญาเอกสาขาเคมีทฤษฎีที่ภาควิชาเคมี มหาวิทยาลัยแห่งซูริค 
ประเทศสวิตเซอร์แลนด์ หัวข้องานวิจัยที่สนใจ ได้แก่ เคมีควอนตัม เมตาไดนามิกส์ การถ่ายโอนอิเล็กตรอน ปัญญาประดิษฐ์ 
และการพัฒนาซอฟต์แวร์เคมีเชิงคำนวณ

\leavevmode
\vspace{-1.5em}

\noindent ประสบการณ์การทำงาน
\vspace{-1em}
\begin{flushleft}
\begin{table}[htbp]
    \resizebox{\textwidth}{!}{%
    \begin{tabular}{ll}
        $\bullet$ พ.ศ. 2563 &ที่ปรึกษาบริษัท New Equilibrium Biosciences, Boston, MA \\
        $\bullet$ พ.ศ. 2564 &ที่ปรึกษาบริษัท ติ๊งกิ้ง แมชชีนส์ จำกัด (Thinking Machines) \\
        $\bullet$ พ.ศ. 2564 &คณะกรรมการจัดการแข่งขันปัญญาประดิษฐ์สําหรับเคมีแห่งประเทศไทย (TMLCC) \\
        $\bullet$ พ.ศ. 2564 &คณะกรรมการจัดงาน PyCon Thailand และ PyCon APAC 2021 \\
        $\bullet$ พ.ศ. 2565 &นักเขียนบทความบริษัท คลาวด์ เอชเอ็ม จำกัด (Cloud HM)
    \end{tabular}
    }
\end{table}
\end{flushleft}

\vspace{-2em}

\noindent ผู้ก่อตั้งเพจและกลุ่ม Facebook 
\vspace{-1em}
\begin{flushleft}
\begin{table}[htbp]
    \resizebox{0.7\textwidth}{!}{%
    \begin{tabular}{l}
        $\bullet$ \href{https://www.facebook.com/vitamin.sci}{วิทย์ตามิน} \\
        $\bullet$ \href{https://www.facebook.com/groups/354664889074376}{Computational Chemistry and Machine 
        Learning Thailand} \\
        $\bullet$ \href{https://www.facebook.com/groups/204730207310003}{Thai Computational Science Students}
    \end{tabular}
    }
\end{table}
\end{flushleft}

\begin{comment}
\vspace{-2em}

\noindent Playlists ของผู้เขียนบน YouTube
\vspace{-1em}
\begin{flushleft}
\begin{table}[htbp]
    \resizebox{0.85\textwidth}{!}{%
    \begin{tabular}{l}
        $\bullet$ \href{https://www.youtube.com/playlist?list=PLt-twymrmZ2eUPDfuXP6A7fbiCZygd-sa}{%
        Python for Scientific Computing - ไพธอนสำหรับการคำนวณทางวิทยาศาสตร์} \\
        $\bullet$ \href{https://www.youtube.com/playlist?list=PLt-twymrmZ2cQXM16ykndkD5Q5u-0qLtM}{%
        สอนเขียน Fortran สำหรับการคำนวณทางวิทยาศาสตร์} \\
        $\bullet$ \href{https://www.youtube.com/playlist?list=PLt-twymrmZ2f5aDzxlmVMKb0-EAkF0KwH}{%
        การเรียนรู้เชิงลึกสำหรับเคมี (TensorFlow)}
    \end{tabular}
    }
\end{table}
\end{flushleft}

\vspace{-1em}
\end{comment}

\noindent ดูบทความและผลงานเพิ่มเติมของผู้เขียนได้ที่ \url{https://rangsimanketkaew.github.io}

\vfill
}
