% LaTeX source for ``ปัญญาประดิษฐ์สำหรับเคมีควอนตัม (Machine Learning for Quantum Chemistry)''
% Copyright (c) 2023 รังสิมันต์ เกษแก้ว (Rangsiman Ketkaew).

% License: Creative Commons Attribution-NonCommercial-NoDerivatives 4.0 International (CC BY-NC-ND 4.0)
% https://creativecommons.org/licenses/by-nc-nd/4.0/

%%%%%%% List of commands: %%%%%%%
% Code block:
% \begin{lstlisting}[style=MyBash]{}
% \begin{lstlisting}[style=MyPython]{}
% \begin{lstlisting}[style=MyC++]{}
% \begin{lstlisting}[style=MyJSON]{}
% ----------------------------------
% Code in line:
% \bashinline{}
% \pyinline{}
% \cppinline{}
% \inlinehighlight{}
%%%%%%%%%%%%%%%%%%%%%%%%%%%%%%%%%

\usepackage{listings} % for code listing
\usepackage{xcolor} % for color

% define color
\colorlet{shadecolor}{gray!10} % increase the value will produce darker color
\colorlet{punct}{red!60!black}
\colorlet{numb}{magenta!60!black}
\definecolor{mymauve}{rgb}{0.58,0,0.82}
\definecolor{deepblue}{rgb}{0,0,0.5}
\definecolor{deepred}{rgb}{0.6,0,0}
\definecolor{deepgreen}{rgb}{0,0.5,0}
\definecolor{pythoncolor}{RGB}{102,102,255}

\lstset{
  backgroundcolor=\color{shadecolor},   % choose the background color; you must add \usepackage{color} or \usepackage{xcolor}; should come as last argument
  basicstyle=\ttfamily\normalsize\linespread{0.5},        % the size of the fonts that are used for the code
  keywordstyle=\color{blue}\ttfamily,       % keyword style
  commentstyle=\color{pink}\ttfamily,    	   % comment style
  breaklines=true,                 % sets automatic line breaking
  breakatwhitespace=true,         % sets if automatic breaks should only happen at whitespace
  captionpos=b,                    % sets the caption-position to bottom
  deletekeywords={...},            % if you want to delete keywords from the given language
  escapeinside={\%*}{*)},          % if you want to add LaTeX within your code
  extendedchars=true,              % lets you use non-ASCII characters; for 8-bits encodings only, does not work with UTF-8
%   firstnumber=1000,                % start line enumeration with line 1000
  frame=tlrb,	                   % adds a frame around the code, use a combination of t l r and b
  frameshape={RYR}{Y}{Y}{RYR},     % rounded corner
  keepspaces=true,                 % keeps spaces in text, useful for keeping indentation of code (possibly needs columns=flexible)
  columns=flexible,
%   basewidth={.88em},
  numbers=left,                    % where to put the line-numbers; possible values are (none, left, right)
  numbersep=10pt,                   % how far the line-numbers are from the code
  numberstyle=\normalsize\color{gray}, % the style that is used for the line-numbers
  rulecolor=\color{lightgray},         % if not set, the frame-color may be changed on line-breaks within not-black text (e.g. comments (green here))
  showspaces=false,                % show spaces everywhere adding particular underscores; it overrides 'showstringspaces'
  showstringspaces=false,          % underline spaces within strings only
  showtabs=false,                  % show tabs within strings adding particular underscores
  stepnumber=1,                    % the step between two line-numbers. If it's 1, each line will be numbered
  stringstyle=\color{mymauve},     % string literal style
  tabsize=4,	                   % sets default tabsize to 2 spaces
  title=\lstname,                   % show the filename of files included with \lstinputlisting; also try caption instead of title
  xleftmargin = 0.8cm,             % left margin for the code
  xrightmargin = -0.5cm,           % right margin for the code
  framexleftmargin = 2em,          % left margin for the whole frame
%   aboveskip=3mm,
  belowskip=-1.5 \baselineskip,
}

\lstdefinestyle{plain}{
    language=Bash,
    keywordstyle=\color{black},
    stringstyle=\color{black},
    commentstyle=\color{black},
}

\lstdefinestyle{MyBash}{
    language=Bash,
    keywordstyle=\color{blue},
    stringstyle=\color{black},
    commentstyle=\color{pink},
    morekeywords={}, % for letter 
    otherkeywords={}, % for non-letter
    deletekeywords={cd,echo,enable,export,jobs,local,source,test},
    morecomment=[l][\color{magenta}]{\#},
}

\lstdefinestyle{MyPython}{
    language=Python,
    keywordstyle=\color{deepblue},
    emph={MyClass,__init__},          % Custom highlighting
    emphstyle=\color{deepred},    % Custom highlighting style
    stringstyle=\color{deepgreen},
    commentstyle=\color{pink},
    morekeywords={self},              % Add keywords here
    morecomment=[l][\color{magenta}]{\#},
}

\lstdefinestyle{MyC++}{
    language=C++,
    keywordstyle=\color{blue},
    stringstyle=\color{red},
    commentstyle=\color{pink},
    morekeywords={},
    morecomment=[l][\color{magenta}]{\/\/},
}

\lstdefinestyle{MyFortran}{
    language=Fortran,
    keywordstyle=\color{blue},
    stringstyle=\color{red},
    commentstyle=\color{green},
    morecomment=[l][\color{magenta}]{!\ } % Comment only with space after !
}

\lstdefinelanguage{MyJSON}{
    stringstyle=\color{black},
    morestring=[b]",
    literate=
     *{0}{{{\color{numb}0}}}{1}
      {1}{{{\color{numb}1}}}{1}
      {2}{{{\color{numb}2}}}{1}
      {3}{{{\color{numb}3}}}{1}
      {4}{{{\color{numb}4}}}{1}
      {5}{{{\color{numb}5}}}{1}
      {6}{{{\color{numb}6}}}{1}
      {7}{{{\color{numb}7}}}{1}
      {8}{{{\color{numb}8}}}{1}
      {9}{{{\color{numb}9}}}{1}
      {:}{{{\color{punct}{:}}}}{1}
      {,}{{{\color{punct}{,}}}}{1}
      {\{}{{{\color{mymauve}{\{}}}}{1}
      {\}}{{{\color{mymauve}{\}}}}}{1}
      {[}{{{\color{mymauve}{[}}}}{1}
      {]}{{{\color{mymauve}{]}}}}{1},
}

\lstdefinestyle{nonumber}{
    number=none,
}

% for skipping lines with dots
% https://tex.stackexchange.com/questions/476658/how-to-skip-lines-in-lstlisting-with-dots
%------------------------
\let\origthelstnumber\thelstnumber
\makeatletter
\newcommand*\Suppressnumber{%
  \lst@AddToHook{OnNewLine}{%
    \let\thelstnumber\relax%
  }%
}

\newcommand*\Reactivatenumber{%
  \lst@AddToHook{OnNewLine}{%
   \let\thelstnumber\origthelstnumber%
  }%
}
\makeatother
%------------------------

% define code in line with highlighting
\newcommand{\bashinline}[1]{\colorbox{shadecolor}{\lstinline[style=MyBash]{#1}}}
\newcommand{\pyinline}[1]{\colorbox{shadecolor}{\lstinline[style=MyPython]{#1}}}
% inline highlight using a special color
\newcommand{\inlinehighlight}[1]{\colorbox{shadecolor}{\lstinline[
    style=MyPython,
    basicstyle=\ttfamily\small\linespread{0.5}\color{pythoncolor},
]{#1}}}
\newcommand{\cppinline}[1]{\colorbox{shadecolor}{\lstinline[style=MyC++]{#1}}}
