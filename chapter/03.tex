% LaTeX source for ``Algorithms for Computer Simulation of Molecular Systems''
% Copyright (c) 2023 รังสิมันต์ เกษแก้ว (Rangsiman Ketkaew).

% License: Creative Commons Attribution-NonCommercial-NoDerivatives 4.0 International (CC BY-NC-ND 4.0)
% https://creativecommons.org/licenses/by-nc-nd/4.0/

\chapter{พลวัตเชิงโมเลกุลแบบแอบ อินิชิโอ}
\label{ch:aimd}

%----------------------------------------
\section{ทำไมต้อง \textit{Ab initio} Molecular Dynamics}
%----------------------------------------

เทคนิคการจำลองแบบ Molecular Dynamics (MD) แบบดั้งเดิมหรือ Classical MD นั้นจะใช้ Potential ที่ได้มาจากการใช้ข้อมูลเชิงการทดลอง 
(Empirical Data) หรือจากการคำนวณ Electronic Structure และหัวใจสำคัญของ MD นั้นก็คือสมการที่ใช้ในการอธิบายอันตรกิริยาระหว่างอะตอม 
(Interatomic Interactions) โดยอันตรกิริยาที่เกิดขึ้นทั้งหมดนั้นเราสามารถแบ่งออกได้เป็นหลาย ๆ ส่วน คือ

\begin{itemize}[topsep=0pt,noitemsep]
    \setlength\itemsep{1em}
    \item Two-Body Contribution
    
    \item Three-Body Contribution
    
    \item Many-Body Contribution
    
    \item Long-Range Interaction
    
    \item Short-Range Interaction
    
    \item เทอมอื่น ๆ
\end{itemize}

จุดเริ่มต้นของเทคนิคการจำลองทางเคมีคอมพิวเตอร์แบบใหม่ที่เกิดขึ้นมาจากการนำวิธี MD และ Electronic Structure มารวมกันนั้นเรียกว่า 
\textit{ab initio} Molecular Dynamics (AIMD) ซึ่งอาจจะมีชื่อเรียกอื่น ๆ ที่เราอาจจะคุ้นเคยกันมาบ้าง เช่น 

\begin{itemize}[topsep=0pt,noitemsep]
    \setlength\itemsep{1em}
    \item Car-Parrinello Molecular Dynamics
    
    \item Hellmann-Feynman Molecular Dynamics
    
    \item First Principles Molecular Dynamics
    
    \item Quantum Chemical Molecular Dynamics
    
    \item On-The-Fly Molecular Dynamics
    
    \item Direct Molecular Dynamics
    
    \item Potential-Free Molecular Dynamics
    
    \item Quantum Molecular Dynamics
\end{itemize}

แต่ไอเดียพื้นฐานที่เป็นหัวใจสำคัญของวิธีการคำนวณแบบ AIMD ทุกวิธีนั้นก็คือการคำนวณแรง (Force) ที่กระทำระหว่างอะตอมโดยการใช้วิธี 
Electronic Structure ในแต่ละ Step ของการคำนวณ MD 

การประยุกต์ใช้วิธี AIMD นั้นกว้างขวางมาก ๆ โดยเฉพาะในด้านวัสดุศาสตร์และเคมี จะเห็นได้จากจำนวนงานบทความงานวิจัยเกี่ยวกับ AIMD 
ที่ได้รับการตีพิมพ์เพิ่มมากขึ้นเรื่อย ๆ ทุกปี ซึ่งจุดเริ่มต้นนั้นก็มาจากเปเปอร์ของ Car และ Parrinello ที่ตีพิมพ์ในปี 1985 นั่นคือ 
\enquote{Unified Approach for Molecular Dynamics and Density–Functional Theory} ที่ทำให้งานวิจัยทางด้านนี้นั้นได้รับความสนใจ 

อย่างไรก็ตาม ถึงแม้ว่าวิธี AIMD นั้นจะทำให้การคำนวณ MD นั้นมีความแม่นยำเพิ่มมากขึ้น แต่ว่าราคาที่นักคำนวณจะต้องจ่ายก็คือความสิ้นเปลืองในการคำนวณ 
(Computational Cost) ในการนำ MD ไปผสมรวมกับวิธี \textit{ab initio} นั่นก็คือความสัมพันธ์ระหว่าง Length และ Relaxation Time 
ที่เราสามารถรันการคำนวณด้วยแบบจำลอง AIMD นั้นสั้นมาก ๆ เมื่อเทียบกับวิธี MD ทั่วไป (สำหรับระบบโมเลกุลเดียวกัน) ถึงแม้ว่าข้อเสียของวิธี AIMD 
นั้นคือใช้เวลาในการคำนวณที่นานกว่า MD เยอะมาก ๆ แต่ว่าเราก็อย่าลืมไปว่าวิธี AIMD นั้นมีข้อดีอีกหลายข้อเลยที่เราจะไม่พูดถึงก็ไม่ได้ 
ข้อดีอย่างแรกก็คือวิธี AIMD สามารถให้ผลการคำนวณที่สอดคล้องกับ Physical Picture จริง ๆ ของระบบที่เราจำลอง ข้อดีอีกอย่างก็คือวิธี AIMD 
นั้นสามารถช่วยให้เราสามารถจำลองปรากฏการณ์ของระบบโมเลกุลที่ไม่สามารถเกิดขึ้นได้ในการจำลองด้วยวิธี MD

%----------------------------------------
\section{ประเภทของ \textit{Ab initio} Molecular Dynamics}
%----------------------------------------

เราสามารถแบ่งประเภทของ AIMD ได้โดยแบ่งตามวิธีการที่เราใช้ในการรวมการคำนวณโครงสร้างเชิงอิเล็กทรอนิกส์กับการจำลองพลวัตโมเลกุลเข้าด้วยกัน 
โดยเราสามารถแบ่งออกได้เป็น 3 วิธี ดังนี้ 

\begin{enumerate}[topsep=0pt,noitemsep]
    \setlength\itemsep{1em}
    \item Born-Oppenheimer Molecular Dynamics
    
    \item Ehrenfest Molecular Dynamics
    
    \item Car-Parrinello Molecular Dynamics
\end{enumerate}

%----------------------------------------
\section{พลวัตเชิงโมเลกุลแบบบอร์น-ออปเพนไฮเมอร์}
%----------------------------------------

%----------------------------------------
\section{พลวัตเชิงโมเลกุลแบบเออเรนเฟสต์}
%----------------------------------------

%----------------------------------------
\section{พลวัตเชิงโมเลกุลแบบคาร์-พาร์ริเนลโล}
%----------------------------------------

%----------------------------------------
\section{วิธีการเลือกใช้เทคนิค AIMD}
%----------------------------------------

%----------------------------------------
\section{การสุ่มตัวอย่างแบบมีประสิทธิภาพ}
%----------------------------------------

การสุ่มตัวอย่างแบบมีประสิทธิภาพ (Enhanced Sampling)

%----------------------------------------
\subsection{การคำนวณพลังงานอิสระ}
%----------------------------------------

%----------------------------------------
\subsection{เทคนิคเมตาไดนามิกส์}
%----------------------------------------

%----------------------------------------
\section{แบบฝึกหัด}
%----------------------------------------
