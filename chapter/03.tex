% LaTeX source for ``Algorithms for Computer Simulation of Molecular Systems''
% Copyright (c) 2023 รังสิมันต์ เกษแก้ว (Rangsiman Ketkaew).

% License: Creative Commons Attribution-NonCommercial-NoDerivatives 4.0 International (CC BY-NC-ND 4.0)
% https://creativecommons.org/licenses/by-nc-nd/4.0/

\chapter{พลวัตเชิงโมเลกุลแบบแอบ อินิชิโอ}
\label{ch:aimd}

%----------------------------------------
\section{ทำไมต้อง \textit{Ab initio} Molecular Dynamics}
%----------------------------------------

เทคนิคการจำลองแบบ Molecular Dynamics (MD) แบบดั้งเดิมหรือ Classical MD นั้นจะใช้ Potential ที่ได้มาจากการใช้ข้อมูลเชิงการทดลอง
(Empirical Data) หรือจากการคำนวณ Electronic Structure และหัวใจสำคัญของ MD นั้นก็คือสมการที่ใช้ในการอธิบายอันตรกิริยาระหว่างอะตอม
(Interatomic Interactions) โดยอันตรกิริยาที่เกิดขึ้นทั้งหมดนั้นเราสามารถแบ่งออกได้เป็นหลาย ๆ ส่วน คือ

\begin{itemize}[topsep=0pt,noitemsep]
  \setlength\itemsep{1em}
  \item Two-Body Contribution

  \item Three-Body Contribution

  \item Many-Body Contribution

  \item Long-Range Interaction

  \item Short-Range Interaction

  \item เทอมอื่น ๆ
\end{itemize}

จุดเริ่มต้นของเทคนิคการจำลองทางเคมีคอมพิวเตอร์แบบใหม่ที่เกิดขึ้นมาจากการนำวิธี MD และ Electronic Structure มารวมกันนั้นเรียกว่า
\textit{ab initio} Molecular Dynamics (AIMD) ซึ่งอาจจะมีชื่อเรียกอื่น ๆ ที่เราอาจจะคุ้นเคยกันมาบ้าง เช่น

\begin{itemize}[topsep=0pt,noitemsep]
  \setlength\itemsep{1em}
  \item Car-Parrinello Molecular Dynamics

  \item Hellmann-Feynman Molecular Dynamics

  \item First Principles Molecular Dynamics

  \item Quantum Chemical Molecular Dynamics

  \item On-The-Fly Molecular Dynamics

  \item Direct Molecular Dynamics

  \item Potential-Free Molecular Dynamics

  \item Quantum Molecular Dynamics
\end{itemize}

แต่ไอเดียพื้นฐานที่เป็นหัวใจสำคัญของวิธีการคำนวณแบบ AIMD ทุกวิธีนั้นก็คือการคำนวณแรง (Force) ที่กระทำระหว่างอะตอมโดยการใช้วิธี
Electronic Structure ในแต่ละ Step ของการคำนวณ MD

การประยุกต์ใช้วิธี AIMD นั้นกว้างขวางมาก ๆ โดยเฉพาะในด้านวัสดุศาสตร์และเคมี จะเห็นได้จากจำนวนงานบทความงานวิจัยเกี่ยวกับ AIMD
ที่ได้รับการตีพิมพ์เพิ่มมากขึ้นเรื่อย ๆ ทุกปี ซึ่งจุดเริ่มต้นนั้นก็มาจากเปเปอร์ของ Car และ Parrinello ที่ตีพิมพ์ในปี 1985 นั่นคือ
\enquote{Unified Approach for Molecular Dynamics and Density–Functional Theory} ที่ทำให้งานวิจัยทางด้านนี้นั้นได้รับความสนใจ

อย่างไรก็ตาม ถึงแม้ว่าวิธี AIMD นั้นจะทำให้การคำนวณ MD นั้นมีความแม่นยำเพิ่มมากขึ้น แต่ว่าราคาที่นักคำนวณจะต้องจ่ายก็คือความสิ้นเปลืองในการคำนวณ
(Computational Cost) ในการนำ MD ไปผสมรวมกับวิธี \textit{ab initio} นั่นก็คือความสัมพันธ์ระหว่าง Length และ Relaxation Time
ที่เราสามารถรันการคำนวณด้วยแบบจำลอง AIMD นั้นสั้นมาก ๆ เมื่อเทียบกับวิธี MD ทั่วไป (สำหรับระบบโมเลกุลเดียวกัน) ถึงแม้ว่าข้อเสียของวิธี AIMD
นั้นคือใช้เวลาในการคำนวณที่นานกว่า MD เยอะมาก ๆ แต่ว่าเราก็อย่าลืมไปว่าวิธี AIMD นั้นมีข้อดีอีกหลายข้อเลยที่เราจะไม่พูดถึงก็ไม่ได้
ข้อดีอย่างแรกก็คือวิธี AIMD สามารถให้ผลการคำนวณที่สอดคล้องกับ Physical Picture จริง ๆ ของระบบที่เราจำลอง ข้อดีอีกอย่างก็คือวิธี AIMD
นั้นสามารถช่วยให้เราสามารถจำลองปรากฏการณ์ของระบบโมเลกุลที่ไม่สามารถเกิดขึ้นได้ในการจำลองด้วยวิธี MD

จริง ๆ แล้วก่อนที่จะมีการพัฒนาวิธี AIMD ขึ้นมานั้น ในอดีตก็มีวิธีที่คล้าย ๆ กันเรียกว่า \enquote{Classical Trajectory Calculation}
ซึ่งมีจุดเริ่มต้นคือคำนวณระบบ Gas Phase ด้วยวิธี MD เพื่อศึกษา \enquote{Global} Potential Energy Surface (PES)
หรือพื้นผิวพลังงานศักย์ แล้วก็ตามด้วยการคำนวณ Dynamical Evolution ของระบบโดยการใช้ Classical Mechanics หรือ Quantum Mechanics
หรือ Semi/Quasiclassical Approximations ซึ่งในกรณีที่ใช้วิธี Classical Mechanics ในการอธิบาย Dynamics ของระบบโมเลกุลนั้น%
มีอุปสรรคก็คือขนาดของระบบ กล่าวคือ ยิ่งระบบมีขนาดใหญ่ การใช้วิธี Classical Mechanics นั้นก็จะยิ่งทำได้ยาก (สิ้นเปลืองการคำนวณ)
นั่นก็เพราะว่าระบบที่มี $N$ อะตอมนั้นก็จะมีจำนวนดีกรีของความอิสระ (Degree of Freedom) เท่ากับ $3N - 6$ ที่จะเป็นตัวกำหนดขนาดของ PES
แล้วถ้าหากว่าเราใช้จำนวน Discretization Points เช่น 10 Points ต่อ Coordinate นั่นคือเรามีจำนวนการคำนวณ Electronic Structure
ที่จะต้องคำนวณเท่ากับ $10^{3N - 6}$ เพื่อที่ทำการ Mapping เพื่อให้ได้ Global PES ของระบบโมเลกุลของเราออกมา
ดังนั้นความสิ้นเปลืองของวิธีแรกนั้นเท่ากับ $10^{N}$ ซึ่งเพิ่มตามขนาดของระบบ ซึ่งเราเรียกปัญหานี้ว่า \enquote{Dimensionality Bottleneck}

คำถามคือ \enquote{ถ้าหากอยากจะรู้ว่า AIMD สิ้นเปลืองแค่ไหน เราจะต้องคำนึงถึงอะไรบ้าง?} ในการตอบคำถามนี้ผมขอเริ่มด้วยการยกตัวอย่างต่อไปนี้
สมมติว่าเรามี Trajectory ของการคำนวณ MD ที่มีจำนวนทั้งหมด $10^{M}$ Steps (ก็คือมีทั้งหมด $10^{M}$ Configurations)
นั่นคือจะต้องมีการคำนวณ Electronic Structure ทั้งหมด $10^{M}$ ครั้ง ถ้าหากว่ามีจำนวน Independent Trajectory ทั้งหมด $10^{n}$
Trajectories ที่จำเป็นที่จะต้องคำนวณเพื่อทำการเฉลี่ย Initial Conditions ดังนั้นจึงมีการคำนวณ AIMD ทั้งหมด $10^{M + n}$
การคำนวณที่จะต้องทำการรัน ลำดับสุดท้าย ถ้าหากว่าเราจะต้องทำการคำนวณ Single-Point Electronic Structure เพื่อคำนวณ Global PES
และแต่ละการคำนวณของ AIMD นั้นใช้เวลา CPU Time เท่ากัน จากข้อมูลทั้งหมดที่เราอ้างขึ้นมาตามสถานการณ์ความเป็นจริงนั้น เราจะสรุปได้ว่าการใช้
AIMD ในการคำนวณ Global PES นั้นจะมีความสิ้นเปลืองอยู่ที่ประมาณ $10^{3N-6-M-n}$ ประเด็นสำคัญก็คือว่า สำหรับระบบที่มี $M$ และ $n$
คงที่และไม่ขึ้นกับ $N$ นั้น การคำนวณ AIMD จะมี Advantage เป็นแบบ \enquote{On-The-Fly} ซึ่งจะมีความสิ้นเปลืองของวิธีคือ $10^{N}$
เพิ่มขึ้นตามขนาดของระบบ

อย่างไรก็ตาม ความสิ้นเปลืองของวิธี AIMD ที่ประมาณ $10^{N}$ นั้นก็ยังเยอะอยู่ดี ดังนั้นจึงได้มีการพัฒนาเทคนิคต่าง ๆ ขึ้นมาเพื่อใช้ในการลดจำนวน
Degrees of Freedom แต่ว่าเทคนิคเหล่านั้นก็เป็นการใช้ Approximations (การประมาณ) เสียส่วนใหญ่ นั่นจึงทำให้ความถูกต้องของ AIMD นั้นลดลงด้วย

ในบทนี้ผมจะพาผู้อ่านทุกท่านไปทำความรู้จักกับวิธี AIMD ซึ่งเป็นภาพรวมกว้าง ๆ โดยเราจะเริ่มต้นกันด้วยสมการ Schr\"{o}dinger แล้วก็จะมีการพูดถึง
Classical MD, Ehrenfest MD, Born-Oppenheimer, และ Car-Parrinello MD ซึ่งเป็นวิธีที่ได้มาจากวิธี Time-Dependent Mean-Field
Approach ซึ่งได้หลังจากการที่เราทำการแยก Degrees of Freedom ของนิวเคลียสกับอิเล็กตรอนออกจากกัน นอกจากนี้ยังมีอีกหนึ่งเรื่องสำคัญที่ผู้อ่าน%
จะได้ศึกษานั่นคือการคำนวณแรง (Force) ของวิธีต่าง ๆ ด้วย ซึ่งหนึ่งในวิธีที่ถูกนำมาใช้ในการคำนวณ Force ที่ได้รับความนิยมเป็นอย่างมากนั่นคือ
Density Functional Theory (DFT) ดังนั้นเราจึงสามารถเรียกวิธีที่ใช้ในการจำลอง AIMD ที่มีความถูกต้องของการคำนวณ Electronic Structure
ในแต่ละ Step ของ MD ได้ว่า Density Functional Theory-based Molecular Dynamics หรือ DFT-MD

%----------------------------------------
\section{จาก MD สู่ \textit{Ab initio} MD}
%----------------------------------------

เรามาเริ่มกันที่สมการ Time-Dependent Schr\"{o}dinger Equation

\begin{equation}
  \label{eq:AIMD_TDSE}
  i \hbar \frac{\partial}{\partial t} \Phi\left(\mathbf{r}_i,\mathbf{R}_I ; t\right)
  =
  H \Phi\left(\mathbf{r}_i,\mathbf{R}_I ; t\right)
\end{equation}

\noindent ซึ่งมี Position Representation ที่เชื่อมโยงกับ Standard Hamiltonian ซึ่งมีนิยามดังต่อไปนี้ (มี5 เทอมรวมเข้าด้วยกัน)

\begin{equation}
  \label{eq:AIMD_Hamiltonian}
  \begin{aligned}
    H
     & = -\sum_I \frac{\hbar^2}{2 M_I} \nabla_I^2
    -\sum_i \frac{\hbar^2}{2 m_{\mathrm{e}}} \nabla_i^2
    +\sum_{i<j} \frac{\mathrm{e}^2}{\left|\mathbf{r}_i-\mathbf{r}_j\right|}
    -\sum_{I, i} \frac{\mathrm{e}^2 Z_I}{\left|\mathbf{R}_I-\mathbf{r}_i\right|}
    +\sum_{I<J} \frac{\mathrm{e}^2 Z_I Z_J}{\left|\mathbf{R}_I-\mathbf{R}_J\right|} \\
     & = -\sum_I \frac{\hbar^2}{2 M_I} \nabla_I^2
    -\sum_i \frac{\hbar^2}{2 m_{\mathrm{e}}} \nabla_i^2
    +V_{\mathrm{n}-\mathrm{e}}\left(\mathbf{r}_i,\mathbf{R}_I\right)                \\
     & = -\sum_I \frac{\hbar^2}{2 M_I} \nabla_I^2
    +H_{\mathrm{e}}\left(\mathbf{r}_i,\mathbf{R}_I\right)
  \end{aligned}
\end{equation}

\noindent สำหรับระดับของความอิสระ (Degrees of Freedom) ของอิเล็กทรอนิกส์ $\mathbf{r}_i$ กับของนิวเคลียร์ $\mathbf{R}_I$
แล้วเราก็จะใช้ Atomic Units (a.u.) เพื่อช่วยให้สมการต่าง ๆ ของเรานั้นดูง่ายและชัดเจนมากขึ้น ดังนั้นเราจะสนใจเทอมที่เป็นอันตรกิริยาระหว่าง
อิเล็กตรอน-อิเล็กตรอน (Electron-Electron), อิเล็กตรอน-นิวเคลียร์ (Electron-Nuclear), และนิวเคลียร์-นิวเคลียร์แบบคูลอมป์
(Nuclear-Nuclear Coulomb) เป็นพิเศษ

เป้าหมายของหัวข้อนี้ก็คือการพิสูจน์ที่มาของ Classical Molecular Dynamics โดยเริ่มจากสมการคลื่นของ Schr\"{O}dinger ซึ่งถึงตรงนี้แล้ว
เราจะทำการแยกฟังก์ชันคลื่นรวมของระบบโมเลกุลของเราออกเป็น 2 พาร์ท $\Phi\left(\mathbf{r}_i,\mathbf{R}_I ; t\right)$
นั่นก็คือพาร์ทที่ขึ้นกับ Nuclear Coordinates และพาร์ทที่ขึ้นกับ Electronic Coordinates ซึ่งจะสามารถเขียนให้อยู่ในรูปที่ง่ายที่สุดได้โดยใช้
ผลคูณระหว่าง Ansatz ดังนี้

\begin{equation}
  \label{eq:simplest_product_ansatz}
  \Phi\left(\mathbf{r}_i,\mathbf{R}_I ; t\right)
  \approx
  \Psi\left(\mathbf{r}_i ; t\right)
  \chi\left(\mathbf{R}_I ; t\right)
  \exp \left[\frac{i}{\hbar} \int_{t_0}^t d t^{\prime} \tilde{E}_{\mathrm{e}}\left(t^{\prime}\right)\right]
\end{equation}

\noindent โดยที่ Nuclear Wavefunction และ Electronic Wavefunction นั้นถูกแยกออกจากกันอย่างสิ้นเชิงและถูก Normalized ในแต่ละ
Time Step ด้วย ดังนี้ $\langle\chi ; t \mid \chi ; t\rangle=1$ และ $\langle\Psi ; t \mid \Psi ; t\rangle=1$ ตามลำดับ
นอกจากนี้แล้วเรายังมีการกำหนดพารามิเตอร์อีกตัวหนึ่งขึ้นมานั่นคือ Phase Factor ดังนี้

\begin{equation}
  \tilde{E}_{\mathrm{e}}
  =
  \int d \mathbf{r} d \mathbf{R} \Psi^{\star}\left(\mathbf{r}_i ; t\right)
  \chi^{\star}\left(\mathbf{R}_I ; t\right)
  H_{\mathrm{e}} \Psi\left(\mathbf{r}_i ; t\right)
  \chi\left(\mathbf{R}_I ; t\right)
\end{equation}

\noindent เพื่อที่จะทำให้สมการสุดท้ายที่เราได้ออกมานั้นมีหน้าตาที่ดูสั้นและกระชับ ดังนี้ $\int d \mathbf{r} d \mathbf{R}$ ซึ่งเป็นการคำนวณ
Integration ทั่วทั้งหมดโมเลกุล $i=1, \ldots$ และ $I=1, \ldots$ สำหรับตัวแปร $\mathbf{r}_i$ และ $\mathbf{R}_I$ ตามลำดับ

นอกจากนี้แล้วเราต้องทราบกันไว้ด้วยว่า Product Ansatz (ที่ไม่รวม Phase Factor) ตามด้านบนนั้นมีความแตกต่างจาก Ansatz ของ Born-Oppenheimer
ซึ่งเป็นการแยกพาร์ทที่คำนวณได้เร็วกว่าและพาร์มที่คำนวณได้ช้ากว่าออกจากกัน ดังนี้

\begin{equation}
  \Phi_{\mathrm{BO}}\left(\mathbf{r}_i,\mathbf{R}_I ; t\right)
  =
  \sum_{k=0}^{\infty}
  \tilde{\Psi}_k\left(\mathbf{r}_i,\mathbf{R}_I\right)
  \tilde{\chi}_k\left(\mathbf{R}_I ; t\right)
\end{equation}

ถ้าหากว่าเรานำสมการ Separation Ansatz ที่ \eqref{eq:simplest_product_ansatz} แทนเข้าไปในสมการที่ \eqref{eq:AIMD_TDSE}
และสมการที่ \eqref{eq:AIMD_Hamiltonian} (หลังจากที่เราทำการคูณทางด้านซ้ายของสมการด้วย  $\langle\Psi|$ และ $\langle\chi|$
และทำให้สอดคล้องตามเงื่อนไขของกฎอนุรักษ์พลังงาน $d\langle H\rangle / d t \equiv 0$ แล้ว) เราจะได้ความสัมพันธ์ต่อไปนี้

\begin{align}
  \label{eq:AIMD_coupled_TDSCF_1}
  i \hbar \frac{\partial \Psi}{\partial t}
   & =
  - \sum_i \frac{\hbar^2}{2 m_{\mathrm{e}}} \nabla_i^2 \Psi
  + \left\{\int d \mathbf{R} \chi^{\star}\left(\mathbf{R}_I ; t\right)
  V_{\mathrm{n}-\mathrm{e}}\left(\mathbf{r}_i,\mathbf{R}_I\right)
  \chi\left(\mathbf{R}_I ; t\right)\right\} \Psi \\
  \label{eq:AIMD_coupled_TDSCF_2}
  i \hbar \frac{\partial \chi}{\partial t}
   & =
  - \sum_I \frac{\hbar^2}{2 M_I} \nabla_I^2 \chi
  + \left\{\int d \mathbf{r} \Psi^{\star}\left(\mathbf{r}_i ; t\right)
  H_{\mathrm{e}}\left(\mathbf{r}_i,\mathbf{R}_I\right)
  \Psi\left(\mathbf{r}_i ; t\right)\right\} \chi
\end{align}

\noindent ซึ่งเซตของสมการที่พัวพันกัน (Coupled Equations) ตามด้านบนนี้เป็นตัวกำหนด Time-Dependent Self-Consistent Field
(TDSCF) ที่ได้มีการเสนอไว้เมื่อนานมาแล้วโดย Paul Dirac ในช่วงปี ค.ศ. 1930

%----------------------------------------
\section{มาเจาะลึก Classical Molecular Dynamics}
%----------------------------------------

ขึ้นตอนต่อไปในการพิสูจน์ Classical Molecular Dynamics ก็คือการประมาณและกำหนดให้นิวเคลียอของอะตอมนั้นเป็นอนุภาคจุด (Point
Particle) ซึ่งเราสามารถทำได้โดยการใช้พิสูจน์ Classical Mechanics ออกมาจาก Quantum Mechanics โดยเราจะเริ่มด้วยการเขียนฟังก์ชันคลื่น
(Wavefunction) ใหม่ ดังนี้

\begin{equation}
  \label{eq:AIMD_corresponding_wfn}
  \chi\left(\mathbf{R}_I ; t\right)
  =
  A\left(\mathbf{R}_I ; t\right) \exp \left[i S\left(\mathbf{R}_I ; t\right) / \hbar\right]
\end{equation}

\noindent โดยเราเขียนในเทอมของ Amplitude Factor $A$ และ Phase $S$ ซึ่งแฟคเตอร์ทั้งสองแฟคเตอร์นี้จะถูกพิจารณาเฉพาะค่าจริง (Real)
เท่านั้น $(A>0)$ และหลังจากที่เราทำการแปลง Nuclear Wavefunction (สมการที่ \eqref{eq:AIMD_coupled_TDSCF_2}) และทำการแยก
Real Part กับ Imaginary Part ออกจากกัน เราจะได้ TDSCF สำหรับนิวเคลียสดังต่อไปนี้

\begin{align}
  \label{eq:AIMD_phase_S}
   & \frac{\partial S}{\partial t}
  + \sum_I \frac{1}{2 M_I}\left(\nabla_I S\right)^2
  + \int d \mathbf{r} \Psi^{\star} H_{\mathrm{e}} \Psi
  =
  \hbar^2 \sum_I \frac{1}{2 M_I} \frac{\nabla_I^2 A}{A} \\
  \label{eq:AIMD_factor_A}
   & \frac{\partial A}{\partial t}
  + \sum_I \frac{1}{M_I}\left(\nabla_I A\right)\left(\nabla_I S\right)
  + \sum_I \frac{1}{2 M_I} A\left(\nabla_I^2 S\right)
  =
  0
\end{align}

\noindent ซึ่งเป็นการเขียนสมการเดิมโดยการใช้ตัวแปรใหม่สองอันก็คือ $A$ และ $S$ และเราเรียกสมการที่ \eqref{eq:AIMD_phase_S} และ
\eqref{eq:AIMD_factor_A} นี้ว่า \enquote{Quantum Fluid Dynamical Representation} ซึ่งเราสามารถสมการทั้งสองอันนี้ไปใช้ในการแก้
Time-Dependent Schr\"{o}dinger Equation นอกจากนี้เรายังสามารถเขียน $A$ ใหม่ได้ให้อยู่ในรูปของ Continuity Equation
ได้โดยการใช้คุณสมบัติ Identification ของ Nuclear Density $|\chi|^2 \equiv A^2$ ซึ่งสามารถคำนวณมาได้จากสมการที่
\eqref{eq:AIMD_corresponding_wfn} โดยสมการ Continuity Equation นี้จะไม่ขึ้นกับ $\hbar$ และยังทำให้มีการอนุรักษ์ Particle
Probability $|\chi|^2$ อีกด้วย

เมื่อเรามีการใช้ Transformation ของโมเมนตัมดังต่อไปนี้มาช่วย

\begin{equation}
  \mathbf{P}_I \equiv \nabla_I S
\end{equation}

\noindent เราจะสามารถเขียนสมการการเคลื่อนที่ของนิวตัน $\dot{\mathbf{P}}_I = -\nabla_I V\left(\mathbf{R}_I\right)$
ได้ดังต่อไปนี้

\begin{align}
  \label{eq:AIMD_Newtonian_1}
  \frac{d \mathbf{P}_I}{d t} & =-\nabla_I \int d \mathbf{r} \Psi^{\star} H_{\mathrm{e}} \Psi      \\
  \label{eq:AIMD_Newtonian_2}
  M_I \ddot{\mathbf{R}}_I(t) & =-\nabla_I \int d \mathbf{r} \Psi^{\star} H_{\mathrm{e}} \Psi      \\
  \label{eq:AIMD_Newtonian_3}
                             & =-\nabla_I V_{\mathrm{e}}^{\mathrm{E}}\left(\mathbf{R}_I(t)\right)
\end{align}

ดังนั้น นิวเคลียสนั้นจะเคลื่อนที่ไปตามหลักการของ Classical Mechanics ท่ามกลาง Effect Potentail $V_{\mathrm{e}}^{\mathrm{E}}$
ซึ่งมาจากอิเล็กตรอน โดย Potential ดังกล่าวนี้เป็นฟังก์ชันของ Nuclear Positions ณ เวลา $t$ ซึ่งได้มาจากการเฉลี่ย
$H_{\mathrm{e}}$ จาก Degrees of Freedom ทั้งหมด เช่น คำนวณค่า Quantum Expectation Value
$\left\langle\Psi\left|H_{\mathrm{e}}\right| \Psi\right\rangle$ ในขณะที่เราบังคับให้ตำแหน่งของนิวเคลียสนั้นถูก
Fixed อยู่กับที่นั่นคือ $\mathbf{R}_I(t)$

อย่างไรก็ตาม เรายังคงมีฟังก์ชันคลื่นของนิวเคลียส (Nuclear Wavefunction) เหลืออยู่ในสมการ TDSCF สำหรับ Degrees of Freedom ของอิเล็กตรอน
และเทอมนี้ควรจะต้องถูกแทนที่ด้วยตำแหน่งของนิวเคลียสเพื่อที่ว่าสมการ TDSCF นั้นจะขึ้นอยู่กับนิวเคลียสเพียงอย่างเดียว ในกรณีนี้เราจะทำการแทนที่
Nuclear Density $\left|\chi\left(\mathbf{R}_I ; t\right)\right|^2$ ในสมการก่อนหน้านี้โดยมีเงื่อนไขของลิมิตว่า
$\hbar \rightarrow 0$ โดยแทนที่ด้วยผลคูณของฟังก์ชันเดลต้า (Delta Functions)
$\prod_I \delta\left(\mathbf{R}_I-\mathbf{R}_I(t)\right)$ ที่มีตำแหน่งจุดกึ่งกลางของฟังก์ชันอยู่ที่ตำแหน่งของนิวเคลียส ณ
ขณะใดขณะหนึ่ง (Instantaneous Positions) $\mathbf{R}_I(t)$ ตามสมการที่ \eqref{eq:AIMD_Newtonian_2}
ซึ่งเราจะได้สมการดังต่อไปนี้ (สำหรับ Position Operator)

\begin{equation}
  \int d \mathbf{R}^{\star}\left(\mathbf{R}_I ; t\right) \mathbf{R}_I \chi\left(\mathbf{R}_I ; t\right)
  \stackrel{\hbar \rightarrow 0}{\longrightarrow}
  \mathbf{R}_I(t)
\end{equation}

โดย Classical Limit อันนี้จะนำไปสู่สมการ Time-Dependent Wave Equation สำหรับอธิบายอิเล็กตรอนดังต่อไปนี้

\begin{align}
  \label{eq:AIMD_TD_wave_electron_1}
  i \hbar \frac{\partial \Psi}{\partial t}
   & = -\sum_i \frac{\hbar^2}{2 m_{\mathrm{e}}} \nabla_i^2 \Psi
  + V_{\mathrm{n}-\mathrm{e}}\left(\mathbf{r}_i,\mathbf{R}_I(t)\right) \Psi \\
  \label{eq:AIMD_TD_wave_electron_2}
   & = H_{\mathrm{e}}\left(\mathbf{r}_i,\mathbf{R}_I(t)\right)
  \Psi\left(\mathbf{r}_i,\mathbf{R}_I ; t\right)
\end{align}

\noindent ซึ่งสมการด้านบนนี้มีความซับซ้อนเพราะว่าเป็นสมการที่มันขึ้นอยู่กับตัวของมันเอง (Self-Consistently) นอกจากนี้แล้ว
$H_{\mathrm{e}}$ และ $\Psi$ นั้นจะขึ้นต่อกันแบบเชิงพารามิเตอร์หรือเรียกว่าอิงพารามิเตอร์ก็ได้ (Parametrically) ตามตำแหน่งของนิวเคลียส
$\mathbf{R}_I(t)$ ที่เวลา $t$ โดยผ่าน $V_{\mathrm{n}-\mathrm{e}}\left(\mathbf{r}_i,\mathbf{R}_I(t)\right)$
นั่นหมายความว่าทั้ง Classical และ Quantum Degress of Freedom นั้นได้ถูกรวมเข้าด้วยกันแล้ว

สำหรับวิธีที่เกี่ยวข้องกับการแก้สมการที่ \eqref{eq:AIMD_Newtonian_2} และ \eqref{eq:AIMD_TD_wave_electron_2} นั้นมีชื่อเรียกว่า
\enquote{Ehrenfest Molecular Dynamics} เพื่อเป็นเกียรติให้กับ Ehrenfest ผู้ที่เป็นคนแรกที่สามารถหาวิธีแก้ปัญหาสำหรับคำถามที่ว่า
\enquote{เราจะสามารถพิสูจน์ Newtonian Classical Dynamics จากสมการคลื่นของ Schr\"{o}dinger ได้อย่างไร?}
ซึ่งนำไปสู่การพัฒนาวิธีการแบบผสม (Hybrid) นั่นก็คือมีเพียงแค่นิวเคลียสเท่านั้นที่ถูกบังคับให้มีพฤติกรรมที่ทำตัวคล้ายกับอนุภาค Classical
Particles ในขณะที่อิเล็กตรอนนั้นยังถูกอธิบายด้วยวิธีทางควอนตัม

ตามที่ได้กล่าวไว้ในตอนต้นของหัวข้อนี้แล้วว่าวิธี MD นั้นมีปัญหาอย่างหนึ่งที่หลีกเลี่ยงไม่ได้ก็คือ Dimensionality Bottleneck ซึ่งจะเพิ่มมากขึ้นตาม
Degrees of Freedom ของนิวเคลียส (จำนวนอะตอม) ซึ่งหนึ่งในวิธีที่เป็นทางออกของปัญหานี้ก็คือการทำการประมาณ Global PES ซึ่งมีสมการคือ

\begin{align}
  V_{\mathrm{e}}^{\mathrm{E}} \approx V_{\mathrm{e}}^{\mathrm{approx}}\left(\mathbf{R}_I\right)
  = &
  \sum_{I=1}^N v_1\left(\mathbf{R}_I\right)+\sum_{I<J}^N v_2\left(\mathbf{R}_I, \mathbf{R}_J\right) \nonumber \\
    & +\sum_{I<J<K}^N v_3\left(\mathbf{R}_I, \mathbf{R}_J, \mathbf{R}_K\right) + \cdots
\end{align}

\noindent โดยเขียนในรูปของการกระจายของเทอมที่เกิดจาก Contribution แบบ Many-Body (Truncated Expansion) และเราจะทำการแทน
Degrees of Freedom เชิงอิเล็กทรอนิกส์ด้วย Interaction Potentials $\left\{v_n\right\}$ ซึ่งจะทำให้เทอมที่เป็น Degrees of
Freedom นั้นหายไป ดังนั้นสิ่งที่เราจะได้ออกมาก็คือสมการที่เป็นลูกผสมระหว่างปัญหาทางควอนตัมและปัญหาแบบคลาสสิกนั้นจะถูกลดรูปให้เหลือเป็น%
เพียงแค่ปัญหาแบบคลาสสิกเท่านั้น (ปัญหาแบบควอนตัมหายไปแล้ว) ซึ่งจะได้ว่า Classical Molecular Dynamics แบบที่เราต้องการนั้นจะมีหน้าตาคือ

\begin{equation}
  M_I \ddot{\mathbf{R}}_I(t)
  =
  -\nabla_I V_{\mathrm{e}}^{\text {approx }}\left(\mathbf{R}_I(t)\right)
\end{equation}

เราสามารถแบ่งประเภทของ AIMD ได้โดยแบ่งตามวิธีการที่เราใช้ในการรวมการคำนวณโครงสร้างเชิงอิเล็กทรอนิกส์กับการจำลองพลวัตโมเลกุลเข้าด้วยกัน
โดยเราสามารถแบ่งออกได้เป็น 3 วิธี ดังนี้

\begin{enumerate}[topsep=0pt,noitemsep]
  \setlength\itemsep{1em}
  \item Ehrenfest Molecular Dynamics

  \item Born-Oppenheimer Molecular Dynamics

  \item Car-Parrinello Molecular Dynamics
\end{enumerate}

%----------------------------------------
\section{พลวัตเชิงโมเลกุลแบบเออเรนเฟสต์}
%----------------------------------------

แก้ชุดสมการดังต่อไปพร้อม ๆ กัน

\begin{align}
  M_I \ddot{\mathbf{R}}_I(t)
   & = -\nabla_I \int d \mathbf{r} \Psi^{\star} H_{\mathrm{e}} \Psi                                                       \\
   & = -\nabla_I\left\langle\Psi\left|H_{\mathrm{e}}\right| \Psi\right\rangle                                             \\
   & = -\nabla_I\left\langle H_{\mathrm{e}}\right\rangle                                                                   \\
  \label{eq:Ehrenfest_coupled_eq_1}
   & = -\nabla_I V_{\mathrm{e}}^{\mathrm{E}}                                                                              \\
  i \hbar \frac{\partial \Psi}{\partial t}
   & = \left[-\sum_i \frac{\hbar^2}{2 m_{\mathrm{e}}}
  \nabla_i^2+V_{\mathrm{n}-\mathrm{e}}\left(\left\{\mathbf{r}_i\right\},\left\{\mathbf{R}_I(t)\right\}\right)\right] \Psi \\
  \label{eq:Ehrenfest_coupled_eq_2}
   & = H_{\mathrm{e}} \Psi
\end{align}

\noindent แล้วเราก็พบว่าเราไม่จำเป็นที่จะต้องทราบ Potential Energy Surface (PES) ก็สามารถแก้สมการ Time-Dependent Electronic
Schr\"{o}dinger ได้แบบ \enquote{On-The-Fly} (หมายความว่า อยากแก้สมการเมื่อไหร่ก็ได้) ซึ่งทำให้เรานั้นสามารถคำนวณแรง (Force)
จาก $\nabla_I\left\langle H_{\mathrm{e}}\right\rangle$ สำหรับแต่ละคอนฟิกูเรชั่น $\left\{\mathbf{R}_I(t)\right\}$
ของโมเลกุลที่ได้มาจาก Trajectory ที่จำลองมาจากวิธี Molecular Dynamics ได้ด้วย (แรงชนิดนี้มีชื่อเรียกอีกชื่อว่า \enquote{Hellmann-Feynman
  Forces}

นอกจากนี้แล้ว ยังมีสมการของ Equations of Motion ที่อยู่ในรูปของ Adiabatic Basis และ Time-Dependent Expansion Coefficients
อีกด้วย ซึ่งมีหน้าตาดังนี้

\begin{align}
  M_I \ddot{\mathbf{R}}_I(t) &
  =
  -\sum_k\left|c_k(t)\right|^2 \nabla_I E_k-\sum_{k, l} c_k^{\star} c_l\left(E_k-E_l\right) \mathbf{d}_I^{k l} \\
  i \hbar \dot{c}_k(t)
                             & = c_k(t) E_k-i \hbar \sum_{I, l} c_l(t) \dot{\mathbf{R}}_I \mathbf{d}_I^{k l}
\end{align}

\noindent ซึ่งมี Coupling Term ดังนี้

\begin{equation}
  \mathbf{d}_I^{k l}\left(\left\{\mathbf{R}_I(t)\right\}\right)
  =
  \int d \mathbf{r} \Psi_k^{\star} \nabla_I \Psi_l
\end{equation}

\noindent แล้วก็มี Property $\mathbf{d}_I^{k k} \equiv \mathbf{0}$

ดังนั้น เราอาจจะมองว่าวิธีการของ Ehrenfest นั้นก็คือเป็นการรวม Non-Adiabatic Transitions ระหว่าง Electronic States ที่แตกต่างกัน
นั่นคือ $\Psi_k$ และ $\Psi_l$ โดยรวมเข้ามาไว้ใน Framework ของ Mean-Field (TDSCF) Approximation นั่นเอง
นอกจากนี้แล้วเรายังสามารถกำหนดเงื่อนไข (Restriction) สำหรับ Electronic State แต่ละอันได้อีกด้วย (ซึ่งกรณีส่วนใหญ่นั่นจะเป็นการอธิบาย
$\Psi_0$ ที่สภาวะพื้น) ซึ่งก็จะทำให้เราได้สมการที่เป็นสำหรับกรณีพิเศษของสมการที่ \eqref{eq:Ehrenfest_coupled_eq_1} และ
\eqref{eq:Ehrenfest_coupled_eq_2} ดังนี้

\begin{align}
  M_I \ddot{\mathbf{R}}_I(t)
   & =
  -\nabla_I\left\langle\Psi_0\left|H_{\mathrm{e}}\right| \Psi_0\right\rangle \\
  i \hbar \frac{\partial \Psi_0}{\partial t}
   & = H_{\mathrm{e}} \Psi_0
\end{align}

\noindent โดยที่ $H_{\mathrm{e}}$ คือ Time-Dependent Hamiltonian ของ Nuclear Coordinates
$\left\{\mathbf{R}_I(t)\right\}$

%----------------------------------------
\section{พลวัตเชิงโมเลกุลแบบบอร์น-ออปเพนไฮเมอร์}
%----------------------------------------

%----------------------------------------
\section{พลวัตเชิงโมเลกุลแบบคาร์-พาร์ริเนลโล}
%----------------------------------------

%----------------------------------------
\section{วิธีการเลือกใช้เทคนิค AIMD}
%----------------------------------------

%----------------------------------------
\section{การสุ่มตัวอย่างแบบมีประสิทธิภาพ}
%----------------------------------------

การสุ่มตัวอย่างแบบมีประสิทธิภาพ (Enhanced Sampling)

%----------------------------------------
\subsection{การคำนวณพลังงานอิสระ}
%----------------------------------------

%----------------------------------------
\subsection{เทคนิคเมตาไดนามิกส์}
%----------------------------------------

%----------------------------------------
\section{แบบฝึกหัด}
%----------------------------------------
