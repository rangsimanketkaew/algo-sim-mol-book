% LaTeX source for ``Algorithms for Computer Simulation of Molecular Systems''
% Copyright (c) 2023 รังสิมันต์ เกษแก้ว (Rangsiman Ketkaew).

% License: Creative Commons Attribution-NonCommercial-NoDerivatives 4.0 International (CC BY-NC-ND 4.0)
% https://creativecommons.org/licenses/by-nc-nd/4.0/

\chapter{พลวัตเชิงโมเลกุลแบบแอบ อินิชิโอ}
\label{ch:aimd}

%----------------------------------------
\section{ทำไมต้อง \textit{Ab initio} Molecular Dynamics}
%----------------------------------------


%----------------------------------------
\section{ประเภทของ \textit{Ab initio} Molecular Dynamics}
%----------------------------------------

เราสามารถแบ่งประเภทของ AIMD ได้โดยแบ่งตามวิธีการที่เราใช้ในการรวมการคำนวณโครงสร้างเชิงอิเล็กทรอนิกส์กับการจำลองพลวัตโมเลกุลเข้าด้วยกัน 
โดยเราสามารถแบ่งออกได้เป็น 3 วิธี ดังนี้ 

\begin{enumerate}
    \item Born-Oppenheimer Molecular Dynamics
    \item Ehrenfest Molecular Dynamics
    \item Car-Parrinello Molecular Dynamics
\end{enumerate}

%----------------------------------------
\section{พลวัตเชิงโมเลกุลแบบบอร์น-ออปเพนไฮเมอร์}
%----------------------------------------

%----------------------------------------
\section{พลวัตเชิงโมเลกุลแบบเออเรนเฟสต์}
%----------------------------------------

%----------------------------------------
\section{พลวัตเชิงโมเลกุลแบบคาร์-พาร์ริเนลโล}
%----------------------------------------

%----------------------------------------
\section{วิธีการเลือกใช้เทคนิค AIMD}
%----------------------------------------

%----------------------------------------
\section{แบบฝึกหัด}
%----------------------------------------
