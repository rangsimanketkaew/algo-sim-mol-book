% LaTeX source for ``Algorithms for Computer Simulation of Molecular Systems''
% Copyright (c) 2023 รังสิมันต์ เกษแก้ว (Rangsiman Ketkaew).

% License: Creative Commons Attribution-NonCommercial-NoDerivatives 4.0 International (CC BY-NC-ND 4.0)
% https://creativecommons.org/licenses/by-nc-nd/4.0/

\chapter{พลวัตเชิงโมเลกุลแบบดั้งเดิม}
\label{ch:md}

%----------------------------------------
\section{การประยุกต์ใช้ Molecular Dynamics}
%----------------------------------------

ก่อนที่เราจะไปศึกษาวิธีการจำลองระบบโมเลกุลที่มีความซับซ้อนนั้นเราควรเริ่มต้นด้วยการศึกษาวิธีอย่างง่ายก่อนนั่นก็คือ Molecular Dynamics (MD)
(จริง ๆ จะว่าไปแล้ววิธีนี้ก็ไม่ได้ง่ายนะครับ ถ้าลงรายละเอียดลึก ๆ แล้วก็มีความซับซ้อนมากพอสมควร ซึ่งในปัจจุบันนั้นก็มีการทำวิจัยที่เกี่ยวข้องกับ%
การพัฒนาเทคนิคของวิธี MD อย่างต่อเนื่อง) เราใช้เทคนิคการจำลองทางคอมพิวเตอร์ในการทำความเข้าใจคุณสมบัติของระบบที่ประกอบไปด้วยโมเลกุล
ๆ โมเลกุลในเชิงโครงสร้างและอันตรกิริยาในระดับจุลภาคหรือหน่วยย่อย ซึ่งหน่วยย่อยในที่นี้ก็คือโมเลกุลนั่นเอง โดยเราสามารถแบ่งวิธีการจำลองออก%
ได้เป็น 2 วิธีคือ Molecular Dynamics (MD) กับ Monte Carlo (MC) และหนังสือเล่มนี้จะเน้นไปที่ MD ซึ่งเป็นวิธีที่สามารถนำมาใช้ในการศึกษา%
คุณสมบัติเชิงพลวัตได้ เช่น สัมประสิทธิ์การเคลื่อนที่ (Transport Coefficients), การตอบสนองต่อการรบกวนแบบที่ขึ้นกับเวลา (Time-dependt
Response to Perturbation), และสเปกตรัม

ตัวอย่างของคุณสมบัติและปรากฏการณ์ของโมเลกุลหรือสสารที่เราสามารถใช้การจำลอง MD เพื่อศึกษาได้นั้นมีดังต่อไปนี้

\noindent \textbf{เคมี (Chemistry)}

\begin{itemize}[topsep=0pt,noitemsep]
    \setlength\itemsep{0.5em}
    \item อันตรกิริยาภายในและภายนอกโมเลกุล (Intra- and Intermolecular Interactions)

    \item ปฏิกิริยาเคมี (Chemical Reactions)

    \item การเปลี่ยนเฟส (Phase Transitions)

    \item การคำนวณพลังงานอิสระ (Free Energy Calculations)
\end{itemize}

\noindent \textbf{วัสดุศาสตร์ (Materials Science)}

\begin{itemize}[topsep=0pt,noitemsep]
    \setlength\itemsep{0.5em}
    \item เทอร์โมไดนามิกส์ที่สภาวะสมดุล (Equilibrium Thermodynamics)

    \item การเปลี่ยนเฟส (Phase Transitions)

    \item Properties of Lattice Defects

    \item Nucleation and Surface Growth

    \item กระบวนการความร้อนและความ (Heat/Pressure Processing)

    \item Ion Implantation

    \item Properties of Nanostructures
\end{itemize}

\noindent \textbf{ชีวเชิงฟิสิกส์และชีวเคมี (Biophysics and Biochemistry)}

\begin{itemize}[topsep=0pt,noitemsep]
    \setlength\itemsep{0.5em}
    \item การพับของโปรตีน (Protein Folding)

    \item การทำนายโครงสร้างของโปรตีน (Protein Structure Prediction)

    \item การเข้ากันได้เชิงชีวะ (Biocombatibility) เช่น Cell Wall Penetration หรือ Chemical Processes

    \item การจำลองการจับกันของโมเลกุล (Molecular Docking)
\end{itemize}

\noindent \textbf{การแพทย์ (Medicine)}

\begin{itemize}[topsep=0pt,noitemsep]
    \setlength\itemsep{0.5em}
    \item การออกแบบโมเลกุลยา (Drug Design)

    \item การค้นหาโมเลกุลยา (Drug Discovery)
\end{itemize}

%----------------------------------------
\section{ประวัติศาสตร์ของ Molecular Dynamics}
%----------------------------------------

\begin{itemize}[noitemsep]
    \setlength\itemsep{1em}
    \item 1953: Nicholas Metropolis และคณะได้ตีพิมพ์บทความวิจัยเรื่อง \enquote{Equation of State Calculations by Fast
              Computing Machines}\autocite{metropolis1953} โดยบทความนี้เป็นเสมือนจุดเริ่มต้นของไอเดีย MD เลยก็ว่าได้ โดยเป็นครั้งแรกที่%
          ได้มีการประยุกต์ใช้เทคนิค Monte Carlo เพื่อแก้สมการที่อธิบายคุณสมบัติเชิงกายภาพของระบบที่ประกอบไปด้วยโมเลกุลที่มีอันตรกิริยาต่อกัน
          โดยขั้นแรกคือสร้างเซตของตัวเลขสุ่ม (Random Number) เพื่อใช้เป็นตัวแทนของ Conformational Space แล้วก็ใช้ค่าของพลังงานเป็นตัว%
          ระบุความน่าจะเป็นของสถานะของระบบที่ศึกษา

    \item 1956: Berni J. Alder และ Thomas E. Wainwright ได้ตีพิมพ์บทความเรื่อง \enquote{Phase Transition for a Hard
              Sphere System}\autocite{alder1957} ซึ่งถือได้ว่าเป็นงานวิจัยที่เป็นจุดเริ่มต้นของ MD เลยก็ว่าได้

    \item 1958: เป็นครั้งแรกที่นักวิทยาศาสตร์ค้นพบโครงสร้างสามมิติของโปรตีนได้โดยใช้เทคนิค X-ray โดยเผยแพร่ในบทความ
          \enquote{A Three-Dimensional Model of the Myoglobin Molecule Obtained by X-Ray Analysis}\autocite{kendrew1958}

    \item 1964: บทความวิจัยเรื่อง \enquote{Correlations in the Motion of Atoms in Liquid Argon}\autocite{rahman1964}
          โดย Aneesur Rahman ซึ่งเป็นผู้ที่ใช้ MD ในการคำนวณระบบของ Liquid Argon ซึ่งระบบที่ศึกษาตอนนั้นมี Argon ทั้งหมด 864 อะตอม
          โดยคำนวณด้วยซุปเปอร์คอมพิวเตอร์  CDC 3600 โดยใช้ Lennard-Jones Potential นอกจากนี้ Aneesur Rahman ได้รับการยอบรับว่า%
          เป็นบิดาแห่งพลวัตเชิงโมเลกุลอีกด้วย (The Father of Molecular Dynamics)

    \item 1971: Aneesur Rahman และ Frank H. Stillinger ได้ตีพิมพ์บทความเรื่อง \enquote{Molecular Dynamics Study of
              Liquid Water}\autocite{rahman1971} ซึ่งเป็นใช้ MD ในการจำลองระบบโมเลกุลน้ำที่มีจำนวนโมเลกุลคือ 216 โมเลกุล

    \item 1975: Michael Levitt และ Arich Warshel ได้เผยแพร่บทความวิจัยเรื่อง \enquote{Computer Simulation of Protein
              Folding}\autocite{levitt1975} ซึ่งเป็นครั้งแรกที่มีการนำเทคนิค MD มาใช้ในการจำลองการพับของโปรตีนโดยเป็นการศึกษาการพับของ
          Bovine Pancreatic Trypsin Inhibitor (BPTI) จากโครงสร้างที่เป็นแบบสายเปิด

    \item 1979: David A. Case และ Martin Karplus ได้จำลองโปรตีนที่มีลิแกนด์เป็นโมเลกุลที่เข้าไปจับกับโปรแกรมด้วยเป็นครั้งแรก
          โดยได้ตีพิมพ์งานวิจัยเรื่อง \enquote{Dynamics of ligand binding to heme protein}\autocite{case1979}

    \item 1980s: ในช่วงต้น ๆ ทศวรรษ 1980 นั้นเป็นช่วงที่มีการศึกษาชีวโมเลกุลด้วยการจำลอง MD เป็นจำนวนมาก รวมไปถึงมีการคำนวณ
          Free Energy ด้วย

    \item 1985: Roberto Car Michele Parrinello ได้พัฒนาเทคนิค Car-Parrinello Molecular Dynamics (CPMD) ซึ่งเสนอใน%
          บทความเรื่อง \enquote{Unified Approach for Molecular Dynamics and Density-Functional Theory}\autocite{car1985}
          โดยเป็นการนำเทคนิค Density Functional Theory มารวมกับ Born-Oppenheimer Molecular Dynamics

    \item 1988: Michael Levitt และ Ruth Sharon ได้คำนวณระบบของโปรตีนที่มีโมเลกุลน้ำเป็นตัวทำละลายและนำเสนอในบทความเรื่อง
          \enquote{Accurate Simulation of Protein Dynamics in Solution}\autocite{levitt1988}

    \item 1990s: ในช่วงต้น ๆ ทศวรรษ 1990 นั้นก็ได้มีการพัฒนาศักย์ (Potential) ที่ใช้ในวิธี MD รวมถึงเทคนิคการเพิ่มประสิทธิภาพใน%
          การสุ่ม (Enhanced Sampling) อย่างต่อเนื่อง
\end{itemize}

%----------------------------------------
\section{ทำความรู้จักกับสนามแรง}
%----------------------------------------

สนามแรง (Force Field) เป็นสมการคณิตศาสตร์ที่อธิบายพื้นผิวพลังงานศักย์ของโมเลกุลได้โดย Force Field นั้นจะเป็นผลรวมของเทอมพลังงานต่าง ๆ
ที่อ้างอิงอยู่กับอะตอมและมีพารามิเตอร์ที่สอดคล้องกับโครงสร้างเชิงอิเล็กทรอนิกส์ของโมเลกุลซึ่งเทอมพลังงานแต่ละเทอมนั้นจะมีการตีความทางกายภาพ
(Physical Interpretation) แตกต่างกันไป ไอเดียเริ่มต้นของ Force Field นั้นก็คือในการจำลอง MD นั้นเราจำเป็นจะต้องคำนวณแรง (Force)
ระหว่างอะตอมแต่ละคู่ของทุกอะตอมในระบบของเราซึ่งอาจจะมีมากถึงหลักพันหรือหลักหมื่นอะตอมเลยทีเดียว โดยทั่วไปแล้ว Time-step ที่เรามักจะใช้ใน%
การจำลอง MD นั้นคือ 1 fs ถ้าหากเราต้องการรัน MD เป็นระยะเวลา 100 ns เราจำเป็นจะต้องคำนวณแรงระหว่างอะตอมทั้งหมดประมาณ
\num{e7}-\num{e8} ครั้งเลยทีเดียว สำหรับ Force Field มาตรฐานของพื้นผิวพลังงานศักย์ของโมเลกุล (Potential Energy Surface หรือ
PES) นั้นมีหน้าตาประมาณนี้

\begin{equation}
    \begin{aligned}
        V\left(R^{3N}\right)
        = & \underbrace{
            \sum^{\text{bonds}}_{i} \frac{k_{i}}{2} (l_{i} - l_{i,0})^{2}
        }_
        {
            \text{Bond Stretches}
        }
        +
        \underbrace{
            \sum^{\text{angles}}_{i} \frac{k_{i}}{2} (\theta_{i} - \theta_{i,0})^{2}
        }_
        {
            \text{Angle Bends}
        }
        +
        \underbrace{
            \sum^{\text{torsions}}_{i} \frac{V_{i}}{2} (1 + \cos(n_{i} \omega_{i} - \gamma_{i}))
        }_
        {
            \text{Torsional Motion}
        }                  \\
          & + \underbrace{
            \sum_{i,j > i}
            \underbrace{
                4 \epsilon_{i,j}
                \left(
                \left( \frac{\sigma_{ij}}{R_{ij}} \right)^{12}
                - \left( \frac{\sigma_{ij}}{R_{ij}} \right)^{6}
                \right)
            }_
            {
                \text{Lennard-Jones Term}
            }
            +
            \underbrace
            {
                \frac{q_{i} q_{j}}{4 \pi \epsilon_{0} R_{ij}}
            }_
            {
                \text{Coulomb Term}
            }
        }_
        {
            \text{Intermolecular Interactions}
        }
    \end{aligned}
\end{equation}

โดยที่พลังงานแต่ละเทอมนั้นมีตัวแปรที่เกี่ยวข้องกับลักษณะเชิงเรขาคณิตของโมเลกุล เช่น ความยาวพันธะ $l_{i}$, มุมพันธะ $\theta_{i}$,
มุมบิด $\omega_{i}$, และระยะห่างระหว่างอะตอม $R_{ij}$ นอกจากนี้พลังงานแต่ละเทอมนั้นยังมีพารามิเตอร์ที่ขึ้นอยู่กับชนิดของอะตอมด้วย
เช่น เทอมที่เป็นพลังงานสำหรับการยืดหดของพันธะ (Bond Strething) นั้นจะมีค่าคงที่แรง (Force Constant) $k_{i}$ และค่าความพันธะ%
ที่สภาวะสมดุล (Equilibrium Bond Length) $l_{i,0}$ ตอนนี้ผู้อ่านคงกำลังคิดว่าพารามิเตอร์ชุดนี้นั้นจะต้องมีค่าเพียงแค่ค่าเดียวสำหรับอะตอม%
ที่เป็นธาตุเดียวกันแต่ว่าในความเป็นจริงกลับไม่ใช่อย่างนั้นเพราะว่าถึงแม้ว่าจะเป็นธาตุชนิดเดียวกันแต่ก็ขึ้นอยู่กับสภาพวาดล้อม (Environment) รอบ ๆ 
ธาตุนั้นด้วย ยกตัวอย่างให้เข้าใจง่ายคือสมมติว่าเรามีอะตอมคาร์บอนในหมู่เมทิล (Mehtyl Group, \ce{-CH3}) และกับอะตอมคาร์บอนในหมู่คาร์บอนิล 
(Carbonyl Group, \ce{C=O}) นั้นจะมี Characteristics ต่างกันดังนั้นจึงทำให้มีคุณสมบัติที่แตกต่างกัน เช่น ประจุของอะตอม ดังนั้นสำหรับ 
Force Field ที่ดีนั้นควรจะต้องมีชุดเซตของพารามิเตอร์ที่แตกต่างกันไปสำหรับโมเลกุลหรือระบบที่ต้องการศึกษา เช่น โปรตีน, โลหะออกไซด์, 
พอลิเมอร์, หรือคริสตัลไอออนิค

ในการสร้าง Force Field สักอันหนึ่งขึ้นมานั้นเราจะต้องมีการกำหนดว่าเราจะมีการใช้เทอมพลังงานอะไรบ้างสำหรับ Force Field ของเราแล้วก็รวม%
ถึงว่าเราจะมีวิธีการในการคำนวณหาค่าของพารามิเตอร์ใน Force Field สำหรับธาตุแต่ละธาตุอย่างไร ในยุคแรก ๆ ของเคมีเชิงคำนวณนั้นนักวิจัย%
มักจะใช้ผลการทดลองนั้นนำมาเทียบหาค่าของพารามิเตอร์ของ Force Field (Parameter Fitting) ซึ่ง Force Field ประเภทนี้จะมีชื่อเรียกว่า 
\textit{Empirical} Force Field 

%----------------------------------------
\section{สนามแรงสำหรับพันธะโควาเลนท์}
%----------------------------------------

%----------------------------------------
\subsection{Bond Stretching}
%----------------------------------------

%----------------------------------------
\subsection{Angle Bending}
%----------------------------------------

%----------------------------------------
\subsection{Dihedral Terms}
%----------------------------------------

%----------------------------------------
\subsection{Cross Terms}
%----------------------------------------

%----------------------------------------
\subsection{สรุป Bonding}
%----------------------------------------

%----------------------------------------
\section{อันตรกิริยาระหว่างโมเลกุล}
%----------------------------------------

%----------------------------------------
\subsection{Electrostatic Interactions}
%----------------------------------------

%----------------------------------------
\subsection{Electronic Polarization}
%----------------------------------------

%----------------------------------------
\subsection{Dispersion และ Short-range Repulsion}
%----------------------------------------

%----------------------------------------
\subsection{Effective Force Field}
%----------------------------------------

%----------------------------------------
\subsection{สรุป Non-bonding}
%----------------------------------------

%----------------------------------------
\section{แรงระหว่างโมเลกุลจากกลศาสตร์ควอนตัม}
%----------------------------------------

%----------------------------------------
\section{อันตรกิริยาระหว่างโมเลกุล}
\idxboth{อันตรกิริยาระหว่างโมเลกุล}{Molecular Interactions}
%----------------------------------------

หัวใจสำคัญของ MD Simulations นั้นก็คืออันตรกิริยาระหว่างโมเลกุลนั่นก็คือ \enquote{แรง (Force)} โดยมีสมการสำคัญ 2 สมการที่ถือได้ว่า%
เป็นสมการหลักของ MD เลยก็ว่าได้ ดังนี้

\begin{equation}
    \label{eq:force_newton}
    m_{i}\bm{\ddot{r}}_{i} = \bm{f}_{i}
\end{equation}

\begin{equation}
    \label{eq:force_der_ener}
    \bm{f}_{i} = -\nabla_{i}V(\bm{r})
\end{equation}

\noindent สมการด้านบนนี้คือสมการการเคลื่อนที่ของนิวตัน (Newtonian Equation of Motion) สำหรับอะตอม $i$ โดยที่เป้าหมายของเรา%
นั้นก็คือการคำนวณแรง $\bm{f}$ ที่กระทำต่ออะตอมซึ่งสามารถคำนวณได้จากพลังงานศักย์ $V(\bm{r})$ นั่นเอง ส่วนเวกเตอร์ $\bm{r}$ นั้น%
ก็คือพิกัดคาร์ทีเซียนของตำแหน่งของอะตอม (นิวเคลียส) ทั้งหมดทุกอะตอมในโมเลกุลซึ่งเป็นพิกัดแบบ 3 มิติ

\begin{equation}
    \bm{r} = (\underbrace{r_{1,x}, r_{1,y}, r_{1,z}}_{\text{อะตอมตัวที่ 1}}, \dots,
    \underbrace{r_{N,x}, r_{N,y}, r_{N,z}}_{\text{อะตอมตัวที่ $N$}})
\end{equation}

โดยในการจำลอง MD นั้นจะเป็นการแก้สมการที่ \ref{eq:force_newton} และ \ref{eq:force_der_ener} พร้อม ๆ กันไปเป็นสเต็ป ๆ
ตลอดช่วงระยะเวลาที่ทำการจำลอง โดยระยะห่างระหว่างสเต็ปนั้นเรียกว่า Time Step ($\Delta t$)

%----------------------------------------
\section{ข้อจำกัดของ MD}
%----------------------------------------

วิธี MD นั้นก็เหมือนกับวิธีการจำลองทางคอมพิวเตอร์อื่น ๆ ที่มีข้อจำกัดทั้งในเชิงตัวโมเดลของวิธีเองกับในเชิงทรัพยากรที่ใช้ในการคำนวณ โดยข้อจำกัด%
ของ MD สามารถแบ่งออกได้เป็น 4 ข้อหลัก ๆ ดังนี้

\paragraph{1. Time Scale} สเกลเวลาหรือ Time Scale คือสเกลที่บอกถึงระดับของช่วงเวลาที่ใช้ในการอธิบายปรากฎการณ์หรือพฤติกรรมของโมเลกุล%
หรือระบบที่เราต้องการศึกษา เช่น การสั่นของพันธะโมเลกุลนั้นจะมี Time Scale ในระดับ Femtosecond ดังนั้น Time Scale ที่เหมาะสมสำหรับ%
การกำหนด Time Step นั่นจึงอยู่ที่ประมาณ 1 fs เพราะว่าถ้าหากเรากำหนด Time Step ที่กว้างหรือช้ากว่านี้เช่น 10 fs เราก็จะไม่สามารถติดตาม%
การสั่นของโมเลกุลได้เพราะว่าช่วงระยะเวลาที่ใช้ในการขยับหรือเปลี่ยนตำแหน่งของโครสร้างของโมเลกุลนั้นมากกว่าการสั่นของโมเลกุลหลายเท่า

สำหรับการจำลองเหตุการณ์หรือ Event ในการจำลอง MD นั้นเราควรจะต้องทราบถึงระยะเวลาที่เร็วที่สุดที่เหตุการณ์นั้นสามารถเกิดขึ้นได้ก่อน เช่น
การพับของโปรตีน (Protein Folding) นั้นจะใช้เวลาประมาณ 1 วินาที ดังนั้นถ้าหากเรากำนดให้ Time Step = 1 fs เราจะต้องทำการจำลอง
MD ประมาณ $10^{15}$ สเต็ปถึงจะสามารถจำลองการพับของโปรตีนได้ อย่างไรก็ตามในความเป็นจริงนั้นปรากฎการณ์ต่าง ๆ ของโมเลกุลที่เกิดขึ้นนั้น%
มักจะเกิดขึ้นในช่วงเวลาระดับ Microsecond ($\mu s$)

\paragraph{2. Length Scale} สเกลขนาดหรือ Length Scale คือสเกลที่บ่งบอกถึงขนาดของระบบที่ถูกจำลองซึ่ง Length Scale นี้จะแบ่ง%
ตามขนาดของระบบที่ใช้ในการศึกษา ถ้าหากเราต้องการที่จะศึกษาคุณสมบัติของระบบที่มีขนาดใหญ่ Length Scale ก็จะต้องสอดคล้องกับระบบด้วย
เช่น การจำลองโครงข่ายพอลิเมอร์ (Polymer) เพื่อให้มีความเหมาะสมและมีขนาดใหญ่ของระบบที่ใหญ่มากพอที่จะเป็นตัวแทนของระบบพอลิเมอร์%
ในธรรมชาติจริง ๆ

\paragraph{3. ความแม่นยำของแรงที่คำนวณได้} หัวใจสำคัญของ MD นั้นก็คือการคำนวณแรงที่เป็นอันตรกิริยาระหว่างอะตอมในโมเลกุล ถ้าหาก%
เราใช้วิธีการคำนวณแรงที่มีความแม่นยำสูงก็จะทำให้เราได้แรงที่มีความถูกต้องมาก แต่วิธีการที่มีความแม่นยำสูงนั้นมักจะต้องแลกมาด้วยการคำนวณที่%
สิ้นเปลือง ดังนั้นเรามักจะทำการ Trade-off หรือชั่งน้ำหนักระหว่างการเลือกวิธีในการคำนวณแรงและความสิ้นเปลืองของวิธีนั้น ๆ เพราะอย่าลืมว่า%
เราต้องคำนวณแรงทุก ๆ สเต็ปของการจำลอง MD

%----------------------------------------
\section{ขั้นตอนการจำลอง MD}
%----------------------------------------

การจำลอง MD นั้นโดยปกติแล้วประกอบไปด้วยขั้นตอนดังต่อไปนี้

\paragraph{1. เลือกโมเดล}

\paragraph{2. เตรียมโครงสร้างเริ่มต้น}

\paragraph{3. รันการจำลอง MD}

\paragraph{4. วิเคราะห์ผลการจำลอง MD}

\paragraph{5. คำนวณคุณสมบัติอื่น ๆ เพิ่มเติม}

%----------------------------------------
\section{แบบฝึกหัด}
%----------------------------------------
