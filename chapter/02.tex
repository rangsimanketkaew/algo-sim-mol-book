% LaTeX source for ``Algorithms for Computer Simulation of Molecular Systems''
% Copyright (c) 2023 รังสิมันต์ เกษแก้ว (Rangsiman Ketkaew).

% License: Creative Commons Attribution-NonCommercial-NoDerivatives 4.0 International (CC BY-NC-ND 4.0)
% https://creativecommons.org/licenses/by-nc-nd/4.0/

\chapter{พลวัตเชิงโมเลกุลแบบดั้งเดิม}
\label{ch:md}

%----------------------------------------
\section{การประยุกต์ใช้ Molecular Dynamics}
%----------------------------------------

ก่อนที่เราจะไปศึกษาวิธีการจำลองระบบโมเลกุลที่มีความซับซ้อนนั้นเราควรเริ่มต้นด้วยการศึกษาวิธีอย่างง่ายก่อนนั่นก็คือ Molecular Dynamics (MD)
(จริง ๆ จะว่าไปแล้ววิธีนี้ก็ไม่ได้ง่ายนะครับ ถ้าลงรายละเอียดลึก ๆ แล้วก็มีความซับซ้อนมากพอสมควร ซึ่งในปัจจุบันนั้นก็มีการทำวิจัยที่เกี่ยวข้องกับ%
การพัฒนาเทคนิคของวิธี MD อย่างต่อเนื่อง) เราใช้เทคนิคการจำลองทางคอมพิวเตอร์ในการทำความเข้าใจคุณสมบัติของระบบที่ประกอบไปด้วยโมเลกุล
ๆ โมเลกุลในเชิงโครงสร้างและอันตรกิริยาในระดับจุลภาคหรือหน่วยย่อย ซึ่งหน่วยย่อยในที่นี้ก็คือโมเลกุลนั่นเอง โดยเราสามารถแบ่งวิธีการจำลองออก%
ได้เป็น 2 วิธีคือ Molecular Dynamics (MD) กับ Monte Carlo (MC) และหนังสือเล่มนี้จะเน้นไปที่ MD ซึ่งเป็นวิธีที่สามารถนำมาใช้ในการศึกษา%
คุณสมบัติเชิงพลวัตได้ เช่น สัมประสิทธิ์การเคลื่อนที่ (Transport Coefficients), การตอบสนองต่อการรบกวนแบบที่ขึ้นกับเวลา (Time-dependt
Response to Perturbation), และสเปกตรัม

ตัวอย่างของคุณสมบัติและปรากฏการณ์ของโมเลกุลหรือสสารที่เราสามารถใช้การจำลอง MD เพื่อศึกษาได้นั้นมีดังต่อไปนี้

\noindent \textbf{เคมี (Chemistry)}

\begin{itemize}[topsep=0pt,noitemsep]
    \setlength\itemsep{0.5em}
    \item อันตรกิริยาภายในและภายนอกโมเลกุล (Intra- and Intermolecular Interactions)

    \item ปฏิกิริยาเคมี (Chemical Reactions)

    \item การเปลี่ยนเฟส (Phase Transitions)

    \item การคำนวณพลังงานอิสระ (Free Energy Calculations)
\end{itemize}

\noindent \textbf{วัสดุศาสตร์ (Materials Science)}

\begin{itemize}[topsep=0pt,noitemsep]
    \setlength\itemsep{0.5em}
    \item เทอร์โมไดนามิกส์ที่สภาวะสมดุล (Equilibrium Thermodynamics)

    \item การเปลี่ยนเฟส (Phase Transitions)

    \item Properties of Lattice Defects

    \item Nucleation and Surface Growth

    \item กระบวนการความร้อนและความ (Heat/Pressure Processing)

    \item Ion Implantation

    \item Properties of Nanostructures
\end{itemize}

\noindent \textbf{ชีวเชิงฟิสิกส์และชีวเคมี (Biophysics and Biochemistry)}

\begin{itemize}[topsep=0pt,noitemsep]
    \setlength\itemsep{0.5em}
    \item การพับของโปรตีน (Protein Folding)

    \item การทำนายโครงสร้างของโปรตีน (Protein Structure Prediction)

    \item การเข้ากันได้เชิงชีวะ (Biocombatibility) เช่น Cell Wall Penetration หรือ Chemical Processes

    \item การจำลองการจับกันของโมเลกุล (Molecular Docking)
\end{itemize}

\noindent \textbf{การแพทย์ (Medicine)}

\begin{itemize}[topsep=0pt,noitemsep]
    \setlength\itemsep{0.5em}
    \item การออกแบบโมเลกุลยา (Drug Design)

    \item การค้นหาโมเลกุลยา (Drug Discovery)
\end{itemize}

%----------------------------------------
\section{ประวัติศาสตร์ของ Molecular Dynamics}
%----------------------------------------

Molecular Dynamics หรือ MD นั้นเป็นสาขาหนึ่งของเคมีทฤษฎีที่มีการค้นคว้าและวิจัยมาอย่างยาวนาน ผมสรุปไทม์ไลน์เรียงตามเหตุการณ์%
ที่เกิดขึ้นในวงวิชาการตั้งแต่อดีตในช่วงยุคแรก ๆ ของการพัฒนาวิธี MD จนถึงปัจจุบันตามนี้ครับ

\begin{itemize}[topsep=0pt,noitemsep]
    \setlength\itemsep{0.5em}
    \item 1953: Nicholas Metropolis และคณะได้ตีพิมพ์บทความวิจัยเรื่อง \enquote{Equation of State Calculations by Fast
              Computing Machines}\autocite{metropolis1953} โดยบทความนี้เป็นเสมือนจุดเริ่มต้นของไอเดีย MD เลยก็ว่าได้ โดยเป็นครั้งแรกที่%
          ได้มีการประยุกต์ใช้เทคนิค Monte Carlo เพื่อแก้สมการที่อธิบายคุณสมบัติเชิงกายภาพของระบบที่ประกอบไปด้วยโมเลกุลที่มีอันตรกิริยาต่อกัน
          โดยขั้นแรกคือสร้างเซตของตัวเลขสุ่ม (Random Number) เพื่อใช้เป็นตัวแทนของ Conformational Space แล้วก็ใช้ค่าของพลังงานเป็นตัว%
          ระบุความน่าจะเป็นของสถานะของระบบที่ศึกษา

    \item 1956: Berni J. Alder และ Thomas E. Wainwright ได้ตีพิมพ์บทความเรื่อง \enquote{Phase Transition for a Hard
              Sphere System}\autocite{alder1957} ซึ่งถือได้ว่าเป็นงานวิจัยที่เป็นจุดเริ่มต้นของ MD เลยก็ว่าได้

    \item 1958: เป็นครั้งแรกที่นักวิทยาศาสตร์ค้นพบโครงสร้างสามมิติของโปรตีนได้โดยใช้เทคนิค X-ray โดยเผยแพร่ในบทความ
          \enquote{A Three-Dimensional Model of the Myoglobin Molecule Obtained by X-Ray Analysis}\autocite{kendrew1958}

    \item 1964: บทความวิจัยเรื่อง \enquote{Correlations in the Motion of Atoms in Liquid Argon}\autocite{rahman1964}
          โดย Aneesur Rahman ซึ่งเป็นผู้ที่ใช้ MD ในการคำนวณระบบของ Liquid Argon ซึ่งระบบที่ศึกษาตอนนั้นมี Argon ทั้งหมด 864 อะตอม
          โดยคำนวณด้วยซุปเปอร์คอมพิวเตอร์  CDC 3600 โดยใช้ Lennard-Jones Potential นอกจากนี้ Aneesur Rahman ได้รับการยอบรับว่า%
          เป็นบิดาแห่งพลวัตเชิงโมเลกุลอีกด้วย (The Father of Molecular Dynamics)

    \item 1971: Aneesur Rahman และ Frank H. Stillinger ได้ตีพิมพ์บทความเรื่อง \enquote{Molecular Dynamics Study of
              Liquid Water}\autocite{rahman1971} ซึ่งเป็นใช้ MD ในการจำลองระบบโมเลกุลน้ำที่มีจำนวนโมเลกุลคือ 216 โมเลกุล

    \item 1975: Michael Levitt และ Arich Warshel ได้เผยแพร่บทความวิจัยเรื่อง \enquote{Computer Simulation of Protein
              Folding}\autocite{levitt1975} ซึ่งเป็นครั้งแรกที่มีการนำเทคนิค MD มาใช้ในการจำลองการพับของโปรตีนโดยเป็นการศึกษาการพับของ
          Bovine Pancreatic Trypsin Inhibitor (BPTI) จากโครงสร้างที่เป็นแบบสายเปิด

    \item 1979: David A. Case และ Martin Karplus ได้จำลองโปรตีนที่มีลิแกนด์เป็นโมเลกุลที่เข้าไปจับกับโปรแกรมด้วยเป็นครั้งแรก
          โดยได้ตีพิมพ์งานวิจัยเรื่อง \enquote{Dynamics of ligand binding to heme protein}\autocite{case1979}

    \item 1980s: ในช่วงต้น ๆ ทศวรรษ 1980 นั้นเป็นช่วงที่มีการศึกษาชีวโมเลกุลด้วยการจำลอง MD เป็นจำนวนมาก รวมไปถึงมีการคำนวณ
          Free Energy ด้วย

    \item 1985: Roberto Car Michele Parrinello ได้พัฒนาเทคนิค Car-Parrinello Molecular Dynamics (CPMD) ซึ่งเสนอใน%
          บทความเรื่อง \enquote{Unified Approach for Molecular Dynamics and Density-Functional Theory}\autocite{car1985}
          โดยเป็นการนำเทคนิค Density Functional Theory มารวมกับ Born-Oppenheimer Molecular Dynamics

    \item 1988: Michael Levitt และ Ruth Sharon ได้คำนวณระบบของโปรตีนที่มีโมเลกุลน้ำเป็นตัวทำละลายและนำเสนอในบทความเรื่อง
          \enquote{Accurate Simulation of Protein Dynamics in Solution}\autocite{levitt1988}

    \item 1990s: ในช่วงต้น ๆ ทศวรรษ 1990 นั้นก็ได้มีการพัฒนาศักย์ (Potential) ที่ใช้ในวิธี MD รวมถึงเทคนิคการเพิ่มประสิทธิภาพใน%
          การสุ่ม (Enhanced Sampling) อย่างต่อเนื่อง
\end{itemize}

%----------------------------------------
\section{สนามแรง}
\idxboth{สนามแรง}{Force Fields}
%----------------------------------------

สนามแรง (Force Field) เป็นสมการคณิตศาสตร์ที่อธิบายพื้นผิวพลังงานศักย์ของโมเลกุลได้โดย Force Field นั้นจะเป็นผลรวมของเทอมพลังงานต่าง ๆ
ที่อ้างอิงอยู่กับอะตอมและมีพารามิเตอร์ที่สอดคล้องกับโครงสร้างเชิงอิเล็กทรอนิกส์ของโมเลกุลซึ่งเทอมพลังงานแต่ละเทอมนั้นจะมีการตีความทางกายภาพ
(Physical Interpretation) แตกต่างกันไป ไอเดียเริ่มต้นของ Force Field นั้นก็คือในการจำลอง MD นั้นเราจำเป็นจะต้องคำนวณแรง (Force)
ระหว่างอะตอมแต่ละคู่ของทุกอะตอมในระบบของเราซึ่งอาจจะมีมากถึงหลักพันหรือหลักหมื่นอะตอมเลยทีเดียว โดยทั่วไปแล้ว Time-step ที่เรามักจะใช้ใน%
การจำลอง MD นั้นคือ 1 fs ถ้าหากเราต้องการรัน MD เป็นระยะเวลา 100 ns เราจำเป็นจะต้องคำนวณแรงระหว่างอะตอมทั้งหมดประมาณ
\num{e7}-\num{e8} ครั้งเลยทีเดียว สำหรับ Force Field มาตรฐานของพื้นผิวพลังงานศักย์ของโมเลกุล (Potential Energy Surface หรือ
PES) นั้นมีหน้าตาประมาณนี้

\begin{tcolorbox}[ams equation]
    \begin{aligned}
        V\left(R^{3N}\right)
        = & \underbrace{
            \sum^{\text{bonds}}_{i} \frac{k_{i}}{2} (l_{i} - l_{i,0})^{2}
        }_
        {
            \text{Bond Stretches}
        }
        +
        \underbrace{
            \sum^{\text{angles}}_{i} \frac{k_{i}}{2} (\theta_{i} - \theta_{i,0})^{2}
        }_
        {
            \text{Angle Bends}
        }
        +
        \underbrace{
            \sum^{\text{torsions}}_{i} \frac{V_{i}}{2} (1 + \cos(n_{i} \omega_{i} - \gamma_{i}))
        }_
        {
            \text{Torsional Motion}
        }                  \\
          & + \underbrace{
            \sum_{i,j > i}
            \underbrace{
                4 \epsilon_{i,j}
                \left(
                \left( \frac{\sigma_{ij}}{R_{ij}} \right)^{12}
                - \left( \frac{\sigma_{ij}}{R_{ij}} \right)^{6}
                \right)
            }_
            {
                \text{Lennard-Jones Term}
            }
            +
            \underbrace
            {
                \frac{q_{i} q_{j}}{4 \pi \epsilon_{0} R_{ij}}
            }_
            {
                \text{Coulomb Term}
            }
        }_
        {
            \text{Intermolecular Interactions}
        }
    \end{aligned}
\end{tcolorbox}

โดยที่พลังงานแต่ละเทอมนั้นมีตัวแปรที่เกี่ยวข้องกับลักษณะเชิงเรขาคณิตของโมเลกุล เช่น ความยาวพันธะ $l_{i}$, มุมพันธะ $\theta_{i}$,
มุมบิด $\omega_{i}$, และระยะห่างระหว่างอะตอม $R_{ij}$ นอกจากนี้พลังงานแต่ละเทอมนั้นยังมีพารามิเตอร์ที่ขึ้นอยู่กับชนิดของอะตอมด้วย
เช่น เทอมที่เป็นพลังงานสำหรับการยืดหดของพันธะ (Bond Strething) นั้นจะมีค่าคงที่แรง (Force Constant) $k_{i}$ และค่าความพันธะ%
ที่สภาวะสมดุล (Equilibrium Bond Length) $l_{i,0}$

\begin{enumerate}[topsep=0pt,noitemsep]
    \setlength\itemsep{1em}
    \item 3 เทอมแรกคือ Bonded เป็นพลังงานที่มาจากภายในของโมเลกุลเอง

    \item เทอมที่ 4 คือ Non-bonded ที่อธิบาย Electrostatic Interaction มีชื่อเรียกว่า Coulomb Interaction

    \item เทอมที่ 5 คือ Non-bonded ที่อธิบายพลังงาน Non-electrostatic ที่เกิดจาก Dipole-dipole Interaction
          (เช่น London Force ที่อธิบาย Interaction ระหว่าง Non-polar Molecules เป็นต้น) มีชื่อเรียกว่า Lennard-Jones Potential
          (หรือเรียกสั้น ๆ ว่า LJ Potential หรือ 12-6 Potential)

\end{enumerate}

ตอนนี้ผู้อ่านคงกำลังคิดว่าพารามิเตอร์ชุดนี้นั้นจะต้องมีค่าเพียงแค่ค่าเดียวสำหรับอะตอมที่เป็นธาตุเดียวกันแต่ว่าในความเป็นจริงกลับไม่ใช่อย่างนั้น%
เพราะว่าถึงแม้ว่าจะเป็นธาตุชนิดเดียวกันแต่ก็ขึ้นอยู่กับสภาพวาดล้อม (Environment) รอบ ๆ ธาตุนั้นด้วย ยกตัวอย่างให้เข้าใจง่ายคือสมมติว่า%
เรามีอะตอมคาร์บอนในหมู่เมทิล (Mehtyl Group, \ce{-CH3}) และกับอะตอมคาร์บอนในหมู่คาร์บอนิล (Carbonyl Group, \ce{C=O})
นั้นจะมี Characteristics ต่างกันดังนั้นจึงทำให้มีคุณสมบัติที่แตกต่างกัน เช่น ประจุของอะตอม ดังนั้นสำหรับ Force Field
ที่ดีนั้นควรจะต้องมีชุดเซตของพารามิเตอร์ที่แตกต่างกันไปสำหรับโมเลกุลหรือระบบที่ต้องการศึกษา เช่น โปรตีน, โลหะออกไซด์, พอลิเมอร์, หรือคริสตัลไอออนิค

\paragraph{สำหรับ Quantum calculation}
ในการศึกษาคุณสมบัติของโมเลกุล (Molecular properties) นั้น Properties หลาย ๆ ตัวนั้นเป็นญาติกับพลังงาน ก็คือถ้าเรารู้พลังงานของโมเลกุล เราก็จะสามารถคำนวณ properties อื่น ๆ ตามมาได้ (ในรูปของอนุพันธ์เทียบกับ perturbation อะไรก็ว่าไป)

\paragraph{สำหรับ Molecular dynamics}
สำหรับ Molecular dynamics: เราใช้ Force Field ในการคำนวณหาพลังงานของโมเลกุล เพื่อนำพลังงานมาคำนวณหาแรง (Force) ที่กระทำต่ออะตอมแต่ละตัว

ในการสร้าง Force Field สักอันหนึ่งขึ้นมานั้นเราจะต้องมีการกำหนดว่าเราจะมีการใช้เทอมพลังงานอะไรบ้างสำหรับ Force Field ของเราแล้วก็รวม%
ถึงว่าเราจะมีวิธีการในการคำนวณหาค่าของพารามิเตอร์ใน Force Field สำหรับธาตุแต่ละธาตุอย่างไร ในยุคแรก ๆ ของเคมีเชิงคำนวณนั้นนักวิจัย%
มักจะใช้ผลการทดลองนั้นนำมาเทียบหาค่าของพารามิเตอร์ของ Force Field (Parameter Fitting) ซึ่ง Force Field ประเภทนี้จะมีชื่อเรียกว่า
\textit{Empirical} Force Field

การนำ Force Field ไปใช้ในการจำลอง Molecular Dynamics นั้นมีขั้นตอนคร่าว ๆ ดังนี้

\begin{enumerate}[topsep=0pt,noitemsep]
    \setlength\itemsep{1em}
    \item นำแรงต่ออะตอมมาคำนวณหาความเร่ง (Acceleration) แล้วนำไปเข้าสมการ Equation of Motions
          เพื่อคำนวณหาความเร็ว (Velocity) และการกระจัด (Displacement) ที่เปลี่ยนไป

    \item นำการกระจัดที่เปลี่ยนไปมาทำการอัพเดทตำแหน่งของอะตอม/โมเลกุล เรียกวิธีการนี้ว่า Propagation

    \item เมื่อเราทำแบบนี้ไปเรื่อย ๆ เราจะได้ Dynamic ของระบบที่สามารถที่จะ Represent คุณสมบัติ Microscopic ของระบบจริง ๆ ได้
\end{enumerate}

โดยขนาดของระบบที่เราใช้ในการจำลองจะสอดคล้องกับ Time-scale ของการรัน Dynamic Simulation ในการศึกษา Properties
ที่แตกต่างกันออกไป นอกจากนี้ยังมีวิธีการคำนวณ Force field ที่ซับซ้อนกว่านี้อีกมากมาย ขึ้นอยู่กับวิธีการ accuracy ที่ต้องการ
สรุปคือ Force field นั้นสำคัญมาก ๆ เพราะเป็นจุดเริ่มต้นของการศึกษา Properties อื่น ๆ อีกมากมายของโมเลกุล ดังนั้น
\textit{เลือก Force Field ไม่ดี = ชีวิตพัง}

%----------------------------------------
\section{สนามแรงสำหรับพันธะโควาเลนท์}
\label{sec:md_ff_covalent_bond}
\idxboth{สนามแรง!พันธะโควาเลนท์}{Force Fields!Covalent Bonding}
%----------------------------------------

ในหัวข้อนี้เราจะมาดูรายละเอียดของ Force Field ที่สำคัญมากที่สุดอันหนึ่งนั่นก็คือ Force Field ที่ใช้ในการอธิบายพันธะโควาเลนท์ (Covalent
Bonding) ซึ่งประกอบไปด้วย การยืดหดของพันธะ (Bond Stretching), การงอของพันธะ (Angle Bending) และการเคลื่อนแบบบิด (Torsional
Motion)

%----------------------------------------
\subsection{Bond Stretching}
\idxen{Bond Stretching}
%----------------------------------------

เริ่มต้นด้วยการพิจารณาพลังงานศักย์ $V(R)$ ของโมเลกุลอะตอมคู่ (Diatomic Molecule) ซึ่งเป็นฟังก์ชันของระยะห่างระหว่างอะตอม
(Bond Distance) $R$ โดยเราสามารถแสดง (Represent) $V(R)$ อันนี้ได้ด้วยศักย์ของมอส (Morse Potential) ดังต่อไปนี้

\begin{figure}[htbp]
    \centering
    \includegraphics[width=0.8\linewidth]{fig/morse-potential.png}
    \caption{Morse Potential}
    \label{fig:morse_potential}
\end{figure}

\begin{equation}
    \label{eq:potential_bond_stretch}
    V(R) = D_{e} \left( e^{-2a(R-R_{e})} -2e^{-a(R-R_{e})} \right)
\end{equation}

\noindent โดยที่ $R_{e}$ คือระยะห่างระหว่างอะตอมที่สภาวะสมดุล เช่น ความยาวพันธะ ณ ตำแหน่งที่พลังงานศักย์ของ Morse Potentail
นั้นมีค่าน้อยที่สุด, $D_{e}$ คือความลึก (Depth) ของพื้นผิวศักย์ (Potentail Surface) ซึ่งก็คือพลังงานการแตกออกหรือการแยกตัว (Dissociation
Energy) และ $a$ คือพารามิเตอร์ที่อธิบายความกว้างของบ่อพลังงานศักย์อันนี้ นอกจากนี้ Morse Potential สามารถถูกเขียนได้ด้วยวิธีอื่น ๆ ได้อีกด้วย

หนึ่งในวิธีที่เราจะสามารถใช้ในการ Represent พันธะโควาเลนท์ก็คือการใช้โมเดลการสั่น Harmonic OSciallator แบบคลาสสิค
เช่น ถ้าเรามีอะตอม 2 อะตอมที่ถูกยึดเข้าด้วยกันด้วยสปริงที่มี Force Constant $k$ เราสามารถใช้ Taylor Expansion ในการอธิบาย Potential
Energy $V(R)$ ได้โดยการใช้ Dunham Expansion Parameter $Q = (\frac{R-R_{e}}{R_{e}})$ ดังนี้

\begin{equation}
    V(Q)
    =
    \underbrace{V(0)}_{\text {Zero level }}
    + \underbrace{V^{\prime}(0) Q}_{V^{\prime}(0)=0 \text { in the minimum }}
    + \underbrace{\frac{1}{2} V^{\prime \prime}(0) Q^2}_{\text {Harmonic term }}
    + \underbrace{\frac{1}{6} V^{(3)}(0) Q^3}_{\text {Anharmonicity }}
    + \underbrace{\frac{1}{24} V^{(4)}(0) Q^4 \ldots,}_{\text {Quartic term }}
\end{equation}

\noindent โดยที่เราไม่ต้องพิจารณา Zero Level ($V(0)$) ก็ได้ เพราะว่า Potential Energy Surface นั้นสามารถเปลี่ยนระดับพลังงาน%
ได้ด้วยค่าพลังงานคงที่ สำหรับเทอมที่เป็นเส้นตรง (Linear Term) ใน $Q$ นั้นมีค่าเท่ากับ 0 เนื่องจากว่า Gradient ของเทอมนี้นั้นเท่ากับ 0
ที่ตำแหน่ง Minimum ส่วนเทอมที่เป็น Quadratic Term กับ Cubic Term ใน $Q$ นั้นคือ Harmonic และ Anharmonic ของพื้นผิวศักย์ตามลำดับ
ถ้าหากว่าเราตัดเทอม Anharmonicity แล้วก็เทอมที่มีอันดับสูงกว่านี้ออกไปจาก Tarlor Expansion สิ่งที่เราจะได้นั้นก็คือการสั่นฮาร์โมนิคแบบ%
ดั้งเดิม (Classical Harmonic Oscillator) นั่นเอง

\begin{equation}
    V(Q)
    =
    \frac{k}{2} Q^{2} \quad \texttt{โดยที่} k \equiv V''(0)
\end{equation}

Taylor Expansion ของ Morse Potential นั้นเป็นหนึ่งโมเดลของพื้นผิวศักย์ที่สามารถอธิบายสถานะของระบบรอบ ๆ จุดต่ำสุด Minimum ได้
ซึ่งจะอธิบายได้ดีก็ต่อเมื่อเราทำการตัดหรือไม่พิจารณาเทอมที่มีอันดับสูงกว่า Harmonic Term ออกไป อย่างไรก็ตาม โมเดลนี้ก็มีจุดอ่อน ถ้าหากว่าเรา%
ดูกรณีที่อะตอมทั้งสองอะตอมนั้นมีระยะห่างกันมาก ๆ ซึ่งก็คือห่างกันอนันต์ $Q \rightarrow \infty$ จะทำให้พลังงานศักย์นั้นเข้าใกล้ค่าอนันต์ด้วย
$V(Q) \rightarrow \infty$ ซึ่งเงื่อนไขอันนี้ทำให้โมเดล Morse Potential นั้นอธิบาย Dissociation ได้ไม่ถูกต้อง แต่ถ้าหากว่าโมเดลอันนี้%
ถูกนำมาใช้ในการอธิบายระบบที่มีหลาย ๆ โมเลกุลและแต่ละโมเลกุลนั้นไม่ทำปฏิกิริยาต่อกัน พูดง่าย ๆ ก็คือไม่มีการสร้างพันธะ (Non-reacting)
เราจะพบว่าการที่เราทำการตัดเทอมที่มีอันดับสูงกว่า Harmomic ออกไปนั้นจะทำให้มันสามารถอธิบายพื้นผิวศักย์ได้อย่างสมเหตุสมผล

%----------------------------------------
\subsection{Angle Bending}
\idxen{Angle Bending}
%----------------------------------------

\begin{figure}[htbp]
    \centering
    \includegraphics[width=0.8\linewidth]{fig/water-angle-bending.png}
    \caption{ซ้าย: โมเลกุลน้ำ, ขวา: Double Minimum Potential สำหรับมุมพันธะของโมเลกุลน้ำ}
    \label{fig:water_angle_bending}
\end{figure}

สำหรับการอธิบาย Angle Bending นั้นผมขอยกตัวอย่างโมเลกุลน้ำ \ce{H2O} ซึ่งมีมุมสมดุลคือ $\theta_{e}$ ระหว่างพันธะ \ce{O-H}
ซึ่งมีความยาวพันธะเท่ากับ 104.5 องศา เมื่อความยาวพันธะยืดออกนั้นเราสามารถใช้การประมาณแบบ Harmonic Oscillator เพื่ออธิบาย Angle
Bending ได้ ดังนี้

\begin{equation}
    V(\theta)
    =
    \frac{k_{\theta}}{2}
    (\theta - \theta_{e})^{2}
\end{equation}

\noindent ซึ่ง Approximation ด้านบนนี้สามารถนำมาใช้ได้เมื่อมุม $\theta$ นั้นเข้าใกล้กับมุมสมดุล $\theta_{e}$

ถ้าหากเราลองมาดูกรณีแปลก ๆ เช่น ถ้ามุมพันธะมีค่าเท่ากับ $\pi$ ซึ่งทำให้โมเลกุลน้ำนั้นเป็นเส้นตรง เราจะพบว่าพลังงานศักย์นั้นจะสูงมาก ๆ
และในความเป็นจริงนั้นแทบจะเป็นไปได้ยากมาก ๆ ที่โมเลกุลน้ำนั้นจะเป็นเส้นตรง (แต่ก็สามารถทำได้โดยการเพิ่มอุณหภูมิให้สูงมาก ๆ)

%----------------------------------------
\subsection{Dihedral Terms}
\idxen{Dihedral Terms}
%----------------------------------------

พารามิเตอร์อีกอันหนึ่งที่สำคัญมากในการอธิบายการเปลี่ยนแปลงของมุมพันธะและระนาบในโมเลกุลนั้นก็คือ Dihedral Angle หรือว่ามุมบิด
ซึ่งมุมบิดนี้ก็มี Potential Energy เป็นของตัวเองด้วยโดยมีสมการดังต่อไปนี้

\begin{equation}
    V(\omega)
    =
    \sum_{n} \frac{V_{n}}{2}
    \left(
    1 + \cos(n\omega - \gamma)
    \right)
\end{equation}

\noindent โดยที่ $n$ นั้นคือเลขจำนวนเต็มที่บ่งบอกถึงจำนวนคาบ เช่น $n = 2$ ก็คือมีคาบเท่ากับ 180 องศา หรือ $n = 3$ ก็คือมีคาบเท่ากับ
120 องศานั่นเอง, $V_{n}$ คือพลังงานศักย์การหมุน (Rotational Energy Barrier) และ $\gamma$ นั้นกำหนดว่ามุมที่มุมบิดนั้นมีค่าเท่ากับ
0 องศา

สมการที่ใช้ในการพลังงานศักย์ของ Dihedral Angle นั้นสามารถเขียนได้โดยการใช้ส่วนจริง (Real Part) ของการแปลงฟูเรียร์ ดังนี้

\begin{equation}
    V(\omega)
    =
    \sum_{n} C_{n} \cos(\omega)^{n}
\end{equation}


\begin{figure}[htbp]
    \centering
    \includegraphics[width=0.8\linewidth]{fig/dihedral-angle.png}
    \caption{ซ้าย: โมเลกุลน้ำ, ขวา: Double Minimum Potential สำหรับมุมพันธะของโมเลกุลน้ำ}
    \label{fig:dihedral_angle}
\end{figure}

สำหรับโมเลกุลที่ควรจะต้องแบนราบหรือมีความเป็น Planar อยู่แล้วนั้น บางครั้งเราก็อยากที่จะเพิ่ม Constraint เข้าไปให้กับพื้นผิวศักย์ของโมเลกุล%
เพื่อทำให้โมเลกุลนั้นมีความเป็น Planar ซึ่งวิธีการเพิ่ม Constraint นั้นทำได้หลายวิธี หนึ่งในนั้นก็คือการใส่ Lagrangian Multipliers
ในขณะที่เราทำการปรับค่าพลังงานของโมเลกุลให้มีค่าต่ำที่สุดหรือที่เราเรียกว่า Constrained Energy Minimization ซึ่งจะทำให้เราได้โมเลกุลที่%
มีพลังงานต่ำที่สุดที่สภาวะที่ถูก Constraint อยู่ด้วย จึงทำให้โมเลกุลนั้นถูกบังคับให้แบนราบหรือมีความเป็น Planar ตลอดเวลา นอกจากนี้ยังมีอีกวิธี%
ก็คือการเติมเทอมพลังงานพิเศษเข้าไปทื่อ ๆ เลยอีกเทอมนึง ซึ่งเทอมพลังงานพิเศษอันนี้ที่เราเติมเข้าไปนั้นมีชื่อเรียกว่า Energy Restraint หรือ
Improper Torsion Term นั่นเอง ซึ่งมีหน้าตาดังนี้

\begin{equation}
    V(\omega)
    =
    k_{\omega}
    (1 - \cos 2\omega)
\end{equation}

\noindent โดยที่โมเลกุลนั้นจะถูกบังคับหรือถูกตรึงให้มีลักษณะที่ \textit{เกือบ} จะเป็นแบนราบอยู่ตลอดเวลาถ้าหากว่าค่า $k_{\omega}$
นั้นมีค่ามากพอ

%----------------------------------------
\subsection{Cross Terms}
\idxen{Cross Terms}
%----------------------------------------

เทอมสุดท้ายที่เรามักจะไม่ค่อยคุนชินกันเท่าไหร่เพราะว่ามักจะไม่ค่อยมีใครพูดถึงแม้แต่ในตำราต่างประเทศหลาย ๆ เล่มนั่นก็คือเทอมที่เรียกว่า Cross Term
ซึ่งเทอมนี้จะถูกนำใส่เข้าไปใน Potential Energy Expression เพื่อใช้อธิบายการคู่ควบกัน (Coupling) ระหว่าง Two-bond Stretch
หรือ Bond Stretch กับ Angle Bending Term โดยผมขอยกตัวอย่างด้วยโมเลกุลน้ำเหมือนเดิมครับ ถ้าหากเราพิจารณา Intermolecular Motion
นั้นเราจะสามารถอธิบาย Motion อันนี้ได้โดยการใช้ Taylor Expansion รอบ ๆ ความยาวพันธะทั้งสองอัน ($R_{1}$ กับ $R_{2}$) และมุมพันธะ
($\theta$) ดังนี้

\begin{align}
    V\left(R_1, R_2, \theta\right)
     & =
    V\left(R_{1,0}, R_{2,0}, \theta_0\right)
    + \left.\left(R_1-R_{1,0}\right) \frac{\partial V}{\partial R_1}\right|_{R_{1,0}, R_{2,0}, \theta_0} \nonumber    \\
     & + \left.\left(R_2-R_{2,0}\right) \frac{\partial V}{\partial R_2}\right|_{R_{1,0}, R_{2,0}, \theta_0} \nonumber \\
     & + \left.\left(\theta-\theta_0\right) \frac{\partial V}{\partial \theta}\right|_{R_{1,0}, R_{2,0}, \theta_0}
    + \left.\frac{1}{2}\left(R_1-R_{1,0}\right)^2
    \frac{\partial^2 V}{\partial R_1^2}\right|_{R_{1,0}, R_{2,0}, \theta_0} \nonumber                                 \\
     & + \left.\frac{1}{2}\left(R_2-R_{2,0}\right)^2
    \frac{\partial^2 V}{\partial R_2^2}\right|_{R_{1,0}, R_{2,0}, \theta_0}
    + \left.\frac{1}{2}\left(\theta-\theta_0\right)^2
    \frac{\partial^2 V}{\partial \theta^2}\right|_{R_{1,0}, R_{2,0}, \theta_0} \nonumber                              \\
     & + \left.\left(R_1-R_{1,0}\right)\left(R_2-R_{2,0}\right)
    \frac{\partial^2 V}{\partial R_1 \partial R_2}\right|_{R_{1,0}, R_{2,0}, \theta_0} \nonumber                      \\
     & + \left.\left(R_1-R_{1,0}\right)\left(\theta-\theta_0\right)
    \frac{\partial^2 V}{\partial R_1 \partial \theta}\right|_{R_{1,0}, R_{2,0}, \theta_0} \nonumber                   \\
     & + \left.\left(R_2-R_{2,0}\right)\left(\theta-\theta_0\right)
    \frac{\partial^2 V}{\partial R_2 \partial \theta}\right|_{R_{1,0}, R_{2,0}, \theta_0} + \ldots
\end{align}

\noindent โดยที่ 3 เทอมสุดท้ายนั้นคือ Coupling Terms อย่างไรก็ตาม โดยปกติแล้วเราไม่จำเป็นต้องใส่เทอม Coupling ทั้งหมดที่เรามี
ซึ่งเทอม Coupling บางเทอมก็สำคัญ บางเทอมก็ไม่สำคัญซึ่งขึ้นอยู่กับโมเลกุล สำหรับโมเลกุลน้ำนั้นเทอม Coupling ที่สำคัญและควรจะต้องมีก็คือ
Coupling Term ที่มาจาก Stretch Bond เพื่อที่ว่าจะสามารถอธิบาย Symmetric Stretch Coordinate ได้ ซึ่งมีสมการดังนี้

\begin{equation}
    Q_1-Q_{1,0}
    =
    \left(R_1-R_{1,0}\right)+\left(R_2-R_{2,0}\right)
\end{equation}

\noindent และสามารถ Antisymmetric Stretch Coordinate ได้เช่นกัน ซึ่งมีสมการดังนี้

\begin{equation}
    Q_2-Q_{2,0}
    =
    \left(R_1-R_{1,0}\right)-\left(R_2-R_{2,0}\right)
\end{equation}

แล้วก็ในการรวม Crossing Term เข้าไปในสมการพลังงานศักย์เพื่อใช้อธิบายพื้นผิวศักย์ของโมเลกุลน้ำนั้น เราจะพบว่าพารามิเตอร์ที่ขึ้นอยู่กับชนิดของ%
อะตอมนั้น (เช่น $R_{1,0}$, $R_{2,0}$ และ $\theta_{0}$) นั้นไม่ได้เกี่ยวข้องหรือสอดคล้องกับ Equilibrium Geometry เลย

%----------------------------------------
\subsection{สรุป Bonding}
%----------------------------------------

โดยสรุปแล้วเราเพิ่งได้ศึกษาเทอมของพลังงานต่าง ๆ ที่เรานำมาใช้ในการสร้าง Force Field เพื่อใช้ในการอธิบาย Covalent Bond ซึ่งประกอบไปด้วย%
เทอมดังต่อไปนี้ Bond Stretching, Angle Bending, Torsional Motion แล้วก็มีการรวมเทอมพิเศษเข้าไปด้วยซึ่งก็คือ Coupling Term
และนอกจากนี้ยังมีเทอมอื่น ๆ อีกที่ผมไม่ได้พูดถึง เช่น เทอมพิเศษที่อธิบาย Hyperconjugation ซึ่งเป็นปรากฏการณ์ที่ $\pi$-conjugation
นั้นส่งผลต่อการยืดหกของพันธะอย่างไรในโมเลกุล โดยเราสามารถแบ่งประเภทของเทอมเหล่านี้ออกเป็นคลาสได้ดังนี้

\begin{itemize}
    \item Class I: มีเพียงแค่ Harmonic Terms เท่านั้น ไม่มีการเติม Coupling Terms เข้าไป

    \item Class II: มี Anharmonic Terms และ Cross Terms

    \item Class III: มีเทอมพื้นฐานทั้งหมดและมีการเพิ่มเทอมพิเศษ เช่น Huperconjugation เข้าไปด้วย
\end{itemize}

%----------------------------------------
\section{อันตรกิริยาระหว่างโมเลกุล}
\idxth{พลังงานระหว่างโมเลกุล}
\idxen{Intermolecular Energy}
\idxth{อันตรกิริยาระหว่างโมเลกุล}
\idxen{Intermolecular Interactions}
%----------------------------------------

ในหัวข้อที่ \ref{sec:md_ff_covalent_bond} เราได้ศึกษาอันตรกิริยาที่อยู่ภายในโมเลกุลไปแล้วนั่นก็คือพันธะโควาเลนท์ซึ่งเป็นอันกิริยาที่เกิดขึ้น%
ระหว่างอะตอม ในหัวข้อนี้ผู้อ่านจะได้ศึกษาอันตรกิริยาที่เกิดขึ้นระหว่างโมเลกุล เช่น อันตรกิริยาแบบอ่อน (Weak Interaction) ซึ่ง \enquote{อ่อน}
ในที่นี้คือเทียบกับพันธะโควาเลนท์ ตัวอย่างเช่น Dispersion Interaction ที่เกิดขึ้นใน Liquid Argon, Hydrogen Bonding ที่เกิดขึ้นใน
Liquid Water, หรือ Ion-Ion Interaction ที่เกิดขึ้นในสารละลายอิเล็กโทรไลต์ นอกจากนี้แล้วยังมีอันตรกิริยาอื่น ๆ ที่เกิดขึ้นระหว่างโมเลกุล%
ที่เราจะต้องพิจารณาด้วย เช่น อันตรกิริยาแบบไกล (Long-Range Interaction) ที่สามารถเกิดขึ้นได้ภายในโมเลกุลเดียวกันสำหรับโมเลกุลที่มี%
ขนาดใหญ่มาก ๆ ซึ่งพลังงานที่เกิดขึ้นจาก Long-Range Interaction นั้นก็เป็นอีกเทอมที่สำคัญมาก ๆ ที่ทำให้การจำลองโมเลกุลนั้นมีความถูกต้องมากขึ้น
\idxboth{อันตรกิริยาระหว่างโมเลกุล!อันตรกิริยาแบบอ่อน}{Intermolecular Interactions!Weak Interaction}

ในหัวข้อนี้เราจะมาโพกัสพลังงานทั้งหมด 4 เทอมที่สำคัญซึ่งเป็นเทอมพลังงานที่เป็นตัวแทนของ Intermolecular Interaction ได้เป็นอย่างดีครับ

\begin{enumerate}[topsep=0pt,noitemsep]
    \setlength\itemsep{1em}
    \item พลังงานไฟฟ้าสถิตย์ (Electrostatic Energy): เป็นเทอมพลังงานที่อธิบายอันตรกิริยาระหว่างไอออนหรือโมเลกุลที่มีความมีขั้ว

    \item พลังงานเหนี่ยวนำ (Induction Energy): เป็นเทอมพลังงานที่อธิบายถึงการเปลี่ยนแปลงของความหนาแน่นของอิเล็กตรอนภายในโมเลกุล%
          ที่เกิดจากการถูก Polarized ด้วยสนามไฟฟ้าจากโมเลกุลรอบ ๆ ซึ่งส่งผลให้เกิดการเหนี่ยวนำ Electric Moment เช่น Induced Dipole
          Moment

    \item พลังงานผลักแบบใกล้ (Short-Range Repulsion Energy): เป็นเทอมพลังงานที่มาจากอันตรกิริยาแบบผลักระหว่างอิเล็กตรอนภาย%
          ซึ่งถูกอธิบายด้วย Pauli Exclusion Principle

    \item พลังงานแพร่กระจาย (Dispersion Energy): เป็นเทอมพลังงานที่อธิบายการ Correlation ของการเคลื่อนที่ของอิเล็กตรอน
\end{enumerate}

%----------------------------------------
\subsection{Electrostatic Energy}
\idxth{พลังงานระหว่างโมเลกุล!พลังงานไฟฟ้าสถิตย์}
\idxen{Intermolecular Energy!Electrostatic Energy}
\idxth{อันตรกิริยาแบบไฟฟ้าสถิตย์}
\idxen{Electrostatic Interactions}
%----------------------------------------

พลังงานที่เกี่ยวข้องกับ Intermolecular Interaction อันแรกที่เราจะมาศึกษากันนั้นคือพลังงานไฟฟ้าสถิตย์ (Electrostatic Energy)
ซึ่งถือว่าเป็นพลังงานพื้นฐานที่สุดเลยก็ว่าได้ โดยพลังงานไฟฟ้าสถิตย์นั้นเกิดขึ้นมาจากอันตรกิริยาทางไฟฟ้าระหว่างอะตอมที่มีประจุ (Charge)
ภายในโมเลกุล ดังนั้นผมจะเริ่มด้วยการอธิบายเรื่องของประจุก่อนเพราะว่าแรงทางไฟฟ้านั้นเกี่ยวข้องโดยตรงกับประจุ ในทางเคมีควอนตัมนั้น
เรากำหนดการกระจายตัวของประจุภายในโมเลกุลด้วยประจุของนิวเคลียส (Nuclear Charges) $\left\{Z_{I}\right\}$
(โดยที่ $I = 1,2, \ldots, N$ และ $N$ คือจำนวนของอะตอม) และความหนาแน่นของอิเล็กตรอน $\rho(\vec{r})$ ซึ่งมีความสัมพันธ์กันดังนี้

\begin{equation}
    \int \rho(\vec{r}) \mathrm{d} \tau
    =
    n
\end{equation}

\noindent โดยที่ $n$ คือจำนวนของอิเล็กตรอนของโมเลกุล สำหรับ Force Field นั้น วิธีที่ง่ายและตรงไปตรงมาที่สุดที่ใช้ในการแสดงถึงการ%
กระจายตัวของประจุภายในโมเลกุลนั้นคือใช้เซตของประจุเชิงอะตอม (Atomic Charges) $\left\{q_I, I=1,2, \ldots, N\right\}$
กับกฎของคูลอมป์ (Coulomb's Law) สำหรับการอธิบายอันตรกิริยาระหว่างประจุเชิงอะตอม ดังนี้

\begin{equation}
    V
    =
    \sum_{I=1}^N \sum_{J=I+1}^N \frac{q_I q_J}{4 \pi \varepsilon_0 R_{I J}},
\end{equation}

\noindent โดยที่ $R_{I J}$ คือระยะห่างระหว่างอะตอม $I$ กับอะตอม $J$ สำหรับการคำนวณของประจุเชิงอะตอมนั้นง่ายมาก โดยเราก็แค่ทำ%
การนำความหนาแน่นของอิเล็กตรอนของอะตอมที่เราสนใจในโมเลกุลมารวมกับประจุของนิวเคลียสของอะตอมนั้น แต่ปัญหาก็คือว่าเราไม่รู้ว่าเราจะทำ%
การแบ่งโมเลกุล (ซึ่งถูกอธิบายความหนาแน่นของอิเล็กตรอน) ออกเป็นชิ้น ๆ อย่างไรเพื่อทำการกำหนดขอบเขตในการคำนวณการประจายของประจุ

เนื่องจากว่าเราไม่มีคำจำกัดความที่แน่นอนสำหรับประจุเชิงอะตอมเนื่องจากว่าประจุนั้นไม่ใช่ปริมาณที่สามารถวัดได้แม้แต่ในทางทดลอง (Non-Observable)
ดังนั้นในปัจจุบันนี้เรามีทฤษฎีเป็นสิบ ๆ ร้อย ๆ ทฤษฎีเลยที่ถูกพัฒนาขึ้นมาเพื่อนิยามและคำนวณประจุเชิงอะตอม

ผมขอเริ่มต้นด้วยเหตุผลที่ว่าประจุเชิงอะตอมนั้นควรที่จะต้อง Reproduce ค่าของโมเมนต์เชิงไฟฟ้าของโมเลกุลได้ (Molecular Electric Moments)
ดังนั้นเราจึงอ้างได้ว่าโมเลกุลนั้นมีประจุรวมทั้งเป็น

\begin{equation}
    q^{\mathrm{mol}}
    =
    \sum_{I=1}^N q_I
\end{equation}

\noindent สำหรับโมเลกุลที่เป็นกลางหรือประจุเท่ากับศูนย์นั้น $(q^{\mathrm{mol}} = 0)$ ถึงแม้ว่าค่าความคลาดเคลื่อนที่มาจากผลต่างระหว่าง%
ของค่าประจุที่เบี่ยงเบนออกจากศูนย์นั้นจะมีน้อย แต่ว่ามันจะทำให้เกิดค่าความคลาดเคลื่อนของค่าพลังงานเชิงไฟฟ้าสถิตย์ที่เยอะมาก ๆ ได้เช่นกัน
สาเหตุก็เพราะว่าค่าระยะห่าง $(1 / R)$ นั้นขึ้นอยู่กับอันตรกิริยาระหว่างประจุนั่นเอง (Charge-Charge Interactions) สำหรับโมเลกุลที่มี%
ขนาดเล็กนั้นเราสามารถคำนวณค่าไดโพลโมเมนต์เชิงโมเลกุล (Molecular Dipole Moment) ได้ดังนี้

\begin{equation}
    \mu_\alpha^{\mathrm{mol}}
    =
    \sum_{I=1}^N q_I R_{I, \alpha},
\end{equation}

\noindent และคำนวณควอนรูโพลโมเมนต์เชิงโมเลกุล (Molecular Quadrupole Moment)

\begin{equation}
    \Theta_{\alpha \beta}^{\mathrm{mol}}
    =
    \sum_{I=1}^N q_I
    \left(
    \frac{3}{2} R_{I, \alpha} R_{I, \beta}
    -\frac{1}{2} R_{I, \gamma} R_{I, \gamma} \delta_{\alpha \beta}
    \right)
\end{equation}

\noindent ซึ่งโมเมนต์ทั้งสองอันนี้ก็สอดคล้องกับค่าประจุเชิงอะตอมนั่นเอง $q_I$

สำหรับการใช้ประจุเชิงอะตอมมาอธิบายการกระจายของประจุภายในโมเลกุลนั้น เรายังมีเทคนิคอื่นอีกหลายเทคนิคที่สามารถเพิ่มประสิทธิภาพของการใช้%
โมเดลประจุเชิงอะตอมเพื่อทำให้แม่นยำมากยิ่งขึ้น หนึ่งในวิธีที่ตรงไปตรงมาก็คือการเพิ่มเทอมไดโพลโมเมนต์เชิงอะตอม (Atomic Dipole Moment,
$\mu_{I,alpha}$) และควอดรูโพลโมเมนต์เชิงอะตอม (Atomic Quadrupole Moment, $Q_{I,\alpha \beta}$) เข้าไป ซึ่งเราจะได้ว่า%
โมเมนต์เชิงโมเลกุลที่ได้จากการเติมโมเมนต์เชิงอะตอมเข้าไปนั้น มีดังนี้

\noindent Molecular Dipole Moment

\begin{equation}
    \label{eq:mol_moment_mu}
    \mu^{\texttt{mol}_{\alpha}}
    =
    \sum^{N}_{I=1} q_{I} R_{I,\alpha} + \mu_{I,alpha}
\end{equation}

\noindent Molecular Quadrupole Moment

\begin{equation}
    \label{eq:mol_moment_Q}
    Q^{\texttt{mol}_{\alpha \beta}}
    =
    \sum^{N}_{I=1} q_{I} R_{I,\alpha} R_{I,\beta}
    + q_{I,\alpha} R_{I,\beta}
    + R_{I,\alpha} \mu_{I,\beta}
    + Q_{I,\alpha \beta}
\end{equation}

\noindent ซึ่งเราสามารถนำสมการที่ \refeq{eq:mol_moment_Q} มาใช้ในการคำนวณหา Quadrupole Moment ได้โดยใช้สมการดังต่อไปนี้

\begin{equation}
    \Theta_{\alpha \beta}
    =
    \frac{3}{2} Q_{\alpha \beta} - \frac{1}{2} Q_{\gamma \gamma} \delta_{\alpha \beta}
\end{equation}

นอกจากนี้เรายังมีกรณีพิเศษอีกบางกรณีที่เราจำเป็นที่จะต้องนำมาพิจารณาเพิ่มเติมเพื่อใช้อธิบายคุณสมบัติทางเคมีของโมเลกุล เช่น แนวคิดของการใช้%
ประจุพิเศษ (Extra Chrage) แล้ววางประจุอันนี้ไว้ด้านนอกของอะตอมซึ่งมีชื่อเรียกว่า Virtual Charge ซึ่งถูกนำมาใช้ในการอธิบายอิเล็กตรอนคู่%
โดดเดี่ยวของอะตอม จริง ๆ แล้วเราสามารถพบกรณีพิเศษแบบนี้ได้แม้แต่ในโมเลกุลเล็ก ๆ หรือระบบง่าย ๆ เช่น โมเลกุลน้ำหรือระบบที่มีอิเล็กตรอน%
ที่เกี่ยวข้องกับพันธะ $\pi$

ถ้าอ่านมาถึงตรงนี้แล้วอย่าเพิ่งสับสนนะครับว่าเราสามารถใช้ได้แค่ประจุเชิงอะตอมได้เพียงอย่างเดียว อย่างที่ผมบอกไปว่าเรามีหลายโมเดลที่เราสามารถนำ%
มาใช้ในการพัฒนา Force Field เพื่อให้ครอบคลุมอันตรกิริยาเชิงไฟฟ้าสถิตย์ แต่ว่าการใช้ประจุเชิงอะตอมนั้นเป็นวิธีที่ได้รับความนิยมมากที่สุดเพราะว่า%
มีความ General มากกว่าวิธีอื่น ๆ และสามารถนำไปใช้ได้กับระบบหลาย ๆ อัน (Systematic Approach) ได้อย่างตรงไปตรงมา

คำถามถัดมาคือ \enquote{แล้วเราจะคำนวณประจุเชิงอะตอมได้ยัง?} คำตอบก็คือเราสามารถใช้เทคนิคทางเคมีควอนตัมได้แต่ว่าเทคนิคนั้นมีเป็นสิบ ๆ
ทฤษฎีเลยที่ถูกเสนอขึ้นมาเพื่อใช้ในการคำนวณประจุเชิงอะตอม วิธีอันหนึ่งที่ถึงแม้ว่าจะโบราณมาก ๆ แล้วก็ตามแต่ก็ยังได้รับความนิยมมาจนถึงปัจจุบัน%
ก็คือนิยามของประจุของมุลลิเกนหรือประจุมุลลิเกน (Mulliken Charge) ซึ่งได้รับการเสนอมาตั้งแต่ปี ค.ศ. 1955, แล้วก็มีทฤษฎีประจุของเฮิร์ชเฟลด์
(Hirshfeld Charge) ซึ่งถูกเสนอในปี ค.ศ. 1977 ซึ่งถูกนำมาใช้อย่างมากในการสร้างพารามิเตอร์ที่ใช้ใน Force Field โดยนำมารวมกันกับ%
ค่าจากการทดลอง สำหรับโมเลกุลขนาดเล็กนั้น เราสามารถคำนวณพารามิเตอร์ของประจุเชิงอะตอมได้โดยการพิสูจน์จากไดโพลโมเมนต์เชิงโมเลกุลและ%
ควอดรูโพลโมเมนต์ แต่ว่าถ้าเป็นกรณีของโมเลกุลที่มีขนาดใหญ่นั้น เราไม่สามารถทำได้เพราะว่าไดโพลโมเมนต์ของโมเลกุลขนาดใหญ่นั้นเกิดขึ้นมาจาก%
ผลรวมของ Contribution หลาย ๆ อันของไฟฟ้าสถิตย์ของอะตอมแต่ละตัวในโมเลกุลซึ่งมันมีความเฉพาะเจาะจงมากเกินไป (มีความ Local มากเกินไป)

%----------------------------------------
\subsubsection{Electronegativity Equalization Model}
\idxen{Electronegativity Equalization Model}
%----------------------------------------

ในหัวข้อย่อยอันนี้เราจะมาดูตัวอย่างของโมเดลที่สามารถคำนวณประจุเชิงอะตอทในโมเลกุลได้ โมเดลนั้นก็คือ Electronegativity Equalization
Model (EEM) ซึ่งจะใช้หลักการที่ว่าประจุของอะตอมแต่ละตัวในโมเลกุลนั้นสามารถที่จะถูกอธิบายได้ด้วยค่า Atomic Electronegativity ($\xi_{I}$)
และค่า Atomic Chemical Hardness ($\eta_{I}$) ถ้าหากว่าค่า Electronegativity ของอะตอม 2 อันนั้นแตกต่างกัน ประจุจะไหล (Flow)
จากอะตอมอันแรกไปอะตอมอันที่สองจนกว่าค่า Molecular Electronegativity นั้นจะมีค่าเฉลี่ยที่เท่า ๆ กันในทุก ๆ ตำแหน่งของโมเลกุล%
\autocite{mortier1986,ionescu2013} ซึ่งค่า Molecular Electronegativity ที่ว่านี้ก็คือ Chemical Potential นั่นเอง
นอกจากนี้แล้วยังมีปริมาณอีกหนึ่งตัวที่ทำให้อะตอมนั้นมีประจุก็คืองาน (Work) ซึ่งสามารถหาได้จากค่า Chemical Hardness หรือ Capacitance
ของอะตอม ยิ่งไปกว่านั้นทั้ง Electronegativity และ Chemical Hardness นั้นต่างก็เป็น Concept หลักที่เรานำมาใช้ใน Density Functional
Theory อีกด้วย\autocite{parr1994a,yang1998}

สำหรับการคำนวณประจุเชิงอะตอมด้วยวิธี EEM นั้น ผมขอเริ่มต้นด้วยสมการของพลังงานของโมเลกุลที่มี $N$ อะตอม, $V$, ดังนี้

\begin{equation}
    V
    =
    \sum_I^N \xi_I q_I+\frac{1}{2}
    \sum_I^N \eta_I q_I^2+\frac{1}{2}
    \sum_{I, J \neq I}^N q_I T_{I J}^{(0)} q_J
    + \mu\left(q^{\mathrm{mol}}-\sum_I^N q_I\right)
\end{equation}

\noindent โดยที่เทอมที่ 3 นั้นเป็น Coulomb Interaction ระหว่างประจุเชิงอะตอม 2 อัน $(q_I$ และ $q_J)$ ส่วนเทอมสุดท้ายนั้นเป็นเทอม%
ที่เป็นตัวบังคับให้ประจุรวมของโมเลกุล $(q^{\mathrm{mol}})$ นั้นมีค่าเท่าเดิมเสมอ แล้วก็ $\mu$ นั้นเป็น Lagrangian Multiplier

ในการหาประจุเชิงอะตอมนั้นสามารถทำได้โดยการปรับค่าพลังงานของโมเลกุลให้ต่ำที่สุดซึ่งก็คือการหาอนุพันธ์ของพลังงานเทียบกับประจุเชิงอะตอมและ
Lagrangian Multiplier ซึ่งจะทำให้เราได้ว่า

\begin{equation}
    \frac{\partial V}{\partial q_K}
    =
    0
    = \xi_K+\eta_K q_K+\sum_{L \neq K}^N T_{K L}^{(0)} q_L-\mu
\end{equation}

\noindent และ

\begin{equation}
    \frac{\partial V}{\partial \mu}
    =
    0
    =
    q^{\mathrm{mol}}-\sum_I^N q_I
\end{equation}

\noindent โดยที่เทอมสุดท้ายนั้นก็คือเทอมบังคับ (Constraint Term) นั่นเอง ในการหาประจุเชิงอะตอมของอะตอมแต่ละตัวนั้นสามารถหาได้จาก%
การสร้างเมทริกซ์สำหรับการแก้สมการเชิงเส้นทั้งหมด $N+1$ ซึ่งสมการเชิงเส้นทั้งหมดนี้นั้นมีความพัวพันหรือ Coupled กันอยู่และเราสามารถแก้ได้%
โดยใช้เทคนิคการประมาณเชิงตัวเลขแบบมาตรฐานทั่วไป

\begin{equation}
    \left(\begin{array}{ccccc}
        \eta_1        & T_{12}^{(0)}  & \ldots & T_{1 N}^{(0)} & 1      \\
        T_{21}^{(0)}  & \eta_2        & \ldots & T_{2 N}^{(0)} & 1      \\
        \vdots        & \vdots        & \ddots & \vdots        & \vdots \\
        T_{N 1}^{(0)} & T_{N 2}^{(0)} & \ldots & \eta_N        & 1      \\
        1             & 1             & \ldots & 1             & 0
    \end{array}\right)\left(\begin{array}{c}
        q_1    \\
        q_2    \\
        \vdots \\
        q_N    \\
        \mu
    \end{array}\right)
    =
    \left(\begin{array}{c}
        \xi_1  \\
        \xi_2  \\
        \vdots \\
        \xi_N  \\
        q^{\mathrm{mol}}
    \end{array}\right)
\end{equation}

\noindent ซึ่งถ้าหากเราทำการย้ายเมทริกซ์ด้านซ้ายสุดไปอยู่ทางด้านขวาของสมการ (ก็คือ Inverse ของเมทริกซ์) เราจะได้เวกเตอร์ที่ประกอบไปด้วย%
สมาชิกที่เป็นค่าประจุเชิงอะตอมของแต่ละอะตอม ดังนี้

\begin{equation}
    \left(\begin{array}{c}
            q_1    \\
            q_2    \\
            \vdots \\
            q_N    \\
            \mu
        \end{array}\right)
    =
    \left(\begin{array}{ccccc}
        \eta_1        & T_{12}^{(0)}  & \ldots & T_{1 N}^{(0)} & 1      \\
        T_{21}^{(0)}  & \eta_2        & \ldots & T_{2 N}^{(0)} & 1      \\
        \vdots        & \vdots        & \ddots & \vdots        & \vdots \\
        T_{N 1}^{(0)} & T_{N 2}^{(0)} & \ldots & \eta_N        & 1      \\
        1             & 1             & \ldots & 1             & 0
    \end{array}\right)^{-1}\left(\begin{array}{c}
            \xi_1  \\
            \xi_2  \\
            \vdots \\
            \xi_N  \\
            q^{\mathrm{mol}}
        \end{array}\right)
\end{equation}

สรุปก็คือในการคำนวณประจุเชิงอะตอมนั้น เราจำเป็นที่จะต้องแก้สมการด้านบนเพื่อที่จะหาค่า $\xi_I$ และ $\eta_I$ ซึ่งทั้งสองค่านี้นั้นขึ้นอยู่กับชนิด%
ของอะตอม ซึ่งวิธีการหาค่าพารามิเตอร์ของทั้งสองค่าสำหรับแต่ละอะตอมแต่ละชนิดนั้นเรียกว่า Parameterization ซึ่งสามารถทำได้หลายวิธีด้วยกัน เช่น
ใช้ Delft Molecular Mechanics (DMM) Force Field ซึ่งเป็นเทคนิคที่ถูกพัฒนาขึ้นมาเพื่อทำการปรับค่าพารามิเตอร์ของสารประกอบอินทรีย์หรือ
Hydrocarbon โดยการใช้ข้อมูลจากผลการทดลอง

วิธี EEM มีประโยชน์มากโดยเฉพาะการนำมาใช้ในทางเคมีอินทรีย์เพื่อใช้ในการบอกถึงโอกาสของการเกิดปฏิกิริยา Electrophilic และ Nucleophilic
Attack ที่เกิดขึ้นในโมเลกุลเนื่องจากว่าปฏิกิริยาเคมีทั้งสองประเภทนี้ขึ้นอยู่กับความแตกต่างเชิงอิเล็กทรอนิกส์ของค่าศักย์เชิงไฟฟ้าสถิตย์ภายในโมเลกุล

ถ้าหากว่าอยากลองใช้ EEM เพื่อคำนวณประจุเชิงอะตอมสามารถใช้งานได้ที่เว็บไซต์ \url{https://webchem.ncbr.muni.cz/Platform/ChargeCalculator}

\begin{figure}[htbp]
    \centering
    \includegraphics[width=0.8\linewidth]{fig/EEM-paracetamol.png}
    \caption{ตัวอย่างการแสดงผลประจุเชิงอะตอมของโมเลกุล Paracetamol ซึ่งคำนวณด้วยวิธี EEM}
    \label{fig:EEM_paracetamol}
\end{figure}

%----------------------------------------
\subsubsection{Hydrogen Bonding}
\idxth{พันธะไฮโดรเจน}
\idxen{Hydrogen Bonding}
%----------------------------------------

พันธะไฮโดรเจน (Hydrogen Bonding) นั้นเป็นหนึ่งในพันธะที่อธิบายได้ยากมาก ๆ ในทางทฤษฎี ซึ่งทำให้การสร้างโมเดลที่สามารถอธิบาย Hydrogen
Bond นั้นมีความท้าทายไปด้วยโดยเฉพาะการทำให้ Force Field นั้นนำไปใช้ได้กับระบบที่มี Hydrogen Bond แบบที่เข้มมาก ๆ

ตัวอย่างอันหนึ่งของการโมเดล Hydrogen Bonding ใน Force Field นั้นก็คือ Force Field ที่ชื่อว่า YETI\autocite{vedani1988}
ซึ่งมีการเพิ่มเทอมพิเศษเข้าไป ดังนี้

\begin{equation}
    V_{\texttt{YETI}}
    =
    \left(
    \frac{A}{R^{12}_{H \cdots O}}
    - \frac{C}{R^{10}_{H \cdots O}}
    \right)
    \cos^{2} \theta
    \cos^{4} \omega
\end{equation}

\noindent โดยที่ $A$ กับ $C$ นั้นเป็นพารามิเตอร์เฉพาะสำหรับโมเลกุลที่ต้องการศึกษา ซึ่งในที่นี้ก็คือโมเลกุลน้ำ (Water Dimer) นอกจากนี้แล้ว%
จะสังเกตได้ว่าค่าพลังงานนั้นจะขึ้นอยู่กับมุม $\theta$ กับมุม $\omega$ อีกด้วย ซึ่งอันนี้เป็นความตั้งใจของผู้พัฒนา Force Field ที่ต้องการให้โมเดล%
นี้สามารถอธิบาย Hydrogen Bond ที่ขึ้นอยู่กับ Oreintation ของโมเลกุล

%----------------------------------------
\subsection{Induction Energy}
\idxth{พลังงานระหว่างโมเลกุล!พลังงานเหนี่ยวนำ}
\idxen{Intermolecular Energy!Induction Energy}
\idxth{โพลาไรเซชั่นเชิงอิเล็กทรอนิกส์}
\idxen{Electronic Polarization}
%----------------------------------------

พลังงานที่เกี่ยวข้องกับ Intermolecular Interaction อันที่สองก็คือพลังงานเหนี่ยวนำ (Induction Energy) ซึ่งเป็นพลังงานที่เกิดขึ้นมาจาก%
การโพลาไรซ์ของโมเลกุล (Polarization) ซึ่งผู้อ่านอาจจะสงสัยว่าแล้ว Polarization นั้นคืออะไรกันแน่? คำตอบก็คือ Polarization
นั้นเกิดขึ้นเมื่อเราทำการใส่สนามไฟฟ้า (Electric Field) เข้าไปให้กับโมเลกุลซึ่งเป็นการรบกวนโครงสร้างเชิงอิเล็กทรอนิกส์ของโมเลกุลแบบหนึ่ง
เมื่อโมเลกุลนั้นถูกรบกวนด้วยสนามไฟฟ้า สิ่งที่เกิดขึ้นคือกลุ่มก้อยของอิเล็กตรอนซึ่งมีประจุลบรอบ ๆ นิวเคลียสซึ่งมีประจุบวกภายในอะตอมนั้นมีการเปลี่ยน%
ทิศทางไปในทิศตรงข้ามกับสนามไฟฟ้า และทำให้เกิดการแบ่งของประจุภายในอะตอมซึ่งทำให้ด้านหนึ่งของอะตอมนั้นมีความเป็นบวกและอีกด้านหนึ่งนั้น%
มีความเป็นลบ ทำให้โมเลกุลนั้นมีความเป็นขั้วทางไฟฟ้าเกิดขึ้นมาและทำให้เกิดอันตรกิริยาทางไฟฟ้าสถิตย์กับโมเลกุลอื่นรอบ ๆ ได้ ดังนั้นเราจึงเรียก%
ปรากฏการณ์แบบนี้ว่า Induction Interaction นั่นเอง ซึ่งในทางเคมีเชิงคำนวณนั้นเราจำเป็นที่จะต้องใส่เทอมของพลังงานเหนี่ยวนำนี้เข้าไปใน
Force Field ด้วยเพื่อเพิ่มความแม่นยำให้กับโมเดล

เราใช้ Polarizability $(\alpha_{\alpha \beta})$ ในการอธิบายความสามารถในการเกิด Polarization ของโมเลกุลภายใต้สนามไฟฟ้า
$(E_{\beta})$ โดยที่ $(\alpha_{\alpha \beta})$ นั้นเป็นการตอบสนองเชิงเส้น (Linear Response) ต่อสนามไฟฟ้า ซึ่งมีนิยามดังต่อไปนี้

\begin{equation}
    \label{eq:induced_dipole_moment}
    \mu_\alpha^{\text {ind }}
    =
    \alpha_{\alpha \beta} E_\beta
\end{equation}

\noindent โดยที่ $\mu_\alpha^{\text {ind }}$ คือไดโพลโมเมนต์ที่ถูกเหนี่ยวนำ (Induced Dipole Moment) เราสามารถทำการ
Generalize สมการที่ \eqref{eq:induced_dipole_moment} ให้อ้างอิงกับการเกิดโพลาไรเซชั่นของอะตอม (Atomic Polarizability)
$(\alpha_{I, \alpha \beta})$ ซึ่งจะทำให้เรามีสมการของ Induced Dipole Moment สำหรับกรณีปกติ ดังนี้

\begin{equation}
    \mu_{I, \alpha}^{\text {ind }}
    =
    \alpha_{I, \alpha \beta} E_{I, \beta}^{\mathrm{tot}}
\end{equation}

\noindent โดยที่ $\mu_{I, \alpha}^{\text {ind }}$ คือ Atomic Induced Dipole Moment และ $E_{I, \beta}^{\text {tot }}$
คือผลรวมของสนามไฟฟ้าทั้งหมดที่กระทำต่ออะตอม $I$ โดยสนามไฟฟ้าทั้งหมดนั้นก็จะมีการรวมสนามไฟฟ้าจากภายนอกที่เราใส่เข้าไปและสนามไฟฟ้า%
ที่มาจากประจุเชิงอะตอมรอบ ๆ โมเลกุลด้วย เป็นต้น

สำหรับหน้าตาของพลังงานเหนี่ยวนำที่เราจะนำเข้าไปใส่ใน Force Field ของเรานั้นจะมีการรวมพลังงาน 3 อันเข้าไว้ด้วยกันคือ ไฟฟ้าสถิตย์
(Electrostatic Energy), พลังงานที่เกิดขึ้นจากตัวของอะตอมเอง (Self-Energy), และพลังงานที่เกิดจากอันตรกิริยาระหว่างไดโพล-ไดโพล
(Dipole-Dipole Interaction Energy) ซึ่งพลังงานเหนี่ยวนำ $(V_{\text {ind }})$ มีสมการดังต่อไปนี้\autocite{vesely1977}

\begin{equation}
    V_{\text {ind }}
    =
    -\frac{1}{2} \sum_{I, J
        \neq I}^N \mu_{I, \alpha}^{\text {ind }} T_{I J, \alpha \beta}^{(2)}
    \mu_{J, \beta}^{\text {ind }}
    + \sum_I^N V_{I, \text { self }}
    - \sum_I^N \mu_{I, \alpha}^{\text {ind }} E_{I, \alpha}^{\text {ext }}
\end{equation}

\noindent โดยที่ $\mu_{I, \alpha}^{\text {ind }}$ คือ Induced Dipole Moment ของอนุภาค $I$,
$T_{I J, \alpha \beta}^{(2)}$ คือเทนเซอร์ที่อธิบายอันตรกิริยา (ขออนุญาตไม่ลงรายละเอียดครับ), และ $V_{I, \text { self }}$
คือพลังงานอนุภาคเอง (Self-Energy) ซึ่งมีสมการดังนี้

\begin{equation}
    V_{I, \text { self }}
    =
    \frac{1}{2}\left(\alpha_{I, \alpha \beta}\right)^{-1} \mu_{I, \alpha}^{\text {ind }} \mu_{I, \beta}^{\text {ind }}
\end{equation}

ระบบโมเลกุลที่ถูกโพลาไรซ์นั้นจะมีการเปลี่ยนแปลงของพลังงานเหนี่ยวที่ลดลงซึ่งก็คือการ Minimization นั่นเอง ดังนั้นเราจึงกำหนดเงื่อนไขขึ้นมาได้ดังนี้

\begin{equation}
    \label{eq:der_induction_energy}
    \frac{\partial V_{\text {ind }}}{\partial \mu_{K, \gamma}^{\text {ind }}}
    =
    0
    =
    -\left(\sum_{J \neq K}^N T_{K J, \gamma \beta}^{(2)} \mu_{J, \beta}^{\text {ind }}\right)+\left(\alpha_{K, \beta \gamma}\right)^{-1} \mu_{K, \beta}^{\text {ind }}-E_{K, \gamma}^{\text {ext }}
\end{equation}

\noindent ซึ่งมีความหมายก็คือโมเลกุลนั้นจะไม่มีการเปลี่ยนแปลงของพลังงานเหนี่ยวนำเมื่อเทียบกับไดโพลโมเมนต์เหนี่ยวนำเมื่อมีการเกิด Polarization
ซึ่งจะทำให้เราได้ว่า

\begin{equation}
    \label{eq:atomic_induced_dipole_moment}
    \mu_{K, \beta}^{\text {ind }}
    =
    \alpha_{K, \beta \gamma}
    \left(
    E_{K, \gamma}^{\mathrm{ext}}
    + \sum_{J \neq K}^N T_{K J, \gamma \beta}^{(2)} \mu_{J, \beta}^{\mathrm{ind}}
    \right)
\end{equation}

ไดโพลโมเมนต์เหนี่ยวนำเชิงอะตอม (Atomic Induced Dipole Moment) นั้นสามารถหาได้จากการ Coupling กันของสมการทั้งหมด $3 N$
สมการหรือที่เรียกว่า (Coupled Equations) ถ้าหากว่าเรารวมผลของ Polarizability เข้าไปใน Force Field เราจะได้ว่าการคำนวณ%
ไดโพลโมเมนต์เหนี่ยวนำนั้นจะใช้เวลาในการคำนวณนานมาก ๆ เมื่อเทียบกับเทอมอื่นของพลังงานเหนี่ยวนำ เพราะว่าไดโพลโมเมนต์เหนี่ยวนำนั้นเป็น%
เทอมของปัญหาแบบ Many-Body หรือปัญหาที่ขึ้นกับจำนวนของอนุภาคทุกอนุภาคในระบบ

นอกจากนี้เรายังสามารถทำสมการที่ \eqref{eq:der_induction_energy} ให้มีหน้าตาที่ง่ายขึ้นได้โดยการใช้สมการที่
\eqref{eq:atomic_induced_dipole_moment} ดังนี้

\begin{equation}
    \begin{aligned}
        V_{\text {ind }}
         & = -\frac{1}{2} \sum_{I, J \neq I}^N
        \mu_{I, \alpha}^{\text {ind }} T_{I J, \alpha \beta}^{(2)} \mu_{J, \beta}^{\text {ind }}
        +\frac{1}{2} \sum_I^N \mu_{I, \alpha}^{\text {ind }}
        \left(
        E_{I, \alpha}^{\mathrm{ext}}
        +\sum_{J \neq I}^N T_{I J, \alpha \beta}^{(2)} \mu_{J, \beta}^{\text {ind }}
        \right)
        -\sum_I^N \mu_{I, \alpha}^{\text {ind }} E_{I, \alpha}^{\text {ext }}                  \\
         & = -\frac{1}{2} \sum_I^N \mu_{I, \alpha}^{\text {ind }} E_{I, \alpha}^{\text {ext }}
    \end{aligned}
\end{equation}

\noindent ซึ่งจะเห็นว่าเทอม Self-Energy นั้นจะทำให้เทอมอันตรกิริยา Induced Dipole-Induced Dipole กับเทอม Indiced Dipole
และเทอมสนามไฟฟ้านั้นหายไปครึ่งหนึ่ง แล้วถ้าหากว่าเราทำใส่แทนสมการที่ \eqref{eq:atomic_induced_dipole_moment} เข้าไป เราจะได้%
สมการที่สามารถนำไปใช้ในการคำนวณพลังงานเหนี่ยวนำสำหรับ Force Field ดังนี้

\begin{equation}
    V_{\mathrm{ind}}
    =
    -\frac{1}{2} \sum_I^N \alpha_{I, \alpha \beta}
    \left(
    E_{I, \beta}^{\mathrm{ext}}
    + \sum_{J \neq I}^N T_{I J, \beta \gamma}^{(2)} \mu_{J, \gamma}^{\mathrm{ind}}
    \right)
    E_{I, \alpha}^{\mathrm{ext}}
\end{equation}

%----------------------------------------
\subsection{Dispersion และ Short-Range Repulsion}
%----------------------------------------

ในหัวนี้เราจะมาศึกษาอันตรกิริยาระหว่างโมเลกุลอีกแบบหนึ่งซึ่งไม่แตกต่างจากอันตรกิริยา 2 หัวข้อก่อนหน้านี้ที่เพิ่งได้ศึกษาไป ยกตัวอย่างเช่น
ถ้าเราสนใจโมเลกุลอาร์กอนที่มีสถานะเป็นของเหลว (Liquid Argon) และไม่มีการใส่สนามไฟฟ้าภายนอกเข้าไปให้กับระบบโมเลกุลอันนี้
สิ่งที่เกิดขึ้นคือจะไม่มีอันตรกิริยา Electronic และ Polarization เกิดขึ้นมา แต่ว่าเรายังมีพลังงานระหว่างโมเลกุลอยู่อีก
ซึ่งนั่นก็คือพลังงานการแพร่กระจาย (Dispersion Energy) และพลังงานที่เกิดจากแรงผลักแบบพิสัยใกล้ (Short-Range Repulsion Energy)
ถ้าพร้อมแล้วก็ลุยกันเลยครับ

%----------------------------------------
\subsubsection{Dispersion Energy}
\idxth{การแพร่กระจาย}
\idxen{Dispersion}
\idxth{พลังงานระหว่างโมเลกุล!พลังงานแพร่กระจาย}
\idxen{Intermolecular Energy!Dispersion Energy}
%----------------------------------------

พลังงานการแพร่กระจาย (Dispersion Energy) เกิดจากอันตรกิริยาที่มาจากการ Correlation กันของการเคลื่อนที่ของอิเล็กตรอนในโมเลกุล
โดยเรามีสมการที่ชื่อว่า London Equation ที่ถูกพัฒนาขึ้นมาเพื่อใช้ในการคำนวณ Dispersion Energy โดยพิสูจน์มาจากทฤษฎี Second-Order
Perturbation Theory\footnote{London Dispersion มีชื่อเรียกเต็ม ๆ ว่า London Dispersion Force และยังมีชื่อเรียกอื่นอีก เช่น
    London Forces, Instantaneous Dipole-Induced Dipole Forces, Fluctuating Induced Dipole Bonds หรืออาจจะเรียกว่า
    van der Waals (vdW) Force ก็ได้เพราะว่า London Force นั้นจัดว่าเป็น vdW แบบหนึ่ง (ซึ่งจริง ๆ แล้วก็ไม่ถูกซะทีเดียว)}
ซึ่ง London Energy มีสมการดังต่อไปนี้\autocite{eisenschitz1930,london1930,london1937}

\begin{equation}
    \label{eq:dispersion_energy}
    V_{\text{disp}}
    =
    - \frac{C_{6}}{R^{6}}
\end{equation}

\noindent โดยที่ $C_{6}$ คือพารามิเตอร์ที่มีค่าเป็นบวกเสมอซึ่งสมเหตุสมผลกับการที่ Dispersion นั้นจะต้องมีค่าเป็นลบเสมอ (เป็นแรงดึงดูด)
นอกจากนี้จะเห็นได้ว่าพลังงาน Dispersion นั้นจะแปรผกผันกับ $R^{-6}$ ตีความง่าย ๆ ก็คือ Dispersion นั้นเป็นอันตรกิริยาหรือแรงที่อ่อนมาก ๆ
ยิ่งอะตอมหรือโมเลกุลอยู่ห่างกันมากเท่าไหร่ ค่า Dispersion Force นั้นก็จะลดลงเยอะมาก และจริง ๆ แล้วเทอมที่แสดงในสมการที่
\eqref{eq:dispersion_energy} นั้นเป็นเพียงแค่เทอม ๆ หนึ่งจากอนุกรมผลรวมของ Range-Separated Interaction ซึ่งเราสามารถเขียน%
พลังงาน Dispersion ให้มีความสมบูรณ์มากขึ้นได้โดยการรวมเทอมอันดับสูง ๆ จาก Perturbation Expansion ได้ดังนี้

\begin{equation}
    \label{eq:dispersion_energy_expansion}
    V_{\text{disp}}
    =
    - \frac{C_{6}}{R^{6}}
    - \frac{-C_{8}}{R^{8}}
    - \frac{-C_{10}}{R^{10}}
    + \dots
\end{equation}

\noindent อนุกรมด้านบนนี้จริง ๆ แล้วก็คือ Taylor Expansion ของ $\frac{1}{R}$ ซึ่งปกติแล้วเรามักจะทำการตัดอนุกรมด้านบนให้เหลือ%
แค่เทอม $R^{-6}$ เท่านั้นเพื่อความง่ายต่อการคำนวณเพราะว่าเทอมสูง ๆ นั้นมีค่าน้อยมากนั่นเอง

%----------------------------------------
\subsubsection{Repulsion Energy}
\idxth{การผลัก}
\idxen{Repulsion}
\idxth{พลังงานระหว่างโมเลกุล!พลังงานผลัก}
\idxen{Intermolecular Energy!Repulsion Energy}
%----------------------------------------

พลังงานการผลักนั้นมาจากหลักการการกีดกันของเพาลี (Pauli Exclusion Principle) ซึ่งมีใจความว่าอิเล็กตรอนนั้นไม่สามารถมีสถานะทาง%
ควอนตัมที่เหมือนกันพร้อม ๆ กันได้ ตัวอย่างเช่น ถ้าหากว่าเรามีอะตอมของอาร์กอน 2 อะตอมอยู่ใกล้ ๆ กัน (เป็นระบบ Closed Shell)
อาร์กอนทั้ง 2 อะตอมนี้จะผลักกันและสอดคล้องกับการที่ทำให้ Pauli Exclusion Principle นั้นยังเป็นจริงอยู่ โดยพลังงานที่เกิดขึ้นจากการผลัก%
กันระหว่างอะตอมนั้นมักจะถูกพิจารณาหรือศึกษาโดยการแบ่งออกเป็น 2 กรณี ก็คือแรงผลักที่เกิดขึ้นในพิสัยใกล้ (Short-Range) และแรงผลักที่เกิดขึ้น%
ในพิสัยไกล (Long-Range) โดยเราสามารถคำนวณหา Repulsion Energy ได้จากการใช้อันตรกิริยาระหว่างโมเลกุลทั้ง 3 อันก่อนหน้านี้ที่เราเพิ่ง%
ได้ศึกษาไปก็คือ Electrostatic, Induction และ Dispersion Forces มาใช้ในการสร้าง Approximation สำหรับ Repulsion Energy
ซึ่งมีสมการดังนี้

\begin{equation}
    V_{\text{rep}}
    =
    V_{\text{interaction}}
    - V_{\text{electrostatic}}
    - V_{\text{induction}}
    V_{\text{dispersion}}
\end{equation}

\noindent จะเห็นได้ว่าสมการด้านบนนั้นมีการรวมความคลาดเคลื่อนของอันตรกิริยาแบบ Short-Range และ Long-Range เข้าไปด้วยในรูปของ%
เทอมพลังงานแต่ละอันที่อันดับสูง ๆ เช่น Dispersion Energy ดังนั้นพลังงานการผลักหรือ Repulsion Energy $(V_{\text{repulsion}})$
นั้นจึงสามารถถูกปรับพารามิเตอร (Parameterization) ได้จากการพลังงานอันตรกิริยา (Interaction Energy) ของระบบที่มีหลายโมเลกุล 
(เช่น Dimer หรือ Cluster) ด้วยวิธีการเคมีควอนตัมแล้วก็ทำการนำค่าพลังงานของเทอมอื่น ๆ มาหักลบออกไป

%----------------------------------------
\subsubsection{Lennard-Jones Potential}
\idxth{พลังงานระหว่างโมเลกุล!พลังงานศักย์ของเลนนาร์ด-โจนส์}
\idxen{Intermolecular Energy!Lennard-Jones Potential}
%----------------------------------------

โมเดลอีกอันหนึ่งที่ถูกพัฒนาขึ้นมาเพื่อใช้ในการอธิบาย Attraction-Repulsion Interaction ระหว่างอะตอมหรือโมเลกุลก็คือโมเดลพลังงานศักย์%
ของเลนนาร์ด-โจนส์ (Lennard-Jones) ซึ่งมีแนวคิดเริ่มต้นมาจากการรวมเทอม Repulsion Energy ที่พิสัยใกล้กับเทอมแรงดึงดูด London 
Dispersion Energy ที่พิสัยไกลเข้าไว้ด้วยกัน ดังสมการต่อไปนี้

\begin{equation}
    V_{\text{LJ}}
    =
    4 \epsilon
    \left(
        \left(
            \frac{\sigma}{R}
        \right)^{12}
        -
        \left(
            \frac{\sigma}{R}
        \right)^{6}
    \right)
    =
    \frac{a}{R^{12}}
    - \frac{C_{6}}{R^{6}}
\end{equation}

\noindent จะเห็นว่าเรามีเทอม $R^{-12}$ อยู่ในสมการ ซึ่งเทอมนี้เป็นเทอมที่อธิบาย Repulsion ซึ่งมีเลขยกกำลังเป็นสองเท่าของเลขยกกำลังของ
$R^{-6}$ ซึ่งมาจากการที่ Lennard-Jones นั้นแก้สมการมาจากระบบของ Statistical Mechanical Model 

สำหรับ Force Field ที่ใช้กับระบบที่มีมากกว่า 1 โมเลกุลนั้นเราสามารถคำนวณ Lennard-Jones Potential ทั้งหมดได้โดยการใช้ Pair-Wise 
Additive ดังนี้ 

\begin{equation}
    \sum^{N_{A}}_{I=1}
    \sum^{N_{B}}_{J=1}
    V_{\text{LJ}}
    =
    4 \epsilon_{IJ}
    \left(
        \left(
            \frac{\sigma_{IJ}}{R_{IJ}}
        \right)^{12}
        -
        \left(
            \frac{\sigma_{IJ}}{R_{IJ}}
        \right)^{6}
    \right)
\end{equation}

\noindent โดยที่ $N_{A}$ และ $N_{B}$ คือจำนวนอะตอมของโมเลกุล $A$ กับโมเลกุล $B$ ส่วนพารามิเตอร์ $\epsilon_{IJ}$ กับ 
$\sigma_{IJ}$ นั้นเป็นพารามิเตอร์ที่ขึ้นกับอันตรกิริยาระหว่างอะตอม $I$ กับอะตอม $J$ ซึ่งจะมีพารามิเตอร์เยอะมาก ๆ ถ้าหากว่าโมเลกุล%
นั้นมีขนาดใหญ่ ซึ่งวิธีที่เราสามารถใช้ในการลดจำนวนพารามิเตอร์ทั้ง 2 ตัวนี้ได้ก็คือการใช้กฎการผสมของ Lorentz-Berthlot ดังนี้ 

\begin{equation}
    \sigma_{IJ}
    =
    \frac{1}{2}
    (\sigma_{II}
    + \sigma_{JJ})
\end{equation}

และ 

\begin{equation}
    \epsilon_{IJ}
    =
    \sqrt{\epsilon_{II} \epsilon_{JJ}}
\end{equation}

%----------------------------------------
\section{สมการของการเคลื่อนที่}
\idxboth{สมการของการเคลื่อนที่}{Equations of Motion}
%----------------------------------------

หัวใจสำคัญของ MD Simulations นั้นก็คืออันตรกิริยาระหว่างโมเลกุลนั่นก็คือ \enquote{แรง (Force)} โดยมีสมการสำคัญ 2 สมการที่ถือได้ว่า%
เป็นสมการหลักของ MD เลยก็ว่าได้ ดังนี้

\begin{equation}
    \label{eq:force_newton}
    m_{i}\bm{\ddot{r}}_{i} = \bm{f}_{i}
\end{equation}

\begin{equation}
    \label{eq:force_der_ener}
    \bm{f}_{i} = -\nabla_{i}V(\bm{r})
\end{equation}

\noindent สมการด้านบนนี้คือสมการการเคลื่อนที่ของนิวตัน (Newtonian Equation of Motion) สำหรับอะตอม $i$ โดยที่เป้าหมายของเรา%
นั้นก็คือการคำนวณแรง $\bm{f}$ ที่กระทำต่ออะตอมซึ่งสามารถคำนวณได้จากพลังงานศักย์ $V(\bm{r})$ นั่นเอง ส่วนเวกเตอร์ $\bm{r}$ นั้น%
ก็คือพิกัดคาร์ทีเซียนของตำแหน่งของอะตอม (นิวเคลียส) ทั้งหมดทุกอะตอมในโมเลกุลซึ่งเป็นพิกัดแบบ 3 มิติ

\begin{equation}
    \bm{r} = (\underbrace{r_{1,x}, r_{1,y}, r_{1,z}}_{\text{อะตอมตัวที่ 1}}, \dots,
    \underbrace{r_{N,x}, r_{N,y}, r_{N,z}}_{\text{อะตอมตัวที่ $N$}})
\end{equation}

โดยในการจำลอง MD นั้นจะเป็นการแก้สมการที่ \eqref{eq:force_newton} และ \eqref{eq:force_der_ener} พร้อม ๆ กันไปเป็นสเต็ป ๆ
ตลอดช่วงระยะเวลาที่ทำการจำลอง โดยระยะห่างระหว่างสเต็ปนั้นเรียกว่า Time Step $(\Delta t)$

%----------------------------------------
\section{ข้อจำกัดของ MD}
%----------------------------------------

วิธี MD นั้นก็เหมือนกับวิธีการจำลองทางคอมพิวเตอร์อื่น ๆ ที่มีข้อจำกัดทั้งในเชิงตัวโมเดลของวิธีเองกับในเชิงทรัพยากรที่ใช้ในการคำนวณ โดยข้อจำกัด%
ของ MD สามารถแบ่งออกได้เป็น 4 ข้อหลัก ๆ ดังนี้

\paragraph{1. Time Scale} สเกลเวลาหรือ Time Scale คือสเกลที่บอกถึงระดับของช่วงเวลาที่ใช้ในการอธิบายปรากฎการณ์หรือพฤติกรรมของโมเลกุล%
หรือระบบที่เราต้องการศึกษา เช่น การสั่นของพันธะโมเลกุลนั้นจะมี Time Scale ในระดับ Femtosecond ดังนั้น Time Scale ที่เหมาะสมสำหรับ%
การกำหนด Time Step นั่นจึงอยู่ที่ประมาณ 1 fs เพราะว่าถ้าหากเรากำหนด Time Step ที่กว้างหรือช้ากว่านี้เช่น 10 fs เราก็จะไม่สามารถติดตาม%
การสั่นของโมเลกุลได้เพราะว่าช่วงระยะเวลาที่ใช้ในการขยับหรือเปลี่ยนตำแหน่งของโครสร้างของโมเลกุลนั้นมากกว่าการสั่นของโมเลกุลหลายเท่า

สำหรับการจำลองเหตุการณ์หรือ Event ในการจำลอง MD นั้นเราควรจะต้องทราบถึงระยะเวลาที่เร็วที่สุดที่เหตุการณ์นั้นสามารถเกิดขึ้นได้ก่อน เช่น
การพับของโปรตีน (Protein Folding) นั้นจะใช้เวลาประมาณ 1 วินาที ดังนั้นถ้าหากเรากำนดให้ Time Step = 1 fs เราจะต้องทำการจำลอง
MD ประมาณ $10^{15}$ สเต็ปถึงจะสามารถจำลองการพับของโปรตีนได้ อย่างไรก็ตามในความเป็นจริงนั้นปรากฎการณ์ต่าง ๆ ของโมเลกุลที่เกิดขึ้นนั้น%
มักจะเกิดขึ้นในช่วงเวลาระดับ Microsecond $(\mu s)$

\paragraph{2. Length Scale} สเกลขนาดหรือ Length Scale คือสเกลที่บ่งบอกถึงขนาดของระบบที่ถูกจำลองซึ่ง Length Scale นี้จะแบ่ง%
ตามขนาดของระบบที่ใช้ในการศึกษา ถ้าหากเราต้องการที่จะศึกษาคุณสมบัติของระบบที่มีขนาดใหญ่ Length Scale ก็จะต้องสอดคล้องกับระบบด้วย
เช่น การจำลองโครงข่ายพอลิเมอร์ (Polymer) เพื่อให้มีความเหมาะสมและมีขนาดใหญ่ของระบบที่ใหญ่มากพอที่จะเป็นตัวแทนของระบบพอลิเมอร์%
ในธรรมชาติจริง ๆ

\paragraph{3. ความแม่นยำของแรงที่คำนวณได้} หัวใจสำคัญของ MD นั้นก็คือการคำนวณแรงที่เป็นอันตรกิริยาระหว่างอะตอมในโมเลกุล ถ้าหาก%
เราใช้วิธีการคำนวณแรงที่มีความแม่นยำสูงก็จะทำให้เราได้แรงที่มีความถูกต้องมาก แต่วิธีการที่มีความแม่นยำสูงนั้นมักจะต้องแลกมาด้วยการคำนวณที่%
สิ้นเปลือง ดังนั้นเรามักจะทำการ Trade-off หรือชั่งน้ำหนักระหว่างการเลือกวิธีในการคำนวณแรงและความสิ้นเปลืองของวิธีนั้น ๆ เพราะอย่าลืมว่า%
เราต้องคำนวณแรงทุก ๆ สเต็ปของการจำลอง MD

%----------------------------------------
\section{ขั้นตอนการจำลอง MD}
%----------------------------------------

การจำลอง MD นั้นโดยปกติแล้วประกอบไปด้วยขั้นตอนดังต่อไปนี้

\paragraph{1. เลือกโมเดล}

\paragraph{2. เตรียมโครงสร้างเริ่มต้น}

\paragraph{3. รันการจำลอง MD}

\paragraph{4. วิเคราะห์ผลการจำลอง MD}

\paragraph{5. คำนวณคุณสมบัติอื่น ๆ เพิ่มเติม}

%----------------------------------------
\section{การจำลอง MD ด้วยเทคนิค Enhanced Sampling}
%----------------------------------------

%----------------------------------------
\section{การวิเคราะห์ผลการจำลอง MD}
%----------------------------------------

%----------------------------------------
\section{แบบฝึกหัด}
%----------------------------------------
