% LaTeX source for ``Ab initio Moleculardynamics''
% Copyright (c) 2023 รังสิมันต์ เกษแก้ว (Rangsiman Ketkaew).

% License: Creative Commons Attribution-NonCommercial-NoDerivatives 4.0 International (CC BY-NC-ND 4.0)
% https://creativecommons.org/licenses/by-nc-nd/4.0/

\chapter{พลวัตเชิงโมเลกุลแบบแอบ อินิชิโอ}
\label{ch:aimd}

%----------------------------------------
\section{การจำลองเชิงโมเลกุลและเทคนิคเชิงคอมพิวเตอร์}
%----------------------------------------

%----------------------------------------
\section{ประวัติศาสตร์ของ Molecular Dynamics}
%----------------------------------------

\begin{itemize}
    \item 1953: Nicholas Metropolis และคณะได้ตีพิมพ์บทความวิจัยเรื่อง \enquote{Equation of State Calculations by Fast 
    Computing Machines}\autocite{metropolis1953} โดยบทความนี้เป็นเสมือนจุดเริ่มต้นของไอเดีย MD เลยก็ว่าได้ โดยเป็นครั้งแรกที่%
    ได้มีการประยุกต์ใช้เทคนิค Monte Carlo เพื่อแก้สมการที่อธิบายคุณสมบัติเชิงกายภาพของระบบที่ประกอบไปด้วยโมเลกุลที่มีอันตรกิริยาต่อกัน 
    โดยขั้นแรกคือสร้างเซตของตัวเลขสุ่ม (Random Number) เพื่อใช้เป็นตัวแทนของ Conformational Space แล้วก็ใช้ค่าของพลังงานเป็นตัว%
    ระบุความน่าจะเป็นของสถานะของระบบที่ศึกษา
    
    \item 1956: Berni J. Alder และ Thomas E. Wainwright ได้ตีพิมพ์บทความเรื่อง \enquote{Phase Transition for a Hard 
    Sphere System}\autocite{alder1957} ซึ่งถือได้ว่าเป็นงานวิจัยที่เป็นจุดเริ่มต้นของ MD เลยก็ว่าได้
    
    \item 1958: เป็นครั้งแรกที่นักวิทยาศาสตร์ค้นพบโครงสร้างสามมิติของโปรตีนได้โดยใช้เทคนิค X-ray โดยเผยแพร่ในบทความ 
    \enquote{A Three-Dimensional Model of the Myoglobin Molecule Obtained by X-Ray Analysis}\autocite{kendrew1958}

    \item 1964: บทความวิจัยเรื่อง \enquote{Correlations in the Motion of Atoms in Liquid Argon}\autocite{rahman1964} 
    โดย Aneesur Rahman ซึ่งเป็นผู้ที่ใช้ MD ในการคำนวณระบบของ Liquid Argon ซึ่งระบบที่ศึกษาตอนนั้นมี Argon ทั้งหมด 864 อะตอม 
    โดยคำนวณด้วยซุปเปอร์คอมพิวเตอร์  CDC 3600 โดยใช้ Lennard-Jones Potential นอกจากนี้ Aneesur Rahman ได้รับการยอบรับว่า%
    เป็นบิดาแห่งพลวัตเชิงโมเลกุลอีกด้วย (The Father of Molecular Dynamics)

    \item 1971: Aneesur Rahman และ Frank H. Stillinger ได้ตีพิมพ์บทความเรื่อง \enquote{Molecular Dynamics Study of 
    Liquid Water}\autocite{rahman1971} ซึ่งเป็นใช้ MD ในการจำลองระบบโมเลกุลน้ำที่มีจำนวนโมเลกุลคือ 216 โมเลกุล 

    \item 1975: Michael Levitt และ Arich Warshel ได้เผยแพร่บทความวิจัยเรื่อง \enquote{Computer Simulation of Protein 
    Folding}\autocite{levitt1975} ซึ่งเป็นครั้งแรกที่มีการนำเทคนิค MD มาใช้ในการจำลองการพับของโปรตีนโดยเป็นการศึกษาการพับของ 
    Bovine Pancreatic Trypsin Inhibitor (BPTI) จากโครงสร้างที่เป็นแบบสายเปิด

    \item 1979: David A. Case และ Martin Karplus ได้จำลองโปรตีนที่มีลิแกนด์เป็นโมเลกุลที่เข้าไปจับกับโปรแกรมด้วยเป็นครั้งแรก 
    โดยได้ตีพิมพ์งานวิจัยเรื่อง \enquote{Dynamics of ligand binding to heme protein}\autocite{case1979}

    \item 1980s: ในช่วงต้น ๆ ทศวรรษ 1980 นั้นเป็นช่วงที่มีการศึกษาชีวโมเลกุลด้วยการจำลอง MD เป็นจำนวนมาก รวมไปถึงมีการคำนวณ 
    Free Energy ด้วย

    \item 1985: Roberto Car Michele Parrinello ได้พัฒนาเทคนิค Car-Parrinello Molecular Dynamics (CPMD) ซึ่งเสนอใน%
    บทความเรื่อง \enquote{Unified Approach for Molecular Dynamics and Density-Functional Theory}\autocite{car1985} 
    โดยเป็นการนำเทคนิค Density Functional Theory มารวมกับ Born-Oppenheimer Molecular Dynamics

    \item 1988: Michael Levitt และ Ruth Sharon ได้คำนวณระบบของโปรตีนที่มีโมเลกุลน้ำเป็นตัวทำละลายและนำเสนอในบทความเรื่อง 
    \enquote{Accurate Simulation of Protein Dynamics in Solution}\autocite{levitt1988}

    \item 1990s: ในช่วงต้น ๆ ทศวรรษ 1990 นั้นก็ได้มีการพัฒนาศักย์ (Potential) ที่ใช้ในวิธี MD รวมถึงเทคนิคการเพิ่มประสิทธิภาพใน%
    การสุ่ม (Enhanced Sampling) อย่างต่อเนื่อง
\end{itemize}

%----------------------------------------
\section{มาศึกษา Molecular Dynamics กันก่อน}
%----------------------------------------

Chemistry
- Intra- and intermolecular interactions (Molecular Mechanics)
- Chemical reactions
- Phase transitions
- Free energy calculations

Materials Science
- Equilibrium thermodynamics
- Phase transitions
- Properties of lattice defects
- Nucleation and surface growth
- Heat/pressure processing
- Ion implantation
- Properties of nanostructures

Biophysics and biochemistry
- Protein folding and structure prediction
- Biocombatibility (cell wall penetration, chemical processes)
- Docking

Medicine
- Drug design and discovery

%----------------------------------------
\section{ทำไมต้อง \textit{Ab initio} Molecular Dynamics}
%----------------------------------------


%----------------------------------------
\section{ประเภทของ \textit{Ab initio} Molecular Dynamics}
%----------------------------------------

เราสามารถแบ่งประเภทของ AIMD ได้โดยแบ่งตามวิธีการที่เราใช้ในการรวมการคำนวณโครงสร้างเชิงอิเล็กทรอนิกส์กับการจำลองพลวัตโมเลกุลเข้าด้วยกัน 
โดยเราสามารถแบ่งออกได้เป็น 3 วิธี ดังนี้ 

\begin{enumerate}
    \item Born-Oppenheimer Molecular Dynamics
    \item Ehrenfest Molecular Dynamics
    \item Car-Parrinello Molecular Dynamics
\end{enumerate}
