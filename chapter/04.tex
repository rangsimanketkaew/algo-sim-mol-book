% LaTeX source for ``Algorithms for Computer Simulation of Molecular Systems''
% Copyright (c) 2023 รังสิมันต์ เกษแก้ว (Rangsiman Ketkaew).

% License: Creative Commons Attribution-NonCommercial-NoDerivatives 4.0 International (CC BY-NC-ND 4.0)
% https://creativecommons.org/licenses/by-nc-nd/4.0/

\chapter{การพัฒนาซอฟต์แวร์สำหรับเคมีเชิงคำนวณ}
\label{ch:software_dev}

%----------------------------------------
\section{การเขียนโปรแกรมทางเคมีเชิงคำนวณ}
%----------------------------------------

อยากจะศึกษาการเขียนโปรแกรมทางเคมีเชิงคำนวณ เช่น เขียนโปรแกรม Density Functional Theory (DFT) หรือ Implement เทคนิค 
Electronic Structure ต่าง ๆ จะเริ่มยังไงดี? ผมขอให้ความเห็นอย่างนี้ครับว่าการจะที่เขียนโปรแกรมทางเคมีคำนวณขึ้นมาสักโปรแกรมนึงนั้น%
ซับซ้อนมาก (ใช้เวลานานและขึ้นอยู่กับว่าเขียนคนเดียวหรือช่วยกันเขียนหลายคน) ดังนั้นเราควรเริ่มต้นศึกษาและทำความเข้าใจกับโปรแกรม (มาตรฐาน) 
ได้รับความนิยมอยู่แล้ว ผมไม่ได้บอกว่าห้ามเริ่มเขียนโปรแกรมใหม่แบบเริ่มจากศูนย์หรือ From Scratch ด้วยตัวเอง แต่ถ้าหากว่าเราเริ่มต้นจาก%
โปรแกรมที่ได้รับความนิยมและใช้งานกันอย่างแพร่หลายอยู่ก็มีข้อดีหลาย ๆ เช่น 

\begin{itemize}
    \item ประหยัดเวลา ไม่ต้องมานั่งศึกษาหรือเขียนโค้ดใหม่เองทั้งหมด
    \item ได้เรียนรู้วิธีการเขียนโค้ดที่มีประสิทธิภาพจากนักพัฒนาคนอื่น ๆ 
    \item เป็นการต่อยอดและพัฒนาโปรแกรมนั้น ๆ ให้ดีขึ้นไปอีกเพราะเราไม่จำเป็นต้องมา Reinvent the Wheel 
    \item เป็นการสร้างเครือข่ายนักวิจัยและความร่วมมือทางวิชาการในระดับนานาชาติ 
\end{itemize}

\noindent อย่างไรก็ตามถ้าหากใครอยากจะเริ่มเขียนโค้ดเองนั้น (ไม่จำเป็นต้องเป็น DFT อย่างเดียว แต่รวมถึงวิธีการจำลองทางคอมพิวเตอร์อื่น ๆ 
มีตัวอย่างคือ Molecular Dynamics หรือ Monte Carlo เป็นต้น) ก็มีข้อดีหลายข้อเหมือนกัน ดังนี้ 

\begin{itemize}
    \item โดยเราได้ทำความเข้าใจการเขียนโปรแกรมอ้างอิงตามสมการทาง Electronic Structure 
    \item ฝึกเขียนโปรแกรมสำหรับการคำนวณทางวิทยาศาสตร์และได้เรียนรู้เทคนิคการประมาณค่าเชิงตัวเลข
    \item ได้ออกแบบโปรแกรมเองและ Implement วิธีใหม่ ๆ ที่โปรแกรมอื่นไม่มี
    \item ต่อยอดเป็นโปรแกรมในรูปแบบเชิงพานิชย์ได้เพราะว่ามีโปรแกรมทางเคมีคำนวณหลาย ๆ โปรแกรมที่ขาย License
\end{itemize}

ประเด็นหรือคำถามสำคัญคือ แล้วถ้าหากอยากจะเริ่มศึกษาโค้ดของวิธีการคำนวณทางเคมีควอนตัม เช่น โปรแกรม Density Functional Theory 
(DFT) ดี ๆ สักตัวนึงจะเริ่มจากไหนดี? ความเห็นของผมคือแนะนำโปรแกรม PySCF โดยมีเหตุผลดังต่อไปนี้

\begin{itemize}
    \item โปรแกรมมีประสิทธิภาพสูง ทำงานได้เร็วและให้ผลการคำนวณที่ถูกต้องและแม่นยำและยังคำนวณได้หลากหลายวิธี
    \item PySCF เขียนด้วย Python เกือบทั้งหมด (87\% เขียนด้วย Python, 12\% เป็นภาษา C ก็คือพวกไลบรารี่ต่าง ๆ ที่เอามาคำนวณ%
    ในส่วนที่ Python อาจจะคำนวณได้ช้า) ดังนั้นจึงง่ายต่อการศึกษา
    \item มีผู้ใช้งานเยอะเนื่องจากว่าโปรแกรม PySCF นั้นสามารถติดตั้งและใช้งานได้ง่าย เตรียมไฟล์ Input ได้ไม่ยุ่งยาก 
    \item มีทีมพัฒนาที่ใหญ่และแข็งแกร่ง ได้รับการสนับสนุนฟีเจอร์และแก้ไข Bug อย่างต่อเนื่อง
\end{itemize}
    
\noindent จากข้อ 1 ถ้าหากเราต้องการจะ Implement วิธีหรือเทคนิคใหม่ ๆ เข้าไปใน PySCF ก็ทำได้ง่ายเพราะว่าเขียนด้วยภาษา Python 
นอกจากนี้โปรแกรมยังสามารถรันบน GPU ได้ด้วย (มี Plugin พิเศษ gpu2pyscf) ตัวโค้ดถูกเขียนและได้รับการปรับปรุงมาเป็นอย่างดี (Well-written) 
มีการวางโครงสร้างของโปรแกรมที่เรียบร้อย แบ่ง Methods ต่าง ๆ ออกเป็น Module ที่ชัดเจนและมีการจัดวาง Function ที่เหมาะสม 
สำหรับผู้อ่านที่สนใจโปรแกรม PySCF ก็ไปดูได้ที่ https://github.com/pyscf/pyscf

เมื่อเราเลือกโปรแกรมได้แล้ว ขั้นตอนต่อมาก็คือพยายามทำความเข้าใจทฤษฎีของหัวข้องานวิจัยที่เราต้องการศึกษา พยายามหาว่าเราสามารถพัฒนาวิธีนั้น ๆ 
ได้อย่างไรเพื่อที่จะปรับปรุงให้มีความถูกต้องในการคำนวณมากขึ้น (พูดง่าย ๆ ก็คือหาจุดที่ทฤษฎีหรือวิธีนั้น ๆ ถูกปรับปรุงได้หรืออาจจะหากรณีที่ทฤษฎี%
นั้นยัง) พยายามหา Solution ที่ได้จากการปรับปรุงทฤษฎีแล้วเขียนออกมาเป็นสมการทางคณิตศาสตร์ที่เราจะนำไป Implement ในลำดับต่อไป
ขั้นตอนต่อไปก็คือการวางแผนการเขียนโค้ดซึ่งสามารถทำได้ด้วยการเขียนโค้ดเทียมหรือ Pseudo Code คร่าว ๆ ก่อนที่เราจะ Implement จริง ๆ 
โดยเราจะต้องคิดเกี่ยวกับการวางโครงสร้างหรือ Structure ของโปรแกรม เช่น แบ่งโค้ดออกเป็นส่วน ๆ เป็น modules/functions/classes/types 
โดยเราควรจะต้องคำนึงถึงการพัฒนาโปรแกรมต่อ ๆ ไปในอนาคตด้วยว่าโปรแกรมของเรานั้นควรจะต้องรองรับฟีเจอร์ใหม่ ๆ ที่นักพัฒนาคนอื่น ๆ จะเข้า%
มาช่วยพัฒนา 

หลังจากที่เรา Implement เสร็จเรียบร้อยแล้วเราควรจะต้องมีการตรวจสอบการทำงานของฟังก์ชันแต่ละฟังก์ชันอย่างสม่ำเสมอเพื่อตรวจสอบค่าที่ได้จาก%
คำนวณว่ามีความถูกต้องไหม เมื่อได้ค่าที่ถูกต้องแล้วขั้นตอนต่อมาก็คือการปรับปรุงหรือทำความสะอาดให้มีประสิทธิภาพและอ่านได้ง่ายขึ้น 
โดยในขั้นตอนนี้เราอาจจะไปศึกษาโค้ดที่คนอื่นเขียนไว้ก็ได้ว่าเขาเขียนยังไง ใช้วิธีหรือเทคนิคอะไรที่ทำให้โค้ดรันได้เร็วและมีประสิทธิภาพ 
นอกจากนี้ยังมีสิ่งอื่น ๆ ที่ต้องทำด้วย เช่น เขียน Comment หรือทำเอกสารประกอบการใช้งาน Documentation เพื่อที่ว่าตัวเราเองหรือคนอื่น ๆ 
ที่มาอ่านหรือแก้โค้ดต่อจากเราจะได้ทำความเข้าใจโค้ดได้ง่ายและไม่ต้องมานั่งศึกษาเองจากศูนย์
