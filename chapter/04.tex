% LaTeX source for ``Algorithms for Computer Simulation of Molecular Systems''
% Copyright (c) 2023 รังสิมันต์ เกษแก้ว (Rangsiman Ketkaew).

% License: Creative Commons Attribution-NonCommercial-NoDerivatives 4.0 International (CC BY-NC-ND 4.0)
% https://creativecommons.org/licenses/by-nc-nd/4.0/

\chapter{การพัฒนาซอฟต์แวร์สำหรับเคมีเชิงคำนวณ}
\label{ch:software_dev}

%----------------------------------------
\section{การเขียนโปรแกรมทางเคมีเชิงคำนวณ}
%----------------------------------------

ถ้าหากผู้อ่านอยากจะศึกษาการเขียนโปรแกรมทางเคมีเชิงคำนวณจะเริ่มยังไงดี เช่น ต้องการเขียนโปรแกรม Density Functional Theory (DFT) 
หรือ Implement วิธีโครงสร้างเชิงอิเล็กทรอนิกส์ (Electronic Structure) ผมขอให้ความเห็นอย่างนี้ครับว่าการจะที่เขียนโปรแกรมทางเคมี%
คำนวณขึ้นมาสักโปรแกรมหนึ่งนั้นใช้เวลามากพอสมควรเพราะว่ามีรายละเอียดที่ซับซ้อนมาก (เวลาที่ใช้ในการเขียนนั้นขึ้นอยู่กับว่าเขียนคนเดียวหรือช่วย%
กันเขียนหลายคน) ดังนั้นผมแนะนำว่าสำหรับผู้ที่เพิ่งเริ่มต้นการเขียนโปรแกรมทางวิทยาศาสตร์ควรศึกษาจากโปรแกรมมาตรฐานที่ได้รับความนิยมอยู่แล้ว 
ผมไม่ได้บอกว่าห้ามเขียนโปรแกรมใหม่เองแบบเริ่มจากศูนย์หรือ From Scratch แต่ถ้าหากว่าเราเริ่มต้นเรียนรู้จากโปรแกรมที่ได้รับความนิยมและใช้งาน%
กันอย่างแพร่หลายอยู่แล้วก็มีข้อดีดังนี้ 

\begin{itemize}
    \item ประหยัดเวลา ไม่ต้องมานั่งศึกษาหรือเขียนโค้ดใหม่เองทั้งหมด
    \item ได้เรียนรู้วิธีการเขียนโค้ดที่มีประสิทธิภาพจากนักพัฒนาคนอื่น ๆ 
    \item เป็นการต่อยอดและพัฒนาโปรแกรมนั้น ๆ ให้ดีขึ้นไปอีกเพราะเราไม่จำเป็นต้องมา Reinvent the Wheel 
    \item เป็นการสร้างเครือข่ายนักวิจัยและความร่วมมือทางวิชาการในระดับนานาชาติ 
\end{itemize}

\noindent อย่างไรก็ตามถ้าหากใครอยากจะเริ่มเขียนโปรแกรมเองนั้น (ไม่จำเป็นต้องเป็น DFT อย่างเดียว แต่รวมถึงวิธีการจำลองทางคอมพิวเตอร์อื่น ๆ 
ด้วย เช่น Molecular Dynamics หรือ Monte Carlo) ก็มีข้อดีหลายข้อเหมือนกัน ดังนี้ 

\begin{itemize}
    \item ได้ทำความเข้าใจการเขียนโปรแกรมอ้างอิงตามสมการทาง Electronic Structure 
    \item ฝึกทักษะการเขียนโปรแกรมสำหรับการคำนวณทางวิทยาศาสตร์และได้เรียนรู้เทคนิคการประมาณค่าเชิงตัวเลข
    \item ได้ออกแบบโปรแกรมเองและ Implement วิธีใหม่ ๆ ที่โปรแกรมอื่นไม่มี
    \item ต่อยอดเป็นโปรแกรมในรูปแบบเชิงพานิชย์ได้เพราะว่ามีโปรแกรมทางเคมีคำนวณหลาย ๆ โปรแกรมที่ขาย License
\end{itemize}

ประเด็นหรือคำถามสำคัญคือ \enquote{ถ้าหากอยากจะเริ่มศึกษาโค้ดของวิธีการคำนวณทางเคมีควอนตัม เช่น โปรแกรม Density Functional 
Theory (DFT) ดี ๆ สักตัวนึงจะเริ่มจากไหนดี?} ความเห็นของผมคือแนะนำให้ศึกษาโปรแกรม PySCF โดยมีเหตุผลดังต่อไปนี้

\begin{itemize}
    \item โปรแกรมมีประสิทธิภาพสูง ทำงานได้เร็วและให้ผลการคำนวณที่ถูกต้องและแม่นยำและยังคำนวณได้หลากหลายวิธี
    \item มีผู้ใช้งานเยอะเนื่องจากว่าโปรแกรม PySCF นั้นสามารถติดตั้งและใช้งานได้ง่าย เตรียมไฟล์ Input ได้ไม่ยุ่งยาก 
    \item PySCF เขียนด้วย Python เกือบทั้งหมด (87\% เขียนด้วย Python, 12\% เป็นภาษา C ก็คือพวกไลบรารี่ต่าง ๆ ที่เอามาคำนวณ%
    ในส่วนที่ Python อาจจะคำนวณได้ช้า) ดังนั้นจึงง่ายต่อการทำความเข้าใจ
    \item มีทีมพัฒนาที่ใหญ่และแข็งแกร่ง ได้รับการสนับสนุนฟีเจอร์และแก้ไข Bug อย่างต่อเนื่อง
\end{itemize}
    
\noindent จากข้อ 1 ถ้าหากเราต้องการ Implement วิธีหรือเทคนิคใหม่ ๆ เข้าไปใน PySCF ก็ทำได้ง่ายเพราะว่าเขียนด้วยภาษา Python 
นอกจากนี้โปรแกรมยังสามารถทำงานด้วย GPU ได้ด้วย (มี Plugin พิเศษชื่อว่า gpu2pyscf) ตัวโค้ดถูกเขียนและได้รับการปรับปรุงมาเป็นอย่างดี 
(Well-written) มีการวางโครงสร้างของโปรแกรมที่เรียบร้อย แบ่ง Methods ต่าง ๆ ออกเป็น Module ที่ชัดเจนและมีการจัดวาง Function 
ที่เหมาะสม สำหรับผู้อ่านที่สนใจโปรแกรม PySCF ก็ไปดูได้ที่ https://github.com/pyscf/pyscf

เมื่อเราเลือกโปรแกรมได้แล้ว ขั้นตอนต่อมาก็คือพยายามทำความเข้าใจทฤษฎีของหัวข้องานวิจัยที่เราต้องการศึกษา พยายามหาว่าเราสามารถพัฒนา%
วิธีนั้น ๆ ได้อย่างไรเพื่อที่จะปรับปรุงให้มีความถูกต้องในการคำนวณมากขึ้นหรือหากรณีที่ทฤษฎีนั้นยังไม่ครอบคลุม ขั้นตอนต่อไปคือหาวิธีการแก้ไข%
ปัญหาหรือ Solution สำหรับการปรับปรุงทฤษฎีนั้นแล้วเขียนออกมาเป็นสมการทางคณิตศาสตร์ที่เราจะนำไป Implement ได้ ขั้นตอนต่อไปก็คือ%
การวางแผนการเขียนโปรแกรมซึ่งสามารถทำได้ด้วยการเขียนโค้ดเทียมหรือ Pseudo Code ก่อนที่เราจะ Implement จริง ๆ โดยเราจะต้องคิดเกี่ยว%
กับการวางโครงสร้างหรือ Structure ของโปรแกรม เช่น แบ่งโปรแกรมออกเป็นโปรแกรมย่อย ๆ หลายส่วน เช่น แบ่งเป็น modules, functions, 
classes, หรือ types โดยเราควรจะต้องคำนึงถึงการพัฒนาโปรแกรมต่อไปในอนาคตด้วยว่าโปรแกรมของเรานั้นสามารถที่จะรองรับฟีเจอร์ใหม่ ๆ 
ที่นักพัฒนาคนอื่น ๆ จะเข้ามาช่วยพัฒนาเพิ่มเติมได้

หลังจากที่เรา Implement เข้าไปในโปรแกรมเสร็จเรียบร้อยแล้วเราควรจะต้องมีการตรวจสอบการทำงานของโปรแกรมหรือฟังก์ชันต่าง ๆ อย่างสม่ำเสมอ%
เพื่อตรวจสอบค่าที่ได้จากคำนวณว่ามีความถูกต้องและมีความสมเหตุสมผลมากน้อยแค่ไหน เมื่อได้ค่าการคำนวณที่ถูกต้องแล้วขั้นตอนสุดท้ายก็คือการ%
ปรับปรุงหรือทำความสะอาดโค้ดให้มีประสิทธิภาพและอ่านได้ง่ายขึ้น ในขั้นตอนนี้เราสามารถเรียนรู้ได้จากการศึกษาโค้ดที่นักพัฒนาคนอื่นเขียนไว้ก็ได้%
ว่าเขาเขียนอย่างไร ใช้วิธีหรือเทคนิคอะไรที่ทำให้โค้ดรันได้เร็วและมีประสิทธิภาพ นอกจากนี้ยังมีสิ่งอื่น ๆ ที่เราควรจะต้องทำด้วย เช่น เขียน Comment 
หรือทำเอกสารประกอบการใช้งาน (Documentation) เพื่อที่ว่าตัวเราเองหรือนักพัฒนาคนอื่น ๆ ที่มาอ่านหรือแก้ไขโค้ดของเรานั้นสามารถทำความ%
เข้าใจโค้ดได้ง่ายและไม่ต้องมานั่งศึกษาเองจากศูนย์

%----------------------------------------
\section{แบบฝึกหัด}
%----------------------------------------
