% LaTeX source for ``Algorithms for Computer Simulation of Molecular Systems''
% Copyright (c) 2023 รังสิมันต์ เกษแก้ว (Rangsiman Ketkaew).

% License: Creative Commons Attribution-NonCommercial-NoDerivatives 4.0 International (CC BY-NC-ND 4.0)
% https://creativecommons.org/licenses/by-nc-nd/4.0/

\chapter{Equation of State สำหรับของเหลว}
\label{ap:equation_of_state}

ในหัวข้อนี้เราจะมาศึกษา MD กันด้วย Virial Theorem กันก่อนซึ่งเป็นหลักการที่ถูกนำมาประยุกต์ใช้ในการคำนวณระบบที่มีอนุภาพ N ตัว (N-body) 

ของเหลว (Liquid) เป็นสถานะหนึ่งของสสารที่พลังงานศักย์นั้นมีดีกรีหรือระดับของพลังงานที่เท่ากับพลังงานจลน์ โดยเราสามารถใช้การคำนวณ MD 
มาคำนวณหา Equation of State ของของเหลวได้ เริ่มต้นสมมติว่าเรามีระบบที่มี $N$ โมเลกุลซึ่งมวลของแต่ละโมเลกุลนั้นมีค่าเท่ากับ $m$ 
และโมเลกุลทั้งหมดนั้นถูกจำลองอยู่ภายในกล่องทรงลูกบาศก์ที่มีความยาวของแต่ละด้านเท่ากับ $L$ เรากำหนดให้พิกัดเริ่มต้นของโมเลกุลนั้นคือ 
$\vec{r_i} = (x_i, y_i, z_i)\,$ และมีความเร็วคือ $\vec{v_i} = (u_i, v_i, w_i)$ และมีแรงคือ $\vec{F_i} = (f_i, g_i, 
h_i), i = 1, 2, ..., N$ 

จากกฎข้อที่สองของกฎการเคลื่อนที่ของนิวตันนั้น เรามีความสัมพันธ์คือ 

\begin{equation}
    \vec{F_i}=m\frac{d^2\vec{r_i}}{dt^2}
\end{equation}

\noindent ซึ่งเราสามารถเขียนได้เป็น

\begin{eqnarray}
    \frac d{dt}\left(\frac{d\left(x_i^2\right)}{dt}\right)  
    & = & \frac{d}{dt}
        \left( 2x_i\frac{dx_i}{dt}\right) \\
    & = & 2\frac{dx_i}{dt}\frac{dx_i}{dt} + 2x_i\frac{d^2x_i}{dt^2} \\
    & = & 2u_i^2 + 2x_i\frac{d^2x_i}{dt^2}
\end{eqnarray}

\noindent สมการด้านบนนั้นเป็นการบอกใบ้เราว่าแรงที่กระทำต่อระยะกระจัดที่โมเลกุลเคลื่อนที่ไปในแต่ละทิศทางนั้นสามารถเขียนได้เป็นอนุพันธ์ของ%
ฟังก์ชันที่ขึ้นกับความเร็วและความเร่งดังนี้

\begin{eqnarray}
    f_ix_i & = & mx_i \frac{d^2x_i}{dt^2} \\
    & = & \frac{1}{2} m \frac{d}{dt}\left(\frac{d\left( x_i^2\right)}{dt}\right) - mu_i^2
\end{eqnarray}

\begin{eqnarray}
    g_iy_i & = & my_i \frac{d^2y_i}{dt^2} \\
    & = & \frac{1}{2} m \frac d{dt}\left(\frac{d\left( y_i^2\right)}{dt}\right) - mv_i^2
\end{eqnarray}

\begin{eqnarray}
    h_iz_i & = & mz_i \frac{d^2z_i}{dt^2} \\
    & = &\frac{1}{2} m \frac{d}{dt}\left(\frac{d\left( z_i^2\right)}{dt}\right) -mw_i^2
\end{eqnarray}

\noindent เมื่อเรารวมสมการด้านบนทั้งสามสมการเข้าด้วยกัน เราจะได้ว่า

\begin{equation}
    \frac{1}{2} m \frac{d}{dt}\left\{ \frac d{dt}\left( x_i^2+y_i^2+z_i^2\right)
    \right\} = \vec{F_i} \cdot \vec{r_i}+m\left| \vec{v_i}\right|^2
\end{equation}

\noindent และเมื่อเราทำการรวมสมการสำหรับทุกโมเลกุลในระบบ $(i = 1, 2, \dots, N)$ เราจะได้ว่า

\begin{equation}
    \label{eq:eoq_system}
    \frac{1}{2} m \sum_{i=1}^{N} \frac d{dt}\left\{\frac d{dt}r_i^2\right\}
    =
    \sum_{i=1}^{N} \vec{F_i}\cdot \vec{r_i}+m\sum_{i=1}^{N} \left| \vec{v_i}\right|^2
\end{equation}

ต่อจากนั้นก็เราก็ทำการกำหนดให้กรอบอ้างอิงของระบบของเรานั้นอยู่ที่จุดกึ่งกลางของกล่องทรงลูกบาศน์ ซึ่งจะทำให้เรามีความสัมพันธ์ดังต่อไปนี้

\begin{equation}
    r_i^2\equiv \left| \vec{r_i}\right|^2 
    = 
    x_i^2+y_i^2+z_i^2
\end{equation} 

\noindent ซึ่งก็คือระยะห่างจากจุดกำเนิด (จุดศูนย์กลางของกล่อง) ถึงโมเลกุล $i$ ยกกำลังสอง โดยทางด้านซ้ายของสมการที่ 
\ref{eq:eoq_system} นั้นจริง ๆ แล้วก็คืออัตราของความเร่งของระยะห่างยกกำลังสองเฉลี่ย (Mean Square Distance) นั่นเอง ซึ่งปริมาณตัว%
นี้จะต้องมีค่าเท่ากับ 0 เสมอเพราะว่าของเหลวนั้นอยู่ในระบบปิด (ในที่นี้คือ Fixed Cube) ส่วนเทอมที่ 2 ของทางด้านขวาของสมการที่ 
\ref{eq:eoq_system} นั้นมีค่าเป็นสองเท่าของพลังงานจลน์รวมของระบบซึ่งหลายคนน่าจะเคยศึกษามาก่อนแล้วจากวิชาเคมีเชิงฟิสิกส์ในระดับ%
ปริญญาตรีว่าโมเลกุลแต่ละโมเลกุลนั้นจะมีค่าเฉลี่ยของพลังงานจลน์ของการเลื่อนตำแหน่ง (Translation) เท่ากับ $\frac 32k\theta$ 
โดยที่ $k$ คือค่าคงที่โบลท์ซมันน์ (Boltzmann Constant) และ $\theta$ คืออุณหภูมิสัมบูรณ์ ดังนั้นเทอมที่สองของด้านขวาของสมการนี้จะ%
ต้องมีค่าเฉลี่ยเท่ากับ $3Nk\theta$

สำหรับ $\sum_{i=1}^{N} \vec{F_i}\cdot \vec{r_i}$ ซึ่งเป็นเทอมแรกของทางด้านขวาของสมการที่ \ref{eq:eoq_system} นั้นมีชื่อ%
เรียกว่า \emph{Virial of Clausius} และมีมิติ (หน่วย) ที่เหมือนกันกับทอร์ก (Torque) โดย Virial นั้นสามารถหาได้จากค่าคาดหวัง 
(Expectation Value) ซึ่งเป็นค่าเฉลี่ย ดังนี้  

\begin{equation}
    \label{eq:virial_clausius}
    -PV + \left\langle \sum_{i\neq j}F(r_{ij})r_{ij}\right\rangle 
\end{equation}

คราวนี้เราจะมาทำการพิสูจน์สมการของ Virial ของสมการที่ \ref{eq:virial_clausius} โดยเราจะพิจารณาที่ละเทอม โดยเทอมแรกนั้นคือ%
แรงที่เกิดขึ้นจากขอบ (Boundary) ของระบบหรือด้านข้างของกล่องลูกบาศน์นั่นเอง ส่วนเทอมที่สองคือแรงที่กระทำต่อโมเลกุลซึ่งเป็นแรงระหว่าง%
โมเลกุล

\paragraph{1) แรงที่เกิดขึ้นที่ผนังของกล่อง} เริ่มต้นนั้นเราสามารถคำนวณหาแรงที่เกิดขึ้นบนผนัง ณ ตำแหน่ง $x=\frac L2$ ได้เท่ากับ 
$PL^2$ โดยที่ $P$ คือความดันที่กระทำต่อผนังและ $L^2$ คือพื้นที่ของผนัง ณ ตำแหน่ง $x=\frac L2$ แล้วเราก็ใช้กฎข้อที่ 3 ของกฎการ%
เคลื่อนที่ของนิวตันเราจะได้ว่าผลรวมของแรงทั้งหมดที่กระทำต่อโมเลกุลที่อยู่ใกล้ ๆ ผนังนั้นจะมีค่าเท่ากับ  $-PL^2$ ซึ่งเราจะได้ว่า Contribution 
ที่เกี่ยวข้องกับ Virial นั้นคือ

\begin{equation}
    \label{eq:virial_per_wall}
    -\frac 12PL^3
\end{equation}

\noindent เมื่อเราพิจารณาผนังทั้ง 6 ด้านของลูกบาศก์นั้นเราจะต้องทำการรวม Virial ของทั้ง 6 ด้านเข้าด้วยกัน พูดง่าย ๆ คือเราคูณสมการที่
\ref{eq:virial_per_wall} ด้วย 6 ซึ่งจะทำให้เราได้สมการ Virial ที่สมบูรณ์มากขึ้น ดังนี้

\begin{equation}
    -3PL^3 = -3PV
\end{equation}

\noindent โดยที่ $V=L^3$ คือปริมาตรของลูกบาศก์

\paragraph{2) แรงระหว่างโมเลกุล} เรากำหนดให้ $x$ เป็นองค์ประกอบเชิงเวกเตอร์ของแรงที่กระทำต่อโมเลกุล $i$ ที่เกิดขึ้นจากโมเลกุล $j$ 
โดยเขียนแทนด้วย $f(r_{ij})$ ซึ่งเราจะมีความสัมพันธ์ดังนี้

\begin{equation}
    f_i = \sum_{j=1 \\ j\neq i}^{N} f\left( r_{ij}\right) 
\end{equation}

\noindent ในทำนองเดียวกันเราก็จะมีแรงที่เหมือนกันกับกรณีข้างต้นด้วย ดังต่อไปนี้

\begin{equation}
    g_i = \sum_{j=1 \\ j\neq i}^{N} g\left( r_{ij}\right) 
\end{equation}

\begin{equation}
    h_i = \sum_{j=1 \\ j\neq i}^{N} h\left( r_{ij}\right) 
\end{equation}

ดังนั้นส่วนที่เป็นแรงที่เกิดขึ้นจากอันตรกิริยาระหว่างโมเลกุลของ Virial นั้นจึงสามารถอธิบายได้ด้วยสมการดังต่อไปนี้

\begin{equation}
    \sum_{i=1}^{N} \left[ \left\{ \sum_{j=1 \\ j\neq i}^{N} f(r_{ij})\right\}
    x_i+\left\{ \sum_{j=1 \\ j\neq i}^{N} g(r_{ij})\right\} y_i+\left\{
    \sum_{j=1 \\ j\neq i}^{N} h(r_{ij})\right\} z_i\right] 
\end{equation}

ลำดับไปต่อเราเราสามารถคำนวณหา Contribution ที่เกิดขึ้นกับสมการด้านบนได้โดยมีความสัมพันธ์เพิ่มเติมคือ

\begin{eqnarray}
    f\left(r_{ij}\right) x_i+f\left( r_{ji}\right) x_j & = &f\left( r_{ij}\right)
    x_i-f\left( r_{ij}\right) x_j \\
    & = &f\left( r_{ij}\right) \left( x_i-x_j\right) 
\end{eqnarray}

\begin{equation}
    g\left(r_{ij}\right) y_j+g\left( r_{ji}\right) y_j=g\left( r_{ij}\right)
    \left(y_i-y_j\right) 
\end{equation}

\begin{equation}
    h\left(r_{ij}\right) z_i+h\left( r_{ji}\right) z_j=h\left( r_{ij}\right)
    \left(z_i-z_j\right) 
\end{equation}

เราสามารถใช้สมการข้างต้นนี้ได้เพราะว่าแรงที่โมเลกุล $j$ กระทำต่อโมเลกุล $i$ นั้นมีขนาดที่เท่ากับแรงที่โมเลกุล $i$ กระทำต่อโมเลกุล $j$ 
แต่ว่ามีขนาดตรงข้ามกัน

ถ้าหากเรากำหนดให้ $\left(f\left( r_{ij}\right) ,g\left( r_{ij}\right) ,h\left(r_{ij}\right) \right) $ มีทิศทางที่%
อยู่ในแนวเดียวกันกับเส้นตรงระหว่าง $\vec{r_i}$ ถึง $\vec{r_j}$ เช่น

\begin{equation}
    \vec{F}(r_{ij}) \cdot (\vec{r_j} - \vec{r_i})=F\left( r_{ij}\right) r_{ij}
\end{equation}

\noindent โดยที่ $F\left( r_{ij}\right) =\left| \vec{F}(r_{ij})\right| $ และ 
$r_{ij}=\left| \vec{r_j} - \vec{r_i}\right|$ เราจะได้ว่า Virial นั้นจริง ๆ แล้วมีค่าเท่ากับสมการดังต่อไปนี้ 

\begin{equation}
    -3PV+\left\langle \sum_{i\neq j}F\left( r_{ij}\right) r_{ij}\right\rangle 
\end{equation}

\noindent ซึ่งเราสามารถเขียนใหม่ได้เป็น

\begin{equation}
    3PV = 3NK\theta +\left\langle \sum_{i\neq j}F\left( r_{ij}\right)
    r_{ij}\right\rangle 
\end{equation}
