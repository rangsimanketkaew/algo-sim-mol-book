% LaTeX source for ``Algorithms for Computer Simulation of Molecular Systems''
% Copyright (c) 2023 รังสิมันต์ เกษแก้ว (Rangsiman Ketkaew).

% License: Creative Commons Attribution-NonCommercial-NoDerivatives 4.0 International (CC BY-NC-ND 4.0)
% https://creativecommons.org/licenses/by-nc-nd/4.0/

\chapter{เทคนิคทางโครงสร้างเชิงอิเล็กทรอนิกส์}
\label{ap:elec_struct_technique}

%----------------------------------------
\section{Static Correlation กับ Dynamic Correlation}
%----------------------------------------

การที่เราจะเรียนกลศาสตร์ควอนตัมให้เข้าใจและไปคุยกับคนอื่นรู้เรื่องได้นั้น เราจะต้องรู้ความหมายหรือนิยามของคำศัพท์ทางเทคนิค (Technical Terms) 
กันก่อน สมมติว่าคนสองคนกำลังพูดถึงสิ่งเดียวกันแต่ตีความกันคนละความหมายก็จบข่าวใช่ไหมครับ ดังนั้นการเข้าใจ Terminology ในทางกลศาสตร์ควอนตัม, 
เคมีเชิงฟิสิกส์โดยเฉพาะ Electronic Structure นั้นจึงสำคัญมาก ๆ

ทำไมคำสองคำนี้จึงสำคัญ? จริง ๆ ทฤษฎีพิเศษทางเคมีควอนตัมนั้นมักจะเกี่ยวข้องกับ Correlation ของอิเล็กตรอน เช่น Density Matrix Functional 
Theory (DMFT) ซึ่งผมคิดว่าเป็นทฤษฎีที่กำลังจะเข้ามาเปลี่ยนวงการเคมีควอนตัมเลยเพราะมันแก้ปัญหาหลาย ๆ อย่างของ Density Functional Theory 
(DFT) ได้ ดังนั้นในการทำความเข้าใจทฤษฎีเหล่านั้น เราก็ควรที่จะต้องเข้าใจความหมายของคำว่า Correlation กันก่อน 

ผมขอเริ่มที่คำว่า Correlation ก่อน ถ้าแปลเป็นภาษาไทยเราจะเรียกว่า \enquote{สหสัมพันธ์} ซึ่ง สห คือ \enquote{พร้อม ๆ กัน} ส่วน 
\enquote{สัมพันธ์} ก็คือ \enquote{ความสัมพันธ์เกี่ยวเนื่องกัน} แล้วอะไรล่ะที่มันเกี่ยวเนื่องเชื่อมโยงกัน? คำตอบก็คือ \textit{อิเล็กตรอน} 
เพราะในทางกลศาสตร์ควอนตัมที่เน้นทางด้านเคมีนั้นเราติดปัญหาอยู่อย่างเดียวคือการที่จะอธิบายระบบที่มีอิเล็กตรอนหลายตัวนั้นมันทำได้ยาก (มี Proof 
ออกมาแล้วว่าทำไมไม่ได้เลยในกรณีที่เรายังใช้ทฤษฎี Schordinger อยู่) ซึ่งนิยามทางคณิคศาสตร์ของ Correlation ก็คือความน่าจะเป็น (Probability) 
ของการที่เราจะเจออิเล็กตรอนตัวที่ 1 ที่ตำแหน่ง a กับอิเล็กตรอนตัวที่ 2 ที่ตำแหน่ง b ซึ่งเราควรที่จะสามารถคำนวณหา Probability 
ของสถานการณ์นี้ได้อย่างง่าย ๆ โดยการนำ probability ของการพบอิเล็กตรอนทั้งสองตัวนี้มาคูณกัน แต่ว่าจริง ๆ แล้วมันไม่ได้ง่ายขนาดนั้น 

แนวคิดหรือ Concept ของ Correlation ก็คืออันตรกิริยา (Interaction) ระหว่างอิเล็กตรอน ซึ่งต้องเป็นแรงผลักแบบ instantaneous ด้วยนะ 
โดยเราเรียกแรงผลักแบบที่เกิดขึ้น ณ ขณะใดขณะหนึ่งแบบทันทีใด นี้ว่า "Dynamic Correlation" นั่นเอง ซึ่งการนิยามคำนี้นั้นเริ่มต้นมาจากการศึกษา%
การทำลายหรือแตกออกของพันธะ (Bond Dissociation) ในโมเลกุล ถ้าเราค่อย ๆ ดึงอะตอม 2 อะตอมที่มันมีพันธะเคมีกันอยู่ให้ห่างออกจากกัน 
เราจะพบว่าอิเล็กตรอนก็จะอยู่ห่างกันมากขึ้น ทำให้แรงผลักลดลง จึงทำให้พลังงานสหสัมพันธ์หรือ Correlation Energy ลดลงตามไปด้วย แล้วที่เราใช้คำว่า 
Dynamic เพราะมันคือผลที่เกิดจากการเคลื่อนที่ของอิเล็กตรอน (Electron Motion) นั่นเอง

แต่ว่าเรากลับพบว่ามันมีหลาย ๆ กรณีที่มันตรงข้ามกับสิ่งที่ผมเพิ่งอธิบายไปเมื่อกี้นี้ ซึ่งเราพบว่าในกรณีแปลก ๆ พวกนั้นค่าพลังงาน Correlation Energy 
มันกลับเพิ่มขึ้น คำถามคือ เป็นไปได้ไง? สมมติฐานก็คือ แสดงว่ามันต้องมี Correlation แบบอื่นที่นอกเหนือจาก Dynamic Correlation หลบซ่อนอยู่แน่ ๆ 
ซึ่งเราเรียก Correlation แบบนั้นว่า \enquote{Static Correlation} นั่นเอง

สุดท้ายแล้วนักเคมีทฤษฎีก็ค้นพบว่าสาเหตุที่มันเป็นแบบนี้เพราะว่ามันมีสิ่งที่เรียกว่า (Near-)degenerate Configuration เพิ่มขึ้นมาซึ่งมันส่งผลหรือ 
Contribute ต่อพฤติกรรมของฟังก์ชันคลื่นในระหว่างที่พันธะเคมีแตกออกแบบเยอะมาก ๆ เราเลยเรียกระบบพวกนี้ว่า \enquote{(Strongly) Statically 
Correlated System} นี่จึงเป็นสาเหตุที่ทำให้วิธีการคำนวณ เช่น Hartree-Fock ที่ใช้ Single Slater Determinant นั้นใช้งานไม่ได้หรือ 
Fail นั่นเอง

คำว่า Near-degenerate State แปลตรงตัวเลยคือระดับพลังงานของออร์บิทัลที่อิเล็กตรอนมันอยู่หรือถูกกระตุ้นไปให้ไปอยู่นั้นมันใกล้กันมาก ๆ 
ซึ่งเราจะพบเหตุการณ์แบบนี้ได้ก็เช่นกรณีที่เราสนใจการกระตุ้นอิเล็กตรอนหลาย ๆ ตัว (หลาย ๆ Configuration) ซึ่งมันก็จะมีเทคนิคที่แตกต่างกันไป%
ในการจัดการ (Treat) คอนฟิกุเรชั่น (Configuration) ของ Excited Electrons พวกนี้ เช่นอาจจะ Treat พร้อมกันหมดทุกกันด้วยวิธี CASSCF 
หรือทำการตัดหรือแยกกัน treat ด้วยวิธี CCS, CCSD เป็นต้น ซึ่งผมไม่ได้ลงรายละเอียดในหนังสือเล่มนี้

สรุปสั้น ๆ อีกครั้งคือ Static Correlation นั้นมาจากการอธิบายสถานการณ์ที่การที่ฟังก์ชันคลื่นของ Hartree-Fock (HF) ที่เราใช้เป็น Reference 
Wavefunction นั้นไม่ Fail หรือล้มเหลวในการคำนวณสิ่งต่าง ๆ นั่นก็เพราะว่าโมเดล HF นั้นมันใช้แนวคิดที่ว่าอิเล็กตรอนนั้นมี Instantaneous 
Interaction กับสนามเฉลี่ย (Mean Field) หรือค่าเฉลี่ยของอิเล็กตรอนทั้งหมด แทนที่จะเป็น Instantaneous Interaction 
ระหว่างอิเล็กตรอนตัวอื่น ๆ แต่ละตัว ซึ่งในความเป็นจริงนั้นมันควรจะต้องเป็นแบบหลัง

ดังนั้น Dynamic Correlation จึงถูกนำมาใช้ในการอธิบายระบบต่าง ๆ แทนเพราะว่ามันทำให้ Hartree-Fock Reference นั้นถูกต้องมากขึ้น 
แต่ต้องใส่ดอกจันทร์ตัวหนา ๆ เลยว่าให้ผลการคำนวณถูกต้องแบบ Qualitative เท่านั้น (ให้ผลการคำนวณในภาพรวมแบบที่มีแนวโน้มถูกต้อง) 
แต่ไม่ถูกต้องแบบ Quantitative (ให้ผลการคำนวณที่ผิดคำนวณผิดหรือคลาดเคลื่อน)

ถ้าให้เข้าใจง่ายกว่านี้อีกก็คือ \enquote{Correlation} นั้นมันสื่อถึงความห่วยหรือไร้ประสิทธิภาพ (Deficiency) ของวิธี Hartree-Fock ที่ใช้ 
Single Slater Determinant นั่นเอง โดยปกติแล้วเราสามารถคำนวณหาพลังงาน Correlation Energy ได้ดังนี้

\begin{equation}
    E_{corr} = E_{exact} - E_{HF}
\end{equation}

\noindent ก็คือการนำค่าพลังงานจริงมาลบออกด้วยค่าพลังงานที่ได้จากวิธี HF จะได้ Correlation Energy ($E_{corr}$) นั่นหมายความว่า 
Correlation Energy นั้นคือส่วนที่หายไปที่ HF นั้นต้องการเข้ามาเติมเต็ม ซึ่งมันก็มีวิธีต่าง ๆ มากมายที่เราเรียกกันว่า Post-HF นั้นเข้ามาช่วยในการ 
Correction โดยการรวม Configuration แบบต่าง ๆ ของ Excited States เข้าไปนั่นเอง วิธี Post-HF ก็มีหลายอัน เช่น $n$th-order 
Møller-Plesset Perturbation Theory (MPn), Multi-configurational Self-consistent Field (MCSCF), Configuration 
Interaction (CI), Full CI

แต่เราต้องเข้าใจให้ถูกต้องอีกนะว่าไม่ใช่วิธี Post-HF ทุกวิธีที่สามารถแก้ปัญหา Correlation โดยการใส่เทอม Dynanic Correlation เข้าไปอย่างเดียว 
ตัวอย่างเช่น วิธี MPn Perturbation นั้นใช้ Dynamic Correlation ในขณะที่วิธีอย่าง MCSCF นั้นใช้ Static Correlation

แล้วคำถามคือทำไมวิธี Post-HF ต่าง ๆ ถึงไม่รวมทั้ง Static Correlation และ Dynamic Correlation เข้าไปพร้อม ๆ กัน 
คำอธิบายคือ จริง ๆ แล้วมันเป็นไปไม่ได้เลยที่เราจะแยก Static Correlation กับ Dynamic Correlation ออกจากกันนั่นก็เพราะว่า Correlation 
ทั้งสองอันนี้มีพื้นฐานมาจาก Physical Interaction ที่เหมือนกัน ดังนั้นวิธีการที่ Cover หรือรวม Dynamic Correlation เข้าไปแล้วนั้นก็จะรวม 
Effect ของ Correlation แบบที่เป็น Non-dynamic Effect ซึ่งก็คือ Static Correlation เข้าไปด้วย และในทำนองเดียวกันกับวิธีที่รวมเฉพาะ 
Static Correlation เข้าไป ก็จะรวม Dynamic Correlation เข้าไปด้วยโดยปริยายแล้วนั่นเอง ซึ่ง Correlation ทั้งสองอันนี้มันถูกผสมหรือ 
Mixed กันอยู่ในเทอมสูง ๆ ของ Wavefunction Configuration

\noindent หมายเหตุ 1: ตามที่เราศึกษากันมาว่า Hartree-Fock นั้นไม่มี Correlation ผสมอยู่เลย จริง ๆ แล้วก็ไม่ถูกซะทีเดียว เพราะว่า HF 
นั้นไม่ยอมให้มีอิเล็กตรอน 2 ตัวใด ๆ มี State เหมือนกันได้ ดังนั้น HF จึงมีความเป็น Correlation อยู่นิดหน่อยนั่นเอง (เรียกว่า Fermi Correlation)

\noindent หมายเหตุ 2: Single Slater Determinant นั้นเป็น Representation ของ Wavefunction ที่ไม่ค่อยดีเท่าไหร่ 
ไม่เหมาะนำมาใช้อธิบายระบบ Many-electron หรือระบบที่มีอิเล็กตรอนหลายตัว

%----------------------------------------
\section{Density Matrix Renormalization Group}
%----------------------------------------

สรุปการพัฒนาทฤษฎี Density Matrix Renormalization Group (DMRG)

DMRG เป็นหนึ่งในทฤษฎีที่ถูกพัฒนามาจาก Quantum Renormalization Group Theory โดยเป็นการใช้ Density Matrix Formulation 
ที่เสนอโดย Steven White ศาสตราจารย์ทางด้านฟิสิกส์ที่ University of California, Irvine ในช่วงปี 1992 แล้วก็ถูกนำมาประยุกต์ใช้กับ%
งานวิจัยทาง Quantum Chemistry ตั้งแต่นั้นเป็นต้นมา

วิธี DMRG นั้นเป็น Vairational-based Method ซึ่งนำมาใช้ในการคำนวณ Wavefunction ซึ่งถูกเขียนหรือถูก Represented ด้วยสิ่งที่เรียกว่า 
Matrix Product State (MPS) หรืออีกชื่อคือ Tensor Chain หรือ Tensor Network 

นักวิจัยได้นำทฤษฎีนี้ไปใช้ศึกษาระบบพิเศษบางประเภทที่ซับซ้อน เช่น Strongly Correlated System ซึ่งก็คือระบบที่อิเล็กตรอนนั้นมี Correlation 
ต่อกันสูงมาก ๆ โดยให้นึกถึงโมเลกุลหรือวัสดุจำพวก Conductor-Insulator Material หรือ Transition Metal Oxide Complex เป็นต้น 
ที่ทฤษฎีทั่วไปไม่สามารถตอบโจทย์

แม้ว่า DMRG จะถูกพัฒนามานานกว่า 30 ปีแล้ว แต่ก็ยังไม่ได้เป็นที่แพร่หลายมากนักในกลุ่มนักเคมีเชิงคำนวณ ยิ่งสายแอพพลิเคชั่นก็ไม่ต้องพูดถึงเลย 
เพราะว่าตัวทฤษฎีนั้นเรียกได้ว่ายังอยู่ในขั้นของการพัฒนาเพื่อให้สามารถนำไปใช้งานได้กับระบบทางเคมีจริง ๆ ได้อยู่

ผมคิดผู้อ่านหลาย ๆ คนอาจจะยังไม่เคยได้ยินแม้แต่ชื่อทฤษฎีอันนี้มาก่อน เท่าที่ผมทราบ (อย่างน้อยก็ ณ วันที่ผมเขียนหนังสือเล่มนี้ซึ่งก็คือเดือนกันยายน 
พ.ศ. 2566) ในประเทศไทยก็ยังไม่มีกลุ่มวิจัยไหนที่นำทฤษฎีนี้มาใช้เลย แต่ก็ไม่ใช่เรื่องแปลกอะไรเพราะแม้แต่ในต่างประเทศก็มีกลุ่มวิจัยแค่ไม่กี่ที่ที่ใน%
โลกเท่านั้นที่ทำวิจัยโดยใช้วิธีนี้นั่นก็เพราะว่าตัวทฤษฎีนั้นมีความยาก ซับซ้อน และสิ้นเปลืองในเชิงการคำนวณพอสมควร

ถ้าสนใจอ่านเปเปอร์เฉพาะทางที่เกี่ยวข้องกับการพัฒนา DMRG สำหรับโจทย์งานวิจัย Electronic structure ลองอ่านเปเปอร์ของกลุ่มวิจัย 
Garnet Chan (Caltech) ซึ่งเป็นผู้พัฒนา library สำหรับ DMRG calculation (กลุ่มวิจัยเดียวกันกับที่พัฒนาโปรแกรม PySCF)
