% LaTeX source for ``Algorithms for Computer Simulation of Molecular Systems''
% Copyright (c) 2023 รังสิมันต์ เกษแก้ว (Rangsiman Ketkaew).

% License: Creative Commons Attribution-NonCommercial-NoDerivatives 4.0 International (CC BY-NC-ND 4.0)
% https://creativecommons.org/licenses/by-nc-nd/4.0/

\chapter{เทคนิคทางโครงสร้างเชิงอิเล็กทรอนิกส์}
\label{ap:elec_struct_technique}

%----------------------------------------
\section{Static Correlation กับ Dynamic Correlation}
\idxen{Static Correlation}
\idxen{Dynamic Correlation}
%----------------------------------------

การที่เราจะเรียนกลศาสตร์ควอนตัมให้เข้าใจและไปคุยกับคนอื่นรู้เรื่องได้นั้น เราจะต้องรู้ความหมายหรือนิยามของคำศัพท์ทางเทคนิค (Technical Terms)
กันก่อน สมมติว่าคนสองคนกำลังพูดถึงสิ่งเดียวกันแต่ตีความกันคนละความหมายก็จบข่าวใช่ไหมครับ ดังนั้นการเข้าใจ Terminology ในทางกลศาสตร์ควอนตัม,
เคมีเชิงฟิสิกส์โดยเฉพาะ Electronic Structure นั้นจึงสำคัญมาก ๆ

ทำไมคำสองคำนี้จึงสำคัญ? จริง ๆ ทฤษฎีพิเศษทางเคมีควอนตัมนั้นมักจะเกี่ยวข้องกับ Correlation ของอิเล็กตรอน เช่น Density Matrix Functional
Theory (DMFT) ซึ่งผมคิดว่าเป็นทฤษฎีที่กำลังจะเข้ามาเปลี่ยนวงการเคมีควอนตัมเลยเพราะมันแก้ปัญหาหลาย ๆ อย่างของ Density Functional Theory
(DFT) ได้ ดังนั้นในการทำความเข้าใจทฤษฎีเหล่านั้น เราก็ควรที่จะต้องเข้าใจความหมายของคำว่า Correlation กันก่อน

ผมขอเริ่มที่คำว่า Correlation ก่อน ถ้าแปลเป็นภาษาไทยเราจะเรียกว่า \enquote{สหสัมพันธ์} ซึ่ง สห คือ \enquote{พร้อม ๆ กัน} ส่วน
\enquote{สัมพันธ์} ก็คือ \enquote{ความสัมพันธ์เกี่ยวเนื่องกัน} แล้วอะไรล่ะที่มันเกี่ยวเนื่องเชื่อมโยงกัน? คำตอบก็คือ \textit{อิเล็กตรอน}
เพราะในทางกลศาสตร์ควอนตัมที่เน้นทางด้านเคมีนั้นเราติดปัญหาอยู่อย่างเดียวคือการที่จะอธิบายระบบที่มีอิเล็กตรอนหลายตัวนั้นมันทำได้ยาก (มี Proof
ออกมาแล้วว่าทำไมไม่ได้เลยในกรณีที่เรายังใช้ทฤษฎี Schordinger อยู่) ซึ่งนิยามทางคณิคศาสตร์ของ Correlation ก็คือความน่าจะเป็น (Probability)
ของการที่เราจะเจออิเล็กตรอนตัวที่ 1 ที่ตำแหน่ง a กับอิเล็กตรอนตัวที่ 2 ที่ตำแหน่ง b ซึ่งเราควรที่จะสามารถคำนวณหา Probability
ของสถานการณ์นี้ได้อย่างง่าย ๆ โดยการนำ probability ของการพบอิเล็กตรอนทั้งสองตัวนี้มาคูณกัน แต่ว่าจริง ๆ แล้วมันไม่ได้ง่ายขนาดนั้น

แนวคิดหรือ Concept ของ Correlation ก็คืออันตรกิริยา (Interaction) ระหว่างอิเล็กตรอน ซึ่งต้องเป็นแรงผลักแบบ instantaneous ด้วยนะ
โดยเราเรียกแรงผลักแบบที่เกิดขึ้น ณ ขณะใดขณะหนึ่งแบบทันทีใด นี้ว่า "Dynamic Correlation" นั่นเอง ซึ่งการนิยามคำนี้นั้นเริ่มต้นมาจากการศึกษา%
การทำลายหรือแตกออกของพันธะ (Bond Dissociation) ในโมเลกุล ถ้าเราค่อย ๆ ดึงอะตอม 2 อะตอมที่มันมีพันธะเคมีกันอยู่ให้ห่างออกจากกัน
เราจะพบว่าอิเล็กตรอนก็จะอยู่ห่างกันมากขึ้น ทำให้แรงผลักลดลง จึงทำให้พลังงานสหสัมพันธ์หรือ Correlation Energy ลดลงตามไปด้วย แล้วที่เราใช้คำว่า
Dynamic เพราะมันคือผลที่เกิดจากการเคลื่อนที่ของอิเล็กตรอน (Electron Motion) นั่นเอง

แต่ว่าเรากลับพบว่ามันมีหลาย ๆ กรณีที่มันตรงข้ามกับสิ่งที่ผมเพิ่งอธิบายไปเมื่อกี้นี้ ซึ่งเราพบว่าในกรณีแปลก ๆ พวกนั้นค่าพลังงาน Correlation Energy
มันกลับเพิ่มขึ้น คำถามคือ เป็นไปได้ไง? สมมติฐานก็คือ แสดงว่ามันต้องมี Correlation แบบอื่นที่นอกเหนือจาก Dynamic Correlation หลบซ่อนอยู่แน่ ๆ
ซึ่งเราเรียก Correlation แบบนั้นว่า \enquote{Static Correlation} นั่นเอง

สุดท้ายแล้วนักเคมีทฤษฎีก็ค้นพบว่าสาเหตุที่มันเป็นแบบนี้เพราะว่ามันมีสิ่งที่เรียกว่า (Near-)degenerate Configuration เพิ่มขึ้นมาซึ่งมันส่งผลหรือ
Contribute ต่อพฤติกรรมของฟังก์ชันคลื่นในระหว่างที่พันธะเคมีแตกออกแบบเยอะมาก ๆ เราเลยเรียกระบบพวกนี้ว่า \enquote{(Strongly) Statically
    Correlated System} นี่จึงเป็นสาเหตุที่ทำให้วิธีการคำนวณ เช่น Hartree-Fock ที่ใช้ Single Slater Determinant นั้นใช้งานไม่ได้หรือ
Fail นั่นเอง

คำว่า Near-degenerate State แปลตรงตัวเลยคือระดับพลังงานของออร์บิทัลที่อิเล็กตรอนมันอยู่หรือถูกกระตุ้นไปให้ไปอยู่นั้นมันใกล้กันมาก ๆ
ซึ่งเราจะพบเหตุการณ์แบบนี้ได้ก็เช่นกรณีที่เราสนใจการกระตุ้นอิเล็กตรอนหลาย ๆ ตัว (หลาย ๆ Configuration) ซึ่งมันก็จะมีเทคนิคที่แตกต่างกันไป%
ในการจัดการ (Treat) คอนฟิกุเรชั่น (Configuration) ของ Excited Electrons พวกนี้ เช่นอาจจะ Treat พร้อมกันหมดทุกกันด้วยวิธี CASSCF
หรือทำการตัดหรือแยกกัน treat ด้วยวิธี CCS, CCSD เป็นต้น ซึ่งผมไม่ได้ลงรายละเอียดในหนังสือเล่มนี้

สรุปสั้น ๆ อีกครั้งคือ Static Correlation นั้นมาจากการอธิบายสถานการณ์ที่การที่ฟังก์ชันคลื่นของ Hartree-Fock (HF) ที่เราใช้เป็น Reference
Wavefunction นั้นไม่ Fail หรือล้มเหลวในการคำนวณสิ่งต่าง ๆ นั่นก็เพราะว่าโมเดล HF นั้นมันใช้แนวคิดที่ว่าอิเล็กตรอนนั้นมี Instantaneous
Interaction กับสนามเฉลี่ย (Mean Field) หรือค่าเฉลี่ยของอิเล็กตรอนทั้งหมด แทนที่จะเป็น Instantaneous Interaction
ระหว่างอิเล็กตรอนตัวอื่น ๆ แต่ละตัว ซึ่งในความเป็นจริงนั้นมันควรจะต้องเป็นแบบหลัง

ดังนั้น Dynamic Correlation จึงถูกนำมาใช้ในการอธิบายระบบต่าง ๆ แทนเพราะว่ามันทำให้ Hartree-Fock Reference นั้นถูกต้องมากขึ้น
แต่ต้องใส่ดอกจันทร์ตัวหนา ๆ เลยว่าให้ผลการคำนวณถูกต้องแบบ Qualitative เท่านั้น (ให้ผลการคำนวณในภาพรวมแบบที่มีแนวโน้มถูกต้อง)
แต่ไม่ถูกต้องแบบ Quantitative (ให้ผลการคำนวณที่ผิดคำนวณผิดหรือคลาดเคลื่อน)

ถ้าให้เข้าใจง่ายกว่านี้อีกก็คือ \enquote{Correlation} นั้นมันสื่อถึงความห่วยหรือไร้ประสิทธิภาพ (Deficiency) ของวิธี Hartree-Fock ที่ใช้
Single Slater Determinant นั่นเอง โดยปกติแล้วเราสามารถคำนวณหาพลังงาน Correlation Energy ได้ดังนี้

\begin{equation}
    E_{corr} = E_{exact} - E_{HF}
\end{equation}

\noindent ก็คือการนำค่าพลังงานจริงมาลบออกด้วยค่าพลังงานที่ได้จากวิธี HF จะได้ Correlation Energy ($E_{corr}$) นั่นหมายความว่า
Correlation Energy นั้นคือส่วนที่หายไปที่ HF นั้นต้องการเข้ามาเติมเต็ม ซึ่งมันก็มีวิธีต่าง ๆ มากมายที่เราเรียกกันว่า Post-HF นั้นเข้ามาช่วยในการ
Correction โดยการรวม Configuration แบบต่าง ๆ ของ Excited States เข้าไปนั่นเอง วิธี Post-HF ก็มีหลายอัน เช่น $n$th-order
Møller-Plesset Perturbation Theory (MPn), Multi-configurational Self-consistent Field (MCSCF), Configuration
Interaction (CI), Full CI

แต่เราต้องเข้าใจให้ถูกต้องอีกนะว่าไม่ใช่วิธี Post-HF ทุกวิธีที่สามารถแก้ปัญหา Correlation โดยการใส่เทอม Dynanic Correlation เข้าไปอย่างเดียว
ตัวอย่างเช่น วิธี MPn Perturbation นั้นใช้ Dynamic Correlation ในขณะที่วิธีอย่าง MCSCF นั้นใช้ Static Correlation

แล้วคำถามคือทำไมวิธี Post-HF ต่าง ๆ ถึงไม่รวมทั้ง Static Correlation และ Dynamic Correlation เข้าไปพร้อม ๆ กัน
คำอธิบายคือ จริง ๆ แล้วมันเป็นไปไม่ได้เลยที่เราจะแยก Static Correlation กับ Dynamic Correlation ออกจากกันนั่นก็เพราะว่า Correlation
ทั้งสองอันนี้มีพื้นฐานมาจาก Physical Interaction ที่เหมือนกัน ดังนั้นวิธีการที่ Cover หรือรวม Dynamic Correlation เข้าไปแล้วนั้นก็จะรวม
Effect ของ Correlation แบบที่เป็น Non-dynamic Effect ซึ่งก็คือ Static Correlation เข้าไปด้วย และในทำนองเดียวกันกับวิธีที่รวมเฉพาะ
Static Correlation เข้าไป ก็จะรวม Dynamic Correlation เข้าไปด้วยโดยปริยายแล้วนั่นเอง ซึ่ง Correlation ทั้งสองอันนี้มันถูกผสมหรือ
Mixed กันอยู่ในเทอมสูง ๆ ของ Wavefunction Configuration

\noindent หมายเหตุ 1: ตามที่เราศึกษากันมาว่า Hartree-Fock นั้นไม่มี Correlation ผสมอยู่เลย จริง ๆ แล้วก็ไม่ถูกซะทีเดียว เพราะว่า HF
นั้นไม่ยอมให้มีอิเล็กตรอน 2 ตัวใด ๆ มี State เหมือนกันได้ ดังนั้น HF จึงมีความเป็น Correlation อยู่นิดหน่อยนั่นเอง (เรียกว่า Fermi Correlation)

\noindent หมายเหตุ 2: Single Slater Determinant นั้นเป็น Representation ของฟังก์ชันคลื่นที่ไม่ค่อยดีเท่าไหร่
ไม่เหมาะนำมาใช้อธิบายระบบ Many-electron หรือระบบที่มีอิเล็กตรอนหลายตัว

%----------------------------------------
\section{Density Matrix Renormalization Group}
\idxen{Density Matrix Renormalization Group}
%----------------------------------------

ในหัวข้อนี้ผมอยากจะให้ผู้อ่านได้รู้จักกับวิธีควอนตัมอีกวิธีหนึ่งที่ตอนนี้ได้รับความสนใจในหมู่นักเคมีทฤษฎีเป็นอย่างมาก นั่นก็คือ Density Matrix
Renormalization Group (DMRG)

DMRG เป็นหนึ่งในทฤษฎีที่ถูกพัฒนามาจาก Quantum Renormalization Group Theory โดยเป็นการใช้ Density Matrix Formulation
ที่เสนอโดย Steven White ศาสตราจารย์ทางด้านฟิสิกส์ที่ University of California, Irvine ในช่วงปี 1992 แล้วก็ถูกนำมาประยุกต์ใช้กับ%
งานวิจัยทาง Quantum Chemistry ตั้งแต่นั้นเป็นต้นมา

วิธี DMRG นั้นเป็น Vairational-based Method ซึ่งนำมาใช้ในการคำนวณ Wavefunction ซึ่งถูกเขียนหรือถูก Represented ด้วยสิ่งที่เรียกว่า
Matrix Product State (MPS) หรืออีกชื่อคือ Tensor Chain หรือ Tensor Network นักวิจัยได้นำทฤษฎี DMRG ไปใช้ศึกษาระบบโมเลกุลแบบพิเศษ
(Special Case) บางประเภทที่มีความซับซ้อนและไม่สามารถที่จะใช้วิธีควอนตัมทั่วไปในการอธิบายหรือคำนวณได้ เช่น Strongly Correlated
System ซึ่งก็คือระบบที่อิเล็กตรอนนั้นมี Correlation ต่อกันสูงมาก ๆ โดยให้นึกถึงโมเลกุลหรือวัสดุจำพวก Conductor-Insulator Material
หรือสารประกอบ Transition Metal Oxide เป็นต้น

แม้ว่า DMRG จะถูกพัฒนามานานกว่า 30 ปีแล้ว แต่ก็ยังไม่ได้เป็นที่แพร่หลายมากนักในกลุ่มนักเคมีเชิงคำนวณ ยิ่งถ้าเป็นการประยุกต์ใช้นั้นก็ไม่ต้องพูดถึงเลย
เพราะว่าตัวทฤษฎีนั้นเรียกได้ว่ายังอยู่ในขั้นของการพัฒนาเพื่อให้สามารถนำไปใช้งานได้กับระบบทางเคมีจริง ๆ ได้อยู่ 

ผมคิดผู้อ่านหลาย ๆ คนอาจจะยังไม่เคยได้ยินแม้แต่ชื่อทฤษฎีอันนี้มาก่อน เท่าที่ผมทราบ (อย่างน้อยก็ ณ วันที่ผมเขียนหนังสือเล่มนี้ซึ่งก็คือเดือนกันยายน
พ.ศ. 2566) ในประเทศไทยก็ยังไม่มีกลุ่มวิจัยไหนที่นำทฤษฎีนี้มาใช้เลย แต่ก็ไม่ใช่เรื่องแปลกอะไรเพราะแม้แต่ในต่างประเทศก็มีกลุ่มวิจัยแค่ไม่กี่ที่ที่ใน%
โลกเท่านั้นที่ทำวิจัยโดยใช้วิธีนี้นั่นก็เพราะว่าตัวทฤษฎีนั้นมีความยาก ซับซ้อน และสิ้นเปลืองในเชิงการคำนวณพอสมควร

ถ้าสนใจอ่านเปเปอร์เฉพาะทางที่เกี่ยวข้องกับการพัฒนา DMRG สำหรับโจทย์งานวิจัย Electronic Structure ลองอ่านเปเปอร์ของกลุ่มวิจัยของ
Professor Garnet Kin-Lic Chan แห่ง California Institute of Technology หรือ Caltech ซึ่งเป็นกลุ่มวิจัยที่พัฒนา Library 
สำหรับการคำนวณ DMRG (กลุ่มวิจัยเดียวกันกับที่พัฒนาโปรแกรม PySCF) และก็มีกลุ่มวิจัยของ Professor Markus Reiher แห่ง ETH Zürich 
ที่พัฒนาทฤษฎี DMRG เพื่อใช้ในการศึกษาและแก้ปัญหาโจทย์ทางเคมีเช่นเดียวกัน

%----------------------------------------
\section{Density Matrix Functional Theory}
\idxen{Density Matrix Functional Theory}
%----------------------------------------

Density Matrix Functional Theory (DMFT) เป็นทฤษฎีที่นักเคมีเชิงทฤษฎีเชื่อว่าจะเข้ามาพลิกโฉมเปลี่ยนแปลงวงการเคมีควอนตัม
โดย DMFT สามารถแก้ปัญหาหลาย ๆ อย่างของ Density Functional Theory (DFT) ได้ โดยในหัวนี้ผู้อ่านจะได้ศึกษา DMFT แบบเบื้องต้นครับ

ขอเท้าความก่อนว่าตัวทฤษฎี DFT นั้นมีปัญหาหลายอย่าง ทำให้ต้องพึ่งพา Approximation ต่าง ๆ มากมายเพื่อเข้ามาช่วยทำให้การคำนวณระบบเคมีแบบต่าง ๆ
นั้นถูกต้องหรือที่เราเรียกว่าการทำ Correction ถ้าหากต้องการรายละเอียดที่ครอบคลุมผมแนะนำให้ทุกคนอ่านบทความรีวิวของ Prof. Kieron Burke
\enquote{Perspective on Density Functional Theory} ซึ่งสรุปไว้ดีมาก ๆ (อ่านได้ฟรี)
ลิงก์: \url{https://pubs.aip.org/aip/jcp/article/136/15/150901/941589}

\paragraph{Density Functional Theory}

เริ่มด้วยการสรุป DFT คร่าว ๆ ก่อน ตัว DFT ที่เราใช้กันอยู่ในปัจจุบันนั้นเป็น Kohn-Sham (KS) Framework ซึ่งจะใช้อ้างอิงกับระบบที่อิเล็กตรอนนั้น%
ไม่มีอันตรกิริยาต่อกัน หรือที่เราเรียกว่า Non-Interacting System ซึ่งจะมี Electron Density ที่เท่ากันกับของ Interacting System
ส่วนพลังงานของระบบที่สภาวะพื้นที่ได้จากการคำนวณด้วย DFT จะหามาจาก Electron Density (ผมเขียนแทนด้วยตัว $p$) โดยมีสมการดังนี้

\begin{equation}
    E_{tot}[p] = T_{s}[p] + E_{ext}[p] + E_{H}[p] + E_{XC}[p]
\end{equation}

\noindent โดยแต่ละเทอมคือ
\begin{itemize}
    \item $E_{tot}$ คือ Total energy
    \item $T_{s}$ คือ Kinetic energy
    \item $E_{ext}$ คือ External energy
    \item $E_{H}$ คือ Coulomb energy
    \item $E_{XC}$ คือ Exchange-correlation energy
\end{itemize}

DFT ของ KS Framework ใช้ KS Orbitals ในการนำมาสร้าง KS Wavefunction ซึ่ง KS Orbitals นั้นจะถูกเขียนด้วย KS Determinant
แล้วเราก็สามารถเขียน Electron Density ให้อยู่ในรูปของ KS Determinant ได้อีกด้วย เรามาดูรายละเอียดเฉพาะ Kinetic Energy กับ
$E_{XC}$ กัน โดยเฉพาะเทอม $E_{XC}$ นั้นจะซับซ้อนกว่าเพื่อนเพราะว่ายังไม่มี Exact Form ที่สามารถคำนวณได้อย่างถูกต้อง 100%

อันแรกคือ Kinetic Energy ($T$) ซึ่งพลังงานจลน์นี้เป็นฟังก์ชันที่ขึ้นกับ Electron Density โดยสามารถหาได้จากการใช้ Kinetic Energy
Operator ซึ่งก็คือ Laplacian\footnote{Laplacian คือ Divergence ของ Gradient อีกทีหนึ่ง ถ้าหากว่าเราพิจารณากรณีที่ระบบมีการเคลื่้อนที่ใน
    1 มิติ เราจะได้ว่าจริง ๆ แล้ว Laplacian ก็คือ Hessian หรืออนุพันธ์อันดับที่สองเทียบกับ Displacement นั่นเอง}

อันที่สองคือ $E_{XC}$ โดยเทอมนี้นั้นมีสมการดังต่อไปนี้

\begin{equation}
    E_{XC}[p] = T[p]- T_{s}[p] + E_{ee}[p] - E_{H}[p]
\end{equation}

สาเหตุที่ผมใส่ $[p]$ ในสมการข้างบนนี้ก็เพราะว่าต้องการจะบอกว่าเทอมทุกเทอมในสมการนี้ขึ้นกับ Electron Density ($p$), แล้วก็ $T$ กับ
$E_{ee}$ นั้นคือ Kinetic Energy และ Electron-Electron Energy ของระบบ Interacting System, ส่วน $T_{s}$ กับ $E_{H}$
นั้นคือ Kinetic Energy และ Coulomb Energy ของระบบ Non-Interacting System จะเห็นได้ว่าในการคำนวณหา $E_{XC}$ นั้นเราจะต้องรู้
Kinetic Energy ของทั้ง Non-Interacting และ Interacting Systems

\paragraph{Density Matrix Functional Theory}

ผมขอเปรียบเทียบ DMFT กับ DFT โดยการเทียบสมการพลังงานให้เห็นกันชัด ๆ ไปเลยว่าทั้งสองทฤษฎีต่างกันยังไง เงื่อนไขแรกที่เราพิจารณานั้นก็คือว่า
DMFT นั้นจะอ้างอิงกับระบบแบบ Interacting System (ไม่เหมือนกับ DFT) ส่วนพลังงานรวมหรือ $E_{tot}$ ของ DMFT นั้นสามารถเขียนได้%
เหมือนกันกับกรณีพลังงานรวมของ DFT นั่นแหละครับ แต่จะต่างกันตรงที่ว่าเทอมทุกเทอมนั้นไม่ได้ขึ้นกับ Electron Density อีกต่อไปแล้ว
แต่ว่าเราจะใช้สิ่งที่เรียกว่า One-electron Reduced Density Matrix (ผมใช้แทนด้วยตัว $y$) แทนตัว Density ดังนี้

\begin{equation}
    E_{tot}[y] = T[y] + E_{ext}[y] + E_{H}[y] + E_{XC}[y]
\end{equation}

ก่อนที่จะอธิบายต่อไป ต้องมาทำความเข้าใจ One-electron Reduced Density Matrix (1-RDM) กันสักนิดนึงก่อน จริง ๆ แล้วนั้น 1-RDM
ถูกพิสูจน์มาจากกลศาสตร์ควอนตัมซึ่งในทางฟิสิกส์นั้นเราเริ่มด้วยการ Quantum State ของระบบของเราโดยใช้ Pure State กับ Mixed State
(Mixed State คือ Pure State มากกว่าหนึ่งอันมารวมกัน) ซึ่ง Density Matrix นั้นก็คือ Matrix ของ Density ที่เกิดจากผลคูณระหว่าง
Probability กับฟังก์ชันคลื่นของ Pure State

ซึ่ง 1-RDM นั้นถูกลดรูปมาจาก Two-electron Reduced Density Matrix (2-RDM) รายละเอียดเพิ่มเติมอ่านได้ที่หน้าที่ 3 ของเอกสาร
\enquote{An Introduction to Reduced Density Matrix Functional Theory}\footnote{ไฟล์ PDF:
    \url{https://quantique.u-strasbg.fr/ISTPC/lib/exe/fetch.php?media=istpc2021:lecture_rdmft_pina_romaniello.pdf}
    และวิดีโอ \url{https://www.youtube.com/watch?v=HN3fXcDCytA}}

เนื่องจากว่า DMFT นั้นถูกพัฒนาขึ้นมาโดยใช้ Interacting System ดังนั้นเราจึงไม่สามารถคำนวณ 1-RDM หรือ $y$ ได้จาก KS Determinant
เหมือนกรณี KS DFT ที่เราใช้ Non-Interacting System ซึ่งนี่เป็นสาเหตุที่ทำให้ตัว DMFT นั้นมีความซับซ้อนกว่า DFT เยอะมาก ๆ

เนื่องจากว่าเราใช้ Determinant ของ KS Orbitals ไม่ได้แล้ว เราจึงจำเป็นต้องใช้ฟังก์ชันคลื่นแบบตรง ๆ ไปเลย แล้วทำการ Diagonalization
ซึ่งทำให้เราได้ว่า Spectral Representation ของ 1-RDM นั้นหาได้จาก Natural Orbitals กับ Natural Occupation Number

เมื่อเรากำหนด 1-RDM ได้แล้ว ทำให้เราได้ว่า Kinetic Energy ของ DMFT นั้นสามารถเขียนให้อยู่ในรูปของ 1-RDM ได้ และ Kinetic Energy
อันใหม่นี้นั้นก็เป็น Kinetic Energy ของระบบ Interacting System นั่นจึงทำให้ $E_{XC}$ ของ DMFT นั้นไม่ขึ้นกับ Kinetic Energy อีกต่อไป
แต่จะขึ้นอยู่กับเพียงแค่ Electron-Electron Interaction Energy เท่านั้น ดังนี้

\begin{equation}
    E_{XC}[y] = E_{ee}[y] - E_{H}[y]
\end{equation}

\noindent ซึ่งถ้าเทียบกับ $E_{XC}[p]$ ของกรณี DFT นั้นจะมีความง่ายกว่าเยอะ เมื่อเรามีนิยามที่แน่นอนของ $E_{XC}$ สำหรับ DMFT แล้ว
เราจึงสามารถพัฒนา Approximation ต่าง ๆ สำหรับ $E_{XC}$ ได้ เหมือนกันกับ DFT ที่เรามี Functionals ต่าง ๆ ให้เลือกใช้นั่นแหละ
เพียงแค่ว่า Functionals ที่เรารู้จักกันใน DFT นั้นไม่สามารถนำมาใช้กับ DMFT ได้

ส่วนการหาค่าพลังงานต่ำสุดของระบบโมเลกุลที่สภาวะพื้นนั้นก็ทำได้โดยการ Minimize ค่า $E_{tot}[y]$ โดยเทียบกับ 1-RDM $[y]$ แทนที่จะเทียบกับ
Electron Density $[p]$ นอกจากนี้เรายังสามารถคำนวณ Electron Density ของระบบจาก DMFT ได้ด้วย โดยการใช้ Occupation Number
กับฟังก์ชันคลื่นนั่นเอง

ตัวทฤษฎี DMFT นั้นถึงแม้ว่าในปัจจุบันนั้นจะยังไม่ค่อยได้ถูกใช้งานแพร่หลายในงานวิจัยทั่ว ๆ ไปมากนัก เนื่องจากว่ายังไม่ค่อย General Purpose
สักเท่าไหร่ ทำให้ยังไม่ค่อยเป็นที่นิยมเมื่อเทียบกับ DFT แต่ถ้าหากว่าถึงวันที่่ DMFT นั้นสามารถนำมาใช้งานได้ง่ายขึ้น นั่นก็อาจจะเรียกได้ว่าเป็นการ%
เปลี่ยนแปลงครั้งยิ่งใหญ่ (Paradigm Shift) เลยก็ว่าได้
