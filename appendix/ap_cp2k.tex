% LaTeX source for ``Algorithms for Computer Simulation of Molecular Systems''
% Copyright (c) 2023 รังสิมันต์ เกษแก้ว (Rangsiman Ketkaew).

% License: Creative Commons Attribution-NonCommercial-NoDerivatives 4.0 International (CC BY-NC-ND 4.0)
% https://creativecommons.org/licenses/by-nc-nd/4.0/

\chapter{ทฤษฎีและโครงสร้างของโปรแกรม CP2K}
\label{ap:cp2k}

%----------------------------------------
\section{Hamiltonian และ Energy Functional}
%----------------------------------------

CP2K เป็นโปรแกรมเคมีควอนตัมที่มีวัตถุประสงค์ในการจำลองระบบโมเลกุลแบบ Periordic (โดยการใช้ Periodic Boundary Condition) โดยระบบโมเลกุลที่เหมาะกับการจำลองประเภทนี้นั้นก็จะเป็นโมเลกุลจำพวก 2D และ 3D เช่น Solid State, Liquid, Material, Crystal รวมถึงระบบโมเลกุลที่มีขนาดใหญ่ หัวใจสำคัญของ CP2K ก็คือการใช้วิธี Gaussian and Plane Wave (GPW) ซึ่งเป็นการใช้ Representation 2 แบบที่แตกต่างกันในการอธิบายความหนาแน่นของอิเล็กตรอน (Electron Density หรือ $n(\boldsymbol{r})$) ซึ่งเป็นอินพุตสำคัญของวิธี Density Functional Theory (DFT) 

Representation อันแรกที่ใช้นั้นผู้อ่านทุกคนน่าจะทราบกันเป็นว่าอย่าง นั่นก็คือการใช้ Gaussian Functions โดนเราจะทำการกระจายหรือ Expand Gaussian Functions อันนี้ไปยังอะตอมทุก ๆ อะตอมในโมเลกุลแล้วก็ทำการรวม (Sum) ทุกเทอมเข้าด้วยกัน ซึ่งเขียนเป็นสมการได้ดังนี้ 
%
\begin{equation}
    n(\boldsymbol{r})
    =
    \sum_{\mu v} P^{\mu v} \varphi_\mu(\boldsymbol{r}) \varphi_v(\boldsymbol{r})
\end{equation}
%
โดยที่ $P^{\mu v}$ คือสมาชิกของ Density Matrix และ $\varphi_\mu(\boldsymbol{r}) = \sum_i d_{i \mu} g_i(\boldsymbol{r})$ โดยที่มี Primitive Gaussian Functions คือ $g_i(\boldsymbol{r})$ และมี Contraction Coefficients คือ $d_{i \mu}$ 

ส่วน Representation อันที่สองนั้นเป็นการใช้ประโยชน์ของ Auxiliary Basis ของ Plane Waves ดังนี้
%
\begin{equation}
    \tilde{n}(\boldsymbol{r})
    =
    \frac{1}{\Omega} \sum_{\boldsymbol{G}} \tilde{n}(\boldsymbol{G}) 
    \exp (\mathrm{i} \boldsymbol{G} \cdot \boldsymbol{r})
\end{equation}
%
โดยที่ $\Omega$ คือปริมาตรของ Unit Cell (ขนาดของระบบที่เรากำหนด) และ $\boldsymbol{G}$ คือ Reciprocal Lattice Vectors ส่วน Expansion Coefficients $\tilde{n}(\boldsymbol{G})$ นั้นมีหน้าที่คือทำให้ $\tilde{n}(\boldsymbol{r})$ นั้นมีค่าเท่ากันกับ $n(\boldsymbol{r})$ บน Regular Grid ใน Unit Cell นอกจากนี้แล้วยังมีเทคนิคที่เราสามารถทำการแปลงสลับไปมาระหว่าง $n(\boldsymbol{r})$, $\tilde{n}(\boldsymbol{r})$ และ $\tilde{n}(\boldsymbol{G})$ โดยการใช้เทคนิคการ Mapping และ Fast Fourier Transforms (FFT)

พลังงานรวมของโมเลกุลที่คำนวณด้วยวิธี Kohn-Sham Density Functional Theory (DFT) โดยการใช้ Gaussian Plane Wave (GPW) ในโปรแกรม CP2K นั้นมีสมการดังต่อไปนี้
%
\begin{equation} 
    \label{eq:e_total_cp2k}
    E[n] 
    = 
    E^{\mathrm{T}}[n]
    + E^{\mathrm{V}}[n] 
    + E^{\mathrm{H}}[n]
    + E^{\mathrm{XC}}[n]
    + E^{\mathrm{II}}
\end{equation}
%
โดยที่สมการ \eqref{eq:e_total_cp2k} นั้นเป็นสูตรของพลังงานรวมที่เกิดจากการนำพลังงานต่าง ๆ มารวมกัน อาจจะเปรียบเทียบสูตรพลังงานกับสูตรการทำอาหารก็ได้ ซึ่งพลังงานแต่ละเทอมนั้นก็คือส่วนผสมที่โปรแกรม CP2K นั้นเลือกใช้ (ตามทฤษฎี) ในทำนองเดียวกัน โปรแกรมเคมีควอนตัมอื่น ๆ นั้นต่างก็มีสูตรที่ใช้ในการคำนวณพลังงานรวมของโมเลกุลเป็นของตัวเอง ขึ้นกับว่าใช้ทฤษฎีไหน สำหรับพลังงานแต่ละเทอมในสมการที่ \eqref{eq:e_total_cp2k} นั้นมี Expression ดังนี้
%
\begin{align}
    \label{eq:e_total_cp2k_full}
    E[n] 
    =&
    \sum_{\mu \nu} P^{\mu \nu}\left\langle\varphi_\mu(\boldsymbol{r})\right| \nonumber
    - \frac{1}{2} \nabla^2\left|\varphi_\nu(\boldsymbol{r})\right\rangle \nonumber \\
    &+ \sum_{\mu \nu} P^{\mu \nu}\left\langle\varphi_\mu(\boldsymbol{r})
    \left|V_{\mathrm{loc}}^{\mathrm{PP}}(r)\right| 
    \varphi_\nu(\boldsymbol{r})\right\rangle \nonumber \\
    &+ \sum_{\mu \nu} P^{\mu \nu}\left\langle\varphi_\mu(\boldsymbol{r})
    \left|V_{\mathrm{nl}}^{\mathrm{PP}}\left(\boldsymbol{r}, \boldsymbol{r}^{\prime}\right)\right| 
    \varphi_\nu\left(\boldsymbol{r}^{\prime}\right)\right\rangle \nonumber \\ 
    &+ 2 \pi \Omega \sum_G \frac{\widetilde{n}^*(\boldsymbol{G}) \tilde{n}(\boldsymbol{G})}{\boldsymbol{G}^2} \nonumber \\ 
    &+ \int e^{\mathrm{xc}}(\boldsymbol{r}) \mathrm{d} \boldsymbol{r} \nonumber \\ 
    &+ \frac{1}{2} \sum_{I \neq J} \frac{Z_I Z_J}{\left|\boldsymbol{R}_I-\boldsymbol{R}_J\right|}
\end{align}

%----------------------------------------
\section{การพัฒนาโปรแกรม CP2K}
%----------------------------------------

\begin{enumerate}[topsep=0pt,noitemsep]
    \setlength\itemsep{0.5em}
    \item \textbf{หลักการทางทฤษฎีและวิธีการคำนวณ}
    \begin{itemize}
        \item \textbf{DFT (Density Functional Theory):} CP2K ใช้ DFT เป็นหลักในการคำนวณโครงสร้างอิเล็กทรอนิกส์ ซึ่งเป็นวิธีการที่ได้รับความนิยมอย่างมากในการศึกษาคุณสมบัติของวัสดุและโมเลกุล
        \item \textbf{GPW (Gaussian and Plane Waves) Method:} นี่คือจุดเด่นที่ทำให้ CP2K มีประสิทธิภาพสูง วิธีการนี้รวมข้อดีของ Basis Set แบบ Gaussian (ที่เหมาะกับการคำนวณใน Real-space) และ Plane Waves (ที่เหมาะกับการคำนวณใน Reciprocal-space) เข้าไว้ด้วยกัน ทำให้สามารถคำนวณได้อย่างแม่นยำและรวดเร็ว โดยเฉพาะสำหรับระบบขนาดใหญ่ที่มีความซับซ้อน
        \item \textbf{Mixed QM/MM (Quantum Mechanics/Molecular Mechanics):} CP2K มีความสามารถในการจำลองระบบที่ซับซ้อนมาก เช่น ระบบชีวภาพ โดยแบ่งออกเป็นส่วนที่ต้องใช้ Quantum Mechanics (QM) เพื่อความแม่นยำสูง และส่วนที่ใช้ Molecular Mechanics (MM) เพื่อลดภาระการคำนวณ
        \item \textbf{Molecular Dynamics (MD):} นอกจากโครงสร้างอิเล็กทรอนิกส์แล้ว CP2K ยังสามารถทำการจำลอง MD เพื่อศึกษาการเคลื่อนที่ของอะตอมและโมเลกุลตามเวลา ซึ่งช่วยให้เข้าใจพฤติกรรมของระบบในสภาวะต่างๆ เช่น อุณหภูมิและความดัน
        \item \textbf{สถิติควอนตัม (Quantum Statistics):} CP2K ยังรองรับการจำลองที่มีการพิจารณาผลกระทบทางควอนตัมของนิวเคลียส (nuclear quantum effects) ผ่านวิธีการต่างๆ เช่น Path Integral Molecular Dynamics (PIMD)
    \end{itemize}

    \item \textbf{โครงสร้างและภาษาโปรแกรม}
    \begin{itemize}
        \item \textbf{Fortran 2008:} CP2K ถูกเขียนด้วยภาษา Fortran 2008 ซึ่งเป็นภาษาที่เหมาะสมอย่างยิ่งสำหรับการคำนวณทางวิทยาศาสตร์และวิศวกรรมที่ต้องการประสิทธิภาพสูง
        \item \textbf{Modular Design:} โค้ดของ CP2K มีโครงสร้างแบบโมดูลาร์ (modular) ทำให้ง่ายต่อการพัฒนา เพิ่มฟังก์ชันใหม่ๆ และบำรุงรักษา
        \item \textbf{Quickstep and FIST:} นี่คือชื่อของส่วนประกอบหลักบางส่วนใน CP2K ที่รับผิดชอบการคำนวณโครงสร้างอิเล็กทรอนิกส์และกลศาสตร์โมเลกุลตามลำดับ
    \end{itemize}

    \item \textbf{การเพิ่มประสิทธิภาพและการประมวลผลแบบขนาน}
    \begin{itemize}
        \item \textbf{MPI (Message Passing Interface):} CP2K ได้รับการออกแบบมาให้ทำงานได้อย่างมีประสิทธิภาพสูงบนระบบคอมพิวเตอร์ประสิทธิภาพสูง (HPC) โดยใช้ MPI สำหรับการสื่อสารระหว่างโปรเซสเซอร์ที่กระจายอยู่
        \item \textbf{Multi-threading (OpenMP) และ CUDA:} นอกจาก MPI แล้ว CP2K ยังรองรับการประมวลผลแบบ Multi-threading (เช่น OpenMP) และการใช้ GPU (CUDA) เพื่อเพิ่มความเร็วในการคำนวณให้มากยิ่งขึ้น
        \item \textbf{การจัดการหน่วยความจำ:} มีการปรับปรุงการใช้หน่วยความจำอย่างต่อเนื่องเพื่อให้สามารถจัดการกับระบบขนาดใหญ่มากๆ ได้
    \end{itemize}
\end{enumerate}
